% -*- TeX -*-
\documentclass[aspectratio=169]{beamer}

\usepackage{listings}
\usepackage{amsmath}
\usepackage{bm}
\usepackage{mathtools} % equations
\usepackage{tikz}
\usetikzlibrary{shapes,calc}

\title{Introduction to PyLith v3.0}
\subtitle{}
\author{Brad Aagaard}
\institute{\includegraphics[scale=0.4]{../../logos/cig_blackfg}}
\date{June 10, 2019}


% ---------------------------------------------------- CUSTOMIZATION
\usetheme{CIG}
% Style information for PyLith presentations.

% Colors
\definecolor{ltorange}{rgb}{1.0, 0.74, 0.41} % 255/188/105
\definecolor{orange}{rgb}{0.96, 0.50, 0.0} % 246/127/0

\definecolor{ltred}{rgb}{1.0, 0.25, 0.25} % 255/64/64
\definecolor{red}{rgb}{0.79, 0.00, 0.01} % 201/0/3

\definecolor{ltpurple}{rgb}{0.81, 0.57, 1.00} % 206/145/255
\definecolor{purple}{rgb}{0.38, 0.00, 0.68} % 97/1/175

\definecolor{ltblue}{rgb}{0.2, 0.73, 1.0} % 51/187/255
\definecolor{mdblue}{rgb}{0.28, 0.50, 0.80} % 72/128/205
\definecolor{blue}{rgb}{0.12, 0.43, 0.59} % 30/110/150

\definecolor{ltltgreen}{rgb}{0.7, 1.00, 0.7} % 96/204/14
\definecolor{ltgreen}{rgb}{0.37, 0.80, 0.05} % 96/204/14
\definecolor{green}{rgb}{0.23, 0.49, 0.03} % 59/125/8
  
\definecolor{dkslate}{rgb}{0.18, 0.21, 0.28} % 47/53/72
\definecolor{mdslate}{rgb}{0.45, 0.50, 0.68} % 114/127/173
\definecolor{ltslate}{rgb}{0.85, 0.88, 0.95} % 216/225/229


\newcommand{\includefigure}[2][]{{\centering\includegraphics[#1]{#2}\par}}
\newcommand{\highlight}[1]{{\bf\usebeamercolor[fg]{structure}#1}}
\newcommand{\important}[1]{{\color{red}#1}}
\newcommand{\issue}[2]{\item[Issue:] {\color{red}#1}\\{\item[Soln:] \color{blue}#2}\\[4pt]}

\setbeamercolor{alerted text}{fg=ltgreen}
\setbeamertemplate{description item}[align left]


\newcommand{\lhs}[1]{{\color{blue}#1}}
\newcommand{\rhs}[1]{{\color{red}#1}}
\newcommand{\annotateL}[2]{%
  {\color{blue}\underbrace{\color{blue}#1}_{\color{blue}\mathclap{#2}}}}
\newcommand{\annotateR}[2]{%
  {\color{red}\underbrace{\color{red}#1}_{\color{red}\mathclap{#2}}}}
\newcommand{\eqnannotate}[2]{%
  {\color{blue}%
  \underbrace{\color{black}#1}_{\color{blue}\mathclap{#2}}}}

\newcommand{\trialvec}[1][]{{\vec{\psi}_\mathit{trial}^{#1}}}
\newcommand{\trialscalar}[1][]{{\psi_\mathit{trial}^{#1}}}
\newcommand{\basisvec}[1][]{{\vec{\psi}_\mathit{basis}^{#1}}}
\newcommand{\basisscalar}[1][]{{\psi_\mathit{basis}^{#1}}}

\newcommand{\tensor}[1]{\bm{#1}}
\DeclareMathOperator{\Tr}{Tr}

\usefonttheme[onlymath]{serif}

% minted shortcuts
\newminted{cfg}{bgcolor=ltslate,autogobble,fontsize=\tiny}
\newminted{bash}{bgcolor=ltltgreen,autogobble,fontsize=\tiny}

% PyLith components
\newcommand{\pylith}[1]{{\ttfamily\color{magenta}#1}}



% --------------------------------------------------------- DOCUMENT
\begin{document}

% ------------------------------------------------------------ SLIDE
\maketitle


% ========================================================== SECTION
\section{Introduction}

% ------------------------------------------------------------ SLIDE
\begin{frame}
  \frametitle{PyLith}
  \summary{A modern, community-driven code for crustal deformation modeling}
  
  \begin{itemize}
  \item Developers
    \begin{itemize}
    \item Brad Aagaard (USGS)
    \item Matthew Knepley (Rice University)
    \item Charles Williams (GNS Science)
    \end{itemize}
  \item Combined dynamic modeling capabilities of EqSim (Aagaard) with
    the quasi-static modeling capabilities of Tecton (Williams)
  \item Use modern software engineering  to develop an open-source,
    community code 
    \begin{itemize}
    \item Modular design
    \item Testing
    \item Documentation
    \item Distribution
    \end{itemize}
  \item PyLith v1.0 was released in 2007
  \end{itemize}

  \vspace*{-2.5cm}\hspace{0.5\textwidth}\includefigure[height=3.0cm]{figs/downloads}

\end{frame}


% ------------------------------------------------------------ SLIDE
\begin{frame}
  \frametitle{Crustal Deformation Modeling}
  \summary{Elasticity problems where geometry does not change significantly}

  \vfill
  Quasi-static modeling associated with earthquakes
  \vfill

  \begin{itemize}
  \item Strain accumulation associated with interseismic deformation
    \begin{itemize}
    \item What is the stressing rate on faults X and Y?
    \item Where is strain accumulating in the crust?
    \end{itemize}
  \item Coseismic stress changes and fault slip
    \begin{itemize}
    \item What was the slip distribution in earthquake A?
    \item How did earthquake A change the stresses on faults X and Y?
    \end{itemize}
  \item Postseismic relaxation of the crust
    \begin{itemize}
    \item What rheology is consistent with observed postseismic deformation?
    \item Can aseismic creep or afterslip explain the deformation?
    \end{itemize}
  \end{itemize}
  \vfill

\end{frame}


% ------------------------------------------------------------ SLIDE
\begin{frame}
  \frametitle{Crustal Deformation Modeling}
  \summary{Elasticity problems where geometry does not change significantly}

  \vfill
  Dynamic modeling associated with earthquakes
  \vfill

  \begin{itemize}
  \item Modeling of strong ground motions
    \begin{itemize}
    \item Forecasting the amplitude and spatial variation in ground
      motion for scenario earthquakes
    \end{itemize}
  \item Coseismic stress changes and fault slip
    \begin{itemize}
    \item How did earthquake A change the stresses on faults X and Y?
    \end{itemize}
  \item Earthquake rupture behavior
    \begin{itemize}
    \item What fault constitutive models/parameters are consistent
      with the observed rupture propagation in earthquake A?
    \end{itemize}
  \end{itemize}
  \vfill

\end{frame}


% ------------------------------------------------------------ SLIDE
\begin{frame}
  \frametitle{Crustal Deformation Modeling}
  \summary{Elasticity problems where geometry does not change significantly}

  \vfill
  Volcanic deformation associated with magma chambers and/or dikes
  \begin{itemize}
  \item Inflation
    \begin{itemize}
    \item What is the geometry of the magma chamber?
    \item What is the potential for an eruption?
    \end{itemize}
  \item Eruption
    \begin{itemize}
    \item Where is the deformation occurring?
    \item What is the ongoing potential for an eruption?
    \end{itemize}
  \item Dike intrusions
    \begin{itemize}
    \item What is the geometry of the intrusion?
    \item What is the pressure change and/or amount of opening/dilatation?
    \end{itemize}
  \end{itemize}
  \vfill

\end{frame}


% ------------------------------------------------------------ SLIDE
\begin{frame}
  \frametitle{Crustal Deformation Modeling}
  \summary{Overview of workflow for typical research problem}

  \includefigure[height=7.2cm]{figs/workflow}
  
\end{frame}


% ========================================================== SECTION
\section{PyLith v3.0}

% ------------------------------------------------------------ SLIDE
\begin{frame}
  \frametitle{PyLith v3.0}
  \summary{}

  \begin{itemize}
  \item Multiphysics formulation through point-wise integration kernels
  \item Higher order spatial and temporal discretizations
  \item Adaptive time stepping via PETSc TS
  \item Improved fault formulation for spontaneous rupture (coming in
    v3.1)
  \item Many other small changes
  \end{itemize}

\end{frame}

\abovedisplayskip=2pt
\abovedisplayshortskip=2pt
\belowdisplayskip=2pt
\belowdisplayshortskip=2pt

% ------------------------------------------------------------ SLIDE
\begin{frame}
  \frametitle{Aside: Finite-Element Method}
  \summary{Strong form to weak form}

  Solve governing equation in integrated sense:
  \begin{equation}
    \int_\Omega \trialscalar \cdot \mathit{PDE} \, d\Omega = 0,
  \end{equation}
  by minimizing the error with respect to the unknown coefficients.
  \vfill
  This leads to equations of the form:
  \begin{equation}
    \int_\Omega \trialscalar \cdot f_0(x,t) + \nabla \trialscalar \cdot f_1(x,t) \, d\Omega = 0.
  \end{equation}

\end{frame}

% ------------------------------------------------------------ SLIDE
\begin{frame}
  \frametitle{Governing Equations}
  \summary{}

  We want to solve equations in which the weak form can be expressed
  as
  \begin{gather}
    \lhs{F(t,s,\dot{s})} = \rhs{G(t,s)} \\
    s(t_0) = s_0
  \end{gather}
  where $\lhs{F}$ and $\rhs{G}$ are vector functions, $t$ is time, and $s$ is the solution vector.
  \vfill
  Using the finite-element method and divergence theorem, we cast the weak form into
  \begin{multline}
    \int_\Omega \trialvec \cdot \lhs{\vec{f}_0(t,s,\dot{s})} + \nabla \trialvec : \lhs{\tensor{f}
    _1(t,s,\dot{s})} \, 
    d\Omega = \\
    \int_\Omega \trialvec \cdot \rhs{\vec{g}_0(t,s)} + \nabla \trialvec : \rhs{\tensor{g}_1(t,s)} \, 
    d\Omega,
  \end{multline}
  where $\vec{f}_0$ and $\vec{g}_0$ are vectors, and $\tensor{f}_1$ and
  $\tensor{g}_1$ are tensors.

\end{frame}

% ------------------------------------------------------------ SLIDE
\begin{frame}
  \frametitle{Explicit Time Stepping}
  \summary{}

  Explicit time stepping with the PETSc TS requires $\lhs{F(t,s,\dot{s})} = \dot{s}$.
  \vfill
  Normally $\lhs{F(t,s,\dot{s})}$ contains the inertial term ($\rho \ddot{u}$).
  \vfill
  Therefore, we transform our equation into the form:
  \begin{align}
    \lhs{F^*(t,s,\dot{s})} = \lhs{\dot{s}} &= \rhs{G^*(t,s)} \\
    \lhs{\dot{s}} &= \lhs{M^{-1}}\rhs{G(t,s)}.
  \end{align}  

\end{frame}

% ------------------------------------------------------------ SLIDE
\begin{frame}
  \frametitle{Solving the Equations}
  \summary{Explicit time stepping requires a subset of the terms used in implicit time stepping.}

  \begin{itemize}
  \item PETSc TS object provides time-stepping and solver implementations
    \begin{itemize}
    \item Application code provides functions for computing RHS and LHS residuals and Jacobians
    \end{itemize}
  \item Explicit time stepping
    \begin{itemize}
    \item Compute RHS residual, $\rhs{G(t,s)}$
    \item Compute lumped inverse of LHS, $\lhs{M^{-1}}$
    \item No need to compute LHS residual, because $\lhs{F(t,s,\dot{s})} = \dot{s}$
    \end{itemize}
  \item Implicit time stepping (Krylov solvers)
    \begin{itemize}
    \item Compute RHS residual, $\rhs{G(t,s)}$
    \item Compute LHS residual, $\lhs{F(t,s,\dot{s})}$
    \item Compute RHS Jacobian, $\rhs{J_G}$ = $\rhs{\frac{\partial G}{\partial s}}$
    \item Compute LHS Jacobian, $\lhs{J_F}$ = $\lhs{\frac{\partial F}{\partial s}} + s_\mathit{tshift}\lhs{\frac{\partial F}{\partial \dot{s}}}$
    \end{itemize}
  \end{itemize}

\end{frame}

% ========================================================== SECTION
\section{Governing Equations}
\subsection{Elasticity}

% ------------------------------------------------------------ SLIDE
\begin{frame}
  \frametitle{Example: Elasticity with Prescribed Slip}
  \summary{Use domain decomposition and Lagrange multipliers to
    prescribe slip}

  \highlight{Implicit time stepping without inertia}
  
\begin{align}
  \vec{s}^T &= (\vec{u} \quad \vec{\lambda})^T, \\
%
  \vec{0} &= \vec{f}(\vec{x},t) + \tensor{\nabla} \cdot \tensor{\sigma}(\vec{u}) \text{ in }
\Omega, \\
% Neumann
  \tensor{\sigma} \cdot \vec{n} &= \vec{\tau}(\vec{x},t) \text{ on }\Gamma_\tau, \\
% Dirichlet
  \vec{u} &= \vec{u}_0(\vec{x},t) \text{ on }\Gamma_u, \\
% Fault
  \vec{0} &= \vec{d}(\vec{x},t) - \vec{u}^+(\vec{x},t) + \vec{u}^-(\vec{x},t) \text{ on }\Gamma_f, \\
  \tensor{\sigma} \cdot \vec{n} &= -\vec{\lambda}(\vec{x},t) \text{ on }\Gamma_{f^+}, \\
  \tensor{\sigma} \cdot \vec{n} &= +\vec{\lambda}(\vec{x},t) \text{ on }\Gamma_{f^-}.
\end{align}

\end{frame}


% ------------------------------------------------------------ SLIDE
\begin{frame}
  \frametitle{Example: Elasticity with Prescribed Slip (cont.)}
  \summary{}

We create the weak form by taking the dot product with the trial
function $\trialvec[u]$ or $\trialvec[\lambda]$ and integrating over the domain:
\begin{align}
  0 &= \int_\Omega \trialvec[u] \cdot \left( \vec{f}(t) + \tensor{\nabla} \cdot \tensor{\sigma} (\vec{u}) \right) \, d\Omega, \\
  0 &= \int_{\Gamma_f} \trialvec[\lambda] \cdot \left( \vec{d}(\vec{x},t) - \vec{u}^+(\vec{x},t) + \vec{u}^-(\vec{x},t) \right) \, d\Gamma.
\end{align}

Using the divergence theorem and incorporating the Neumann boundary and fault interface
conditions, we can rewrite the first equation as
\begin{equation}
  % u+
  \begin{split}
  0 = \int_\Omega \trialvec[u] \cdot \vec{f}(t) + \nabla
  \trialvec[u] : -\tensor{\sigma}
  (\vec{u}) \, d\Omega
  + \int_{\Gamma_\tau} \trialvec[u] \cdot \vec{\tau}(\vec{x},t) \,
  d\Gamma \\
  + \int_{\Gamma_{f}} \trialvec[u^+] \cdot -\vec{\lambda}(\vec{x},t) +
  \trialvec[u^-] \cdot +\vec{\lambda}(\vec{x},t)\, d\Gamma.
  \end{split}
\end{equation}

\end{frame}


% ------------------------------------------------------------ SLIDE
\begin{frame}
  \frametitle{Example: Elasticity with Prescribed Slip (cont.)}
  \summary{}

Identifying $F(t,s,\dot{s})$ and $G(t,s)$, we have
\begin{align}
  F^u(t,s,\dot{s}) &= 0, \\
  F^\lambda(t,s,\dot{s}) &= 0, \\
  G^{u}(t,s) &=  
   \int_\Omega \trialvec[u] \cdot \eqnannotate{\vec{f}(\vec{x},t)}{g_0^u}
  + \nabla \trialvec[u] :
               \eqnannotate{-\tensor{\sigma}(\vec{u})}{g_1^u} \,
               d\Omega \\
  &
  + \int_{\Gamma_\tau} \trialvec[u] \cdot \eqnannotate{\vec{\tau}(\vec{x},t)}{g_0^u} \, d\Gamma 
  + \int_{\Gamma_{f}} \trialvec[u^+] \cdot \eqnannotate{-\vec{\lambda}(\vec{x},t)}{g_0^{u^+}} + \trialvec[u^-] \cdot \eqnannotate{+\vec{\lambda}(\vec{x},t)}{g_0^{u^-}}\, d\Gamma,\\
  G^\lambda(t,s) &= 
\int_{\Gamma_{f}} \trialvec[\lambda] \cdot \eqnannotate{\left(
    \vec{d}(\vec{x},t) - \vec{u^+}(\vec{x},t) + \vec{u^-}(\vec{x},t)\right)}{g_0^\lambda} \, d\Gamma.
\end{align}

\end{frame}


% ------------------------------------------------------------ SLIDE
\begin{frame}
  \frametitle{Example: Elasticity with Prescribed Slip (cont.)}
  \summary{}

  \vspace*{-18pt}
\begin{align}
  % J_G uu
  \begin{split}
  J_G^{uu} &= \frac{\partial G^u}{\partial u} = \int_\Omega \nabla \trialvec[u] : 
\frac{\partial}{\partial u}(-
\tensor{\sigma}) \, d\Omega 
  = \int_\Omega \nabla \trialvec[u] : -\tensor{C} : \frac{1}{2}(\nabla + \nabla^T)\basisvec[u] 
             \, d\Omega \\
&\quad\quad\quad  = \int_\Omega \trialscalar[v]_{i,k} \, \eqnannotate{\left( -C_{ikjl}
    \right)}{J_{g3}^{uu}} \, \basisscalar[u]_{j,l}\,
  d\Omega \end{split} \\
  % J_G u \lambda
  \begin{split}
J_G^{u\lambda} &= \frac{\partial G^u}{\partial \lambda} =
\int_{\Gamma_{f^+}} \trialvec[u] \cdot \frac{\partial}{\partial \lambda}(-\vec{\lambda}) \, d\Gamma
+ \int_{\Gamma_{f^-}} \trialvec[u] \cdot \frac{\partial}{\partial \lambda}(+\vec{\lambda}) \, d\Gamma \\
& \quad\quad\quad = \int_{\Gamma_{f}} \trialscalar[u^+]_i \eqnannotate{-1}{J_{g0}^{u^+\lambda}}\basisscalar[\lambda]_j + \trialscalar[u^-]_i \eqnannotate{+1}{J_{g0}^{u^-\lambda}}\basisscalar[\lambda]_j \, d\Gamma
\end{split} \\
% J_G \lambda u
\begin{split}
J_G^{\lambda u} &= \frac{\partial G^\lambda}{\partial u} =
\int_{\Gamma_{f}} \trialvec[\lambda] \cdot \frac{\partial}{\partial u}\left(\vec{d}(\vec{x},t) - \vec{u^+}(\vec{x},t) + \vec{u^-}(\vec{x},t) \right) \, d\Gamma \\
&\quad\quad\quad = \int_{\Gamma_{f}} \trialscalar[\lambda]_i (\eqnannotate{-1}{J_{g0}^{\lambda u^+}})\basisscalar[u^+]_j
+ \trialscalar[\lambda]_i (\eqnannotate{+1}{J_{g0}^{\lambda u^-}})\basisscalar[u^-]_j \, d\Gamma
\end{split} \\
%
  J_G^{\lambda \lambda} &= 0
\end{align}

\end{frame}

% ========================================================== SECTION
\subsection{Incompressible Elasticity}

% ------------------------------------------------------------ SLIDE
\begin{frame}
  \frametitle{Example: Incompressible Elasticity}
  \summary{}

  \highlight{Implicit time stepping without inertia}
  \begin{gather}
    % Solution
    \vec{s}^T = \left( \vec{u} \quad \ p \right)^T, \\
    % Elasticity
    \vec{0} = \vec{f}(t) + \tensor{\nabla} \cdot \left(\tensor{\sigma}^\mathit{dev}(\vec{u}) - p\tensor{I}\right) \text{ in }\Omega, \\
    % Pressure
    0 = \vec{\nabla} \cdot \vec{u} + \frac{p}{K}, \\
    % Neumann
    \tensor{\sigma} \cdot \vec{n} = \vec{\tau} \text{ on }\Gamma_\tau, \\
    % Dirichlet
    \vec{u} = \vec{u}_0 \text{ on }\Gamma_u, \\
    p = p_0 \text{ on }\Gamma_p.
  \end{gather}

\end{frame}

% ------------------------------------------------------------ SLIDE
\begin{frame}
  \frametitle{Example: Incompressible Elasticity (cont.)}
  \summary{}

  Using trial functions $\trialvec[u]$ and $\trialscalar[p]$ and
  incorporating the Neumann boundary conditions:
  \begin{gather}
    % Displacement
  0 = \int_\Omega \trialvec[u] \cdot \vec{f}(t) + \nabla \trialvec[u] : \left(-\tensor{\sigma}^\mathit{dev}(\vec{u}) + p\tensor{I}
  \right)\, d\Omega + \int_{\Gamma_\tau} \trialvec[u] \cdot \vec{\tau}(t) \, d\Gamma, \\
  % Pressure
  0 = \int_\Omega \trialscalar[p] \cdot \left(\vec{\nabla} \cdot \vec{u} + \frac{p}{K} \right) 
\, d\Omega.
\end{gather}

Identifying $G(t,s)$, we have
\begin{gather}
  0 = \int_\Omega \trialvec[u] \cdot \eqnannotate{\vec{f}(t)}{g_0^u} + \nabla \trialvec[u] :
  \eqnannotate{\left(-\tensor{\sigma}^\mathit{dev}(\vec{u}) + p\tensor{I}\right)}{g_1^u}  \, d\Omega
  + \int_{\Gamma_\tau} \trialvec[u] \cdot \eqnannotate{\vec{\tau}(t)}{g_0^u} \, d\Gamma, \\
%
  0 = \int_\Omega \trialscalar[p] \cdot \eqnannotate{\left(\vec{\nabla} \cdot \vec{u} + 
\frac{p}{K} \right)}{g_0^p} \, d\Omega.
\end{gather}

\end{frame}

% ------------------------------------------------------------ SLIDE
\begin{frame}
  \frametitle{Example: Incompressible Elasticity (cont.)}
  \summary{}

 With two fields we have four Jacobians for the RHS associated with the coupling of 
the two fields.
\begin{align}
  J_G^{uu} &= \frac{\partial G^u}{\partial u} = \int_\Omega \nabla \trialvec[u] : 
\frac{\partial}{\partial u}(-
\tensor{\sigma}^\mathit{dev}) \, d\Omega 
  = \int_\Omega \trialscalar[u]_{i,k} \, \eqnannotate{\left(-C^\mathit{dev}_{ikjl}\right)}
{J_{g3}^{uu}}  \, 
\basisscalar[u]_{j,l}\, d\Omega \\
  J_G^{up} &= \frac{\partial G^u}{\partial p} = \int_\Omega \nabla\trialvec[u] : \tensor{I} 
\basisscalar[p] \,  d\Omega = \int_\Omega \trialscalar[u]_{i,k} \eqnannotate{\delta_{ik}}{J_{g2}^{up}} \, 
\basisscalar[p] \, d\Omega \\
%
  J_G^{pu} &= \frac{\partial G^p}{\partial u} = \int_\Omega \trialscalar[p] \left(\vec{\nabla} 
\cdot \basisvec[u]\right) \, d\Omega = \int_\Omega \trialscalar[p] \eqnannotate{\delta_{jl}}{J_{g1}^{pu}} 
\basisscalar[u]_{j,l} \, d\Omega\\
  J_G^{pp} &= \frac{\partial G^p}{\partial p} = \int_\Omega \trialscalar[p] \eqnannotate{\frac{1}
{K}}{J_{g0}^{pp}} \basisscalar[p] \, d\Omega
\end{align}

  
\end{frame}


% ========================================================== SECTION
\subsection{Incompressible Elasticity}
% ------------------------------------------------------------ SLIDE
\begin{frame}
  \frametitle{Summary of Multiphysics Implementation}
  \summary{}

  We decouple the element definition from the fully-coupled equation,
  using pointwise kernels that look like the PDE.

  \vfill    
  \begin{description}
  \item[Flexibility] The cell traversal, handled by the library,
    accommodates arbitrary cell shapes. The problem can be posed
    in any spatial dimension with an arbitrary number of
    physical fields.
  \item[Extensibility] The library developer needs to maintain only a
    single method, easing language transitions (CUDA, OpenCL). A new
    discretization scheme could be enabled in a single place in the
    code.
  \item[Efficiency] Only a single routine needs to be
    optimized. The application scientist is no longer
    responsible for proper vectorization, tiling, and other
    traversal optimization.
  \end{description}

\end{frame}


% ========================================================== SECTION
\section{Using PyLith}

% ------------------------------------------------------------ SLIDE
\begin{frame}
  \frametitle{Overview of PyLith Workflow}
  \summary{}

  \includefigure[height=7.2cm]{figs/runpylith}
\end{frame}

% ------------------------------------------------------------ SLIDE
\begin{frame}
  \frametitle{PyLith as a Hierarchy of Components}
  \summary{Components are the basic building blocks}

  \begin{minipage}{0.53\textwidth}
    \begin{itemize}
    \item Separate functionality into discrete modules (components)
    \item Alternative implementations use the same interfaces to allow plug-n-play
    \item Top-level interfaces in Python with computational code in C++
      \begin{itemize}
      \item Python dynamic typing permits adding new modules at runtime.
      \item Users can add functionality without modifying the PyLith code.
      \end{itemize}
    \end{itemize}
  \end{minipage}\hfill
  \begin{minipage}{0.43\textwidth}
    \includefigure[height=5cm]{figs/component}
  \end{minipage}

\end{frame}

% ------------------------------------------------------------ SLIDE
\begin{frame}[fragile]
  \frametitle{Parameter Files}
  \summary{Simple syntax for specifying parameters for properties and components}

\begin{cfg}
# Syntax
<h>[pylithapp.COMPONENT.SUBCOMPONENT]</h> ; Inline comment
<f>COMPONENT</f> = OBJECT
<p>PARAMETER</p> = VALUE

# Example
<h>[pylithapp.mesh_generator]</h> ; Header indicates path of mesh_generator in hierarchy
<f>reader</f> = pylith.meshio.MeshIOCubit ; Use mesh from CUBIT/Trelis
<p>reader.filename</p> = mesh_quad4.exo ; Set filename of mesh.
<p>reader.coordsys.space_dim</p> = 2 ; Set coordinate system of mesh.

<h>[pylithapp.problem.solution_outputs.output]</h> ; Set output format
<f>writer</f> = pylith.meshio.DataWriterHDF5
<p>writer.filename</p> = axialdisp.h5

<h>[pylithapp.problem]</h>
<f>bc</f> = [x_neg, x_pos, y_neg] ; Create array of boundary conditions
<f>bc.x_neg</f> = pylith.bc.DirichletTimeDependent ; Set type of boundary condition
<f>bc.x_pos</f> = pylith.bc.DirichletTimeDependent
<f>bc.y_neg</f> = pylith.bc.DirichletTimeDependent

<h>[pylithapp.problem.bc.x_pos]</h> ; Boundary condition for +x
<p>constrained_dof</p> = [0] ; Constrain x DOF
<p>label</p> = edge_xpos ; Name of nodeset from CUBIT/Trelis
<f>db_auxiliary_fields</f> = spatialdata.spatialdb.SimpleDB ; Set type of spatial database
<p>db_auxiliary_fields.label</p> = Dirichlet BC +x edge
<p>db_auxiliary_fields.iohandler.filename</p> = axial_disp.spatialdb ; Filename for database
\end{cfg}

\end{frame}

% ------------------------------------------------------------ SLIDE
\begin{frame}
  \frametitle{Parameters Graphical User-Interface}
  \summary{{\tt cd parametersgui; ./pylith\_paramviewer}}

  \includefigure[height=8.5cm]{figs/paramgui_snapshot}

\end{frame}

% ------------------------------------------------------------ SLIDE
\begin{frame}
  \frametitle{Spatial Databases}
  \summary{User-specified field/value in space for properties and BC values.}

  \begin{itemize}
 \item Examples
    \begin{itemize}
    \item Uniform value for Dirichlet BC (0-D)
    \item Piecewise linear variation in tractions for Neumann BC (1-D)
    \item SCEC CVM-H seismic velocity model (3-D)
    \end{itemize}
  \item Generally independent of discretization for problem
  \item Available spatial databases
    \begin{description}
    \item[UniformDB] Optimized for uniform value
    \item[SimpleDB] Arbitrarily distributed points for variations in 0-D, 1-D, 2-D, or 3-D
    \item[SimpleGridDB] Logically gridded points for variations in 0-D, 1-D, 2-D, or 3-D
    \item[SCECCVMH] SCEC CVM-H seismic velocity model v5.3
    \item[ZeroDB] Special case of UniformDB with zero values
    \end{description}
 \end{itemize}

\end{frame}

% ------------------------------------------------------------ SLIDE
\begin{frame}
  \frametitle{PyLith Design: Focus on Geodynamics}
  \summary{Leverage packages developed by computational scientists}

  \includefigure[height=7.0cm]{figs/packages}

\end{frame}

% ------------------------------------------------------------ SLIDE
\begin{frame}
  \frametitle{PyLith Development Follows CIG Best Practices}
  \summary{\href{https://github.com/geodynamics/best\_practices}{github.com/geodynamics/best\_practices}}

  \begin{itemize}
  \item Version Control
    \begin{itemize}
    \item New features are added in separate branches.
    \item Use 'master' branch as stable development branch.
    \end{itemize}
  \item Coding
    \begin{itemize}
    \item User-friendly specification of parameters at runtime.
    \item Development plan, updated annually.
    \item Users can add features or alternative implementations without modifying code.
    \end{itemize}
  \item Portability
    \begin{itemize}
    \item Build procedure is independent of compilers and optimization flags.
    \item Multiple builds (debug/optimized) from same source.
    \end{itemize}
  \item Documentation and User Workflow
    \begin{itemize}
    \item Extensive example suite with varying levels of complexity.
    \item Changing simulation parameters does not require rebuilding.
    \item Displays version information via {\tt --version} command line argument.
    \end{itemize}
  \end{itemize}

\end{frame}

% ------------------------------------------------------------ SLIDE
\begin{frame}
  \frametitle{Development Tools}
  \summary{Leverage open-source tools for efficient code development.}

  \begin{description}
  \item[GitHub] Code repository supporting simultaneous, independent implementation of new features.
  \item[Doxygen] Document parameters and purpose of every object and its functions.
  \item[CppUnit] Test nearly every function in code during development.
  \item[Travis CI] Run tests when code is committed to repository.
  \item[gcov] Records which lines of code tests cover.
  \end{description}

\end{frame}

% ------------------------------------------------------------ SLIDE
\begin{frame}
  \frametitle{Testing}
  \summary{Multiple levels of testing facilitates identifying bugs at origin.}

  \begin{description}
  \item[unit tests] Serial testing at level of single and multiple functions.
  \item[MMS tests] Serial testing with Method of Manufactured
    Solutions (MMS) to verify implementation of governing equations
  \item[full-scale tests] Serial and parallel pass/fail tests of full problems.
  \item[benchmarks] Serial and parallel tests for code comparisons, etc.
  \end{description}

\end{frame}

\newcommand{\statuslater}[1]{{\color{mdslate}#1}}
\newcommand{\statusdone}[1]{{\color{green}#1}}
\newcommand{\statusbuggy}[1]{{\color{orange}#1}}
\newcommand{\statusinprogress}[1]{{\color{red}#1}}
% ------------------------------------------------------------ SLIDE
\begin{frame}
  \frametitle{PyLith v3.0.0beta1 (Jun 10, 2019)}
  \summary{Incomplete, contains bugs, but can do interesting physics}

  \begin{itemize}
  \item Features (mesh importing) not changed remain stable.
  \item Some implemented features have been thoroughly tested.
  \item Some implemented features have minimal testing.
  \item A few implemented features have no testing.
  \item Several major features in v2.2 have not yet been implemented.
  \end{itemize}

\end{frame}

% ------------------------------------------------------------ SLIDE
\begin{frame}
  \frametitle{PyLith v3.0.0beta1: Governing Equations}
  \summary{}

  \highlight{Elasticity}
  \begin{itemize}
  \item \statusdone{Static and quasi-static problems}
  \item \statusinprogress{Dynamic problems (with inertia)}
  \item \statusdone{Infinitesimal strains}
  \item \statuslater{Small strain}
  \item \statusdone{Gravitational body forces}
  \item \statusdone{Body forces}
  \item Bulk rheologies (constitutive models)
    \begin{itemize}
    \item \statusdone{Isotropic, linear elasticity}
    \item \statusbuggy{Isotropic, linear Maxwell viscoelasticity}
    \item \statusbuggy{Isotropic, linear generalized Maxwell viscoelasticity}
    \item \statusbuggy{Isotropic, power-law viscoelasticity}
    \item \statusinprogress{Isotropic, Drucker-Prager elastoplasticity}
    \end{itemize}
  \end{itemize}

  \hfill%
  \begin{minipage}{0.25\textwidth}
    \statusdone{Done}\\
    \statusbuggy{Buggy}\\
    \statusinprogress{In Progress}\\
    \statuslater{Coming Later}
  \end{minipage}

  
\end{frame}

% ------------------------------------------------------------ SLIDE
\begin{frame}
  \frametitle{PyLith v3.0.0beta1: Governing Equations}
  \summary{Incomplete, contains bugs, but can do interesting physics}

  \highlight{Incompressible Elasticity}
  \begin{itemize}
  \item \statusdone{Static and quasi-static problems}
  \item \statusdone{Infinitesimal strains}
  \item \statusdone{Gravitational body forces}
  \item \statusdone{Body forces}
  \item Bulk rheologies (constitutive models)
    \begin{itemize}
    \item \statusdone{Isotropic, linear elasticity}
    \item \statuslater{Isotropic, linear Maxwell viscoelasticity}
    \item \statuslater{Isotropic, linear generalized Maxwell viscoelasticity}
    \item \statuslater{Isotropic, power-law viscoelasticity}
    \end{itemize}
  \end{itemize}

\end{frame}


% ------------------------------------------------------------ SLIDE
\begin{frame}
  \frametitle{PyLith v3.0.0beta1: Boundary and Interface Conditions}
  \summary{}

  \begin{itemize}
  \item Boundary conditions
    \begin{itemize}
    \item \statusdone{Time-dependent Dirichlet boundary conditions}
    \item \statusdone{Time-dependent Neumann (traction) boundary conditions}
    \item \statusbuggy{Absorbing boundary conditions}
    \end{itemize}
  \item Interface conditions
    \begin{itemize}
    \item \statusbuggy{Kinematic (prescribed slip) fault interfaces w/multiple ruptures}
    \item \statuslater{Dynamic (friction) fault interfaces}
      \begin{itemize}
      \item \statuslater{Static friction}
      \item \statuslater{Linear slip-weakening}
      \item \statuslater{Linear time-weakening}
      \item \statuslater{Dieterich-Ruina rate and state friction w/ageing law}
      \end{itemize}
    \end{itemize}
  \end{itemize}

\end{frame}


% ------------------------------------------------------------ SLIDE
\begin{frame}
  \frametitle{PyLith v3.0.0beta1: Other Features}
  \summary{}

  \begin{itemize}
  \item Importing meshes
    \begin{itemize}
    \item \statusdone{LaGriT: GMV/Pset}
    \item \statusdone{CUBIT/Trelis: Exodus II}
    \item \statusdone{ASCII: PyLith mesh ASCII format (intended for toy problems only)}
    \end{itemize}
  \item \statusdone{Initial conditions}
  \item Output: HDF5 and VTK files
    \begin{itemize}
    \item \statusdone{Solution over domain}
    \item \statusdone{Solution over domain boundary}
    \item \statusbuggy{Solution interpolated to user-specified points w/station
      names}
    \item \statusdone{Solution over materials and boundary conditions}
    \item \statusdone{State variables (e.g., stress and strain) for each material}
    \item \statusbuggy{Fault information (e.g., slip and tractions)}
    \end{itemize}
 \end{itemize}
  
\end{frame}


% ------------------------------------------------------------ SLIDE
\begin{frame}
  \frametitle{PyLith v3.0.0beta1: Other Features  (cont.)}
  \summary{}

  \begin{itemize}
  \item \statusdone{Automatic conversion of units for all parameters}
  \item \statusdone{Parallel uniform global refinement}
  \item \statusdone{PETSc linear and nonlinear solvers}
  \item \statusinprogress{Output of simulation progress estimates runtime}
 \end{itemize}
  
\end{frame}


% ------------------------------------------------------------ SLIDE
\begin{frame}
  \frametitle{How do changes from v2.x to v3.x affect users?}
  \summary{}

  \begin{itemize}
  \item No changes
    \begin{itemize}
    \item Meshes
    \item Formats of spatial database files
    \end{itemize}
  \item \highlight{Substantial changes}
    \begin{itemize}
    \item Parameter (cfg) files
    \item Names of values in spatial database files
    \end{itemize}
  \item HDF5 output is now the default
  \end{itemize}
  
\end{frame}


% ========================================================== SECTION
\section{CUBIT/Trelis}
\subsection{General}

% ------------------------------------------------------------ SLIDE
\begin{frame}
  \frametitle{Mesh Generation Tips}
  \summary{There is no silver bullet in finite-element mesh generation}
 
  \begin{itemize}
  \item Hex/Quad versus Tet/Tri
    \begin{itemize}
    \item Hex/Quad are slightly more accurate and faster
    \item Tet/Tri easily handle complex geometry
    \item Easy to vary discretization size with Tet, Tri, and Quad cells
    \item There is no easy answer\\
      For a given accuracy, a finer resolution Tet mesh that varies
      the discretization size in a more optimal way {\bf\it might} run
      faster than a Hex mesh
    \end{itemize}
  \item Check and double-check your mesh
    \begin{itemize}
    \item Were there any errors when running the mesher?
    \item Are the boundaries, etc marked correctly for your BC?
    \item Check mesh quality (aspect ratio should be close to 1)
    \end{itemize}
  \end{itemize}

\end{frame}


% ------------------------------------------------------------ SLIDE
\begin{frame}
  \frametitle{CUBIT/Trelis Workflow}
  \summary{}
 
  \begin{enumerate}
  \item Create geometry
    \begin{enumerate}
    \item Construct surfaces from points, curves, etc or basic shapes
    \item Create domain and subdivide to create any interior surfaces
      \begin{itemize}
      \item Fault surfaces must be interior surfaces (or a subset) that completely divide domain
      \item Need separate volumes for different constitutive {\em models, not parameters}
      \end{itemize}
    \end{enumerate}
  \item Create finite-element mesh
    \begin{enumerate}
    \item Specify meshing scheme
    \item Specify mesh sizing information
    \item Generate mesh
    \item Smooth to fix any poor quality cells
    \end{enumerate}
  \item Create nodesets and blocks
    \begin{enumerate}
    \item Create block for each constitutive model
    \item Create nodeset for each BC and fault
    \item \highlight{Create nodeset for buried fault edges}
    \item Create nodeset for ground surface for output (optional)
    \end{enumerate}
  \item Export mesh in Exodus II format ({\tt .exo} files)
  \end{enumerate}

\end{frame}


% ------------------------------------------------------------ SLIDE
\begin{frame}
  \frametitle{CUBIT/Trelis Issues}
  \summary{Keep in mind the scales of the observations you are modeling}
 
  \begin{itemize}
  \item Topography/bathymetry
    \begin{itemize}
    \item Ignore topography/bathymetry unless you know it matters
    \item For rectilinear grid, create UV net surface
    \item Convert triangular facets to UV net surface via mapped mesh
    \end{itemize}
  \item Fault surfaces
    \begin{itemize}
    \item Building surfaces from contours is usually easiest
    \item Include features at the resolution that matters
    \end{itemize}
  \item Performance
    \begin{itemize}
    \item Number of points in spline curves/surfaces has huge affect
      on mesh generation runtime
    \item CUBIT/Trelis do not run in parallel
    \item Use uniform global refinement in PyLith for large sims ($>$10M cells) 
    \end{itemize}
  \end{itemize}

\end{frame}


% ------------------------------------------------------------ SLIDE
\begin{frame}
  \frametitle{CUBIT/Trelis Best Practices}
  \summary{}
 
  \begin{description}
    \issue{Changes in geometry cause changes in object ids} {Name objects and use
      APREPRO or Python to eliminate hardwired ids wherever possible}
    \issue{Splines with many points slows down operations}{Reduce the
      number of points per spline}
    \issue{Surfaces meet in small angles creating distorted
      cells}{Trim geometry to eliminate features smaller than cell
      size}
    \issue{Difficulty meshing complex geometry with Hex cells}{Use
      Tet cells even if it requires a finer mesh}
    \issue{Hex mesh over-samples parts of the
      domain}{Use Tet mesh and vary discretization within domain}
    \issue{Extended surfaces create very complex geometry}{Subdivide
      geometry before webcutting to eliminate overly complex geometry}
  \end{description}

\end{frame}


% ========================================================== SECTION
\subsection{PyLith}

% ------------------------------------------------------------ SLIDE
\begin{frame}
  \frametitle{PyLith Tips}
  \summary{}
 
  \begin{itemize}
  \item \highlight{Read the PyLith User Manual}
  \item \highlight{Do not ignore error messages and warnings!}
  \item Use an example/benchmark as a starting point
  \item Quasi-static simulations
    \begin{itemize}
    \item Start with a static simulation and then add time dependence
    \item \highlight{Check that the solution converges at every time step}
    \end{itemize}
  \item Dynamic simulations
    \begin{itemize}
    \item Start with a static simulation
    \item \highlight{Shortest wavelength seismic waves control cell size}
    \end{itemize}
  \item CIG community forums\\
    {\tt \highlight{https://community.geodynamics.org/c/pylith}}
  \item PyLith User Resources\\
    {\small\tt \highlight{https://wiki.geodynamics.org/software:pylith:start}}
  \end{itemize}

\end{frame}


% ------------------------------------------------------------ SLIDE
\begin{frame}
  \frametitle{Getting Started}
  \summary{}

  \begin{enumerate}
  \item \highlight{Create a play area for working with examples}\\
    {\tt cd PATH\_TO\_PYLITH\_DIR} \\
    {\tt mkdir playpen} \\
    {\tt cp -r src/pylith-2.2.1rc1/examples playpen/}
  \item Work through relevant examples
  \item Try to complete relevant exercises listed in the manual
  \item Modify an example to look like your problem of interest
  \end{enumerate}

\end{frame}


% ------------------------------------------------------------ SLIDE
\begin{frame}
  \frametitle{Overview of Examples}
  \summary{Examples progress from simple to more complex}

  \begin{enumerate}
  \item 2d/box
    \begin{itemize}
    \item \statusdone{Axial compress/extension w/Dirichlet BC}
    \item \statusdone{Shearing with Dirichlet and Neumann BC}
    \end{itemize}
  \item 3d/box
    \begin{itemize}
    \item \statusdone{Same as 2d/box in 3D}
    \end{itemize}
  \item 2d/strikeslip
    \begin{itemize}
    \item \statusdone{Variable mesh size in CUBIT/Trelis}
    \item \statusbuggy{Prescribed fault slip}
    \item \statusdone{Dirichlet boundary conditions}
    \end{itemize}
  \item 2d/reverse
    \begin{itemize}
    \item \statusdone{Gravitational body forces with linear elasticity}
    \item \statusdone{Gravitational body forces with incompressible elasticity}
    \item \statusbuggy{Prescribed slip on multiple faults}
    \end{itemize}
  \end{enumerate}

\end{frame}


% ------------------------------------------------------------ SLIDE
\begin{frame}
  \frametitle{Overview of Examples (cont.)}
  \summary{Examples progress from simple to more complex}

  \begin{enumerate}
  \setcounter{enumi}{5}
  \item 2d/subduction
    \begin{itemize}
    \item \statusdone{Meshing a 2-D cross-section of a subduction zone}
    \item \statusbuggy{Prescribed fault slip}
    \item \statuslater{Afterslip driven by traction changes from
        coseismic slip}
    \end{itemize}
  \item 3d/strikeslip (wish list)
    \begin{itemize}
    \item \statuslater{Meshing intersecting strike-slip faults with
        complex geometry}
    \item \statuslater{Prescribed fault slip}
    \end{itemize}
  \item 3d/subduction
    \begin{itemize}
    \item \statusdone{Meshing a 3-D subduction zone with complex
        geometry}
    \item \statusbuggy{Prescribed fault slip}
    \end{itemize}
  \end{enumerate}

\end{frame}




 
% ======================================================================
\end{document}


% End of file
