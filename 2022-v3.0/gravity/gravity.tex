% -*- TeX -*-
\documentclass[aspectratio=169]{beamer}

\title{PyLith v3.0 Tutorial}
\subtitle{Gravitational Body Forces}
\author{Charles Williams \\
  Brad Aagaard \\
  Matthew Knepley}
\institute{\includegraphics[scale=1.5]{../../logos/cig_logo_dots}%
  \hspace{4em}%
\raisebox{1em}{\includegraphics[scale=1.0]{../../logos/cig_short_pylith}}}
\date{June 21, 2022}


% ---------------------------------------------------- CUSTOMIZATION
\usetheme{CIG}
% Style information for PyLith presentations.

% Colors
\definecolor{ltorange}{rgb}{1.0, 0.74, 0.41} % 255/188/105
\definecolor{orange}{rgb}{0.96, 0.50, 0.0} % 246/127/0

\definecolor{ltred}{rgb}{1.0, 0.25, 0.25} % 255/64/64
\definecolor{red}{rgb}{0.79, 0.00, 0.01} % 201/0/3

\definecolor{ltpurple}{rgb}{0.81, 0.57, 1.00} % 206/145/255
\definecolor{purple}{rgb}{0.38, 0.00, 0.68} % 97/1/175

\definecolor{ltblue}{rgb}{0.2, 0.73, 1.0} % 51/187/255
\definecolor{mdblue}{rgb}{0.28, 0.50, 0.80} % 72/128/205
\definecolor{blue}{rgb}{0.12, 0.43, 0.59} % 30/110/150

\definecolor{ltltgreen}{rgb}{0.7, 1.00, 0.7} % 96/204/14
\definecolor{ltgreen}{rgb}{0.37, 0.80, 0.05} % 96/204/14
\definecolor{green}{rgb}{0.23, 0.49, 0.03} % 59/125/8
  
\definecolor{dkslate}{rgb}{0.18, 0.21, 0.28} % 47/53/72
\definecolor{mdslate}{rgb}{0.45, 0.50, 0.68} % 114/127/173
\definecolor{ltslate}{rgb}{0.85, 0.88, 0.95} % 216/225/229


\newcommand{\includefigure}[2][]{{\centering\includegraphics[#1]{#2}\par}}
\newcommand{\highlight}[1]{{\bf\usebeamercolor[fg]{structure}#1}}
\newcommand{\important}[1]{{\color{red}#1}}
\newcommand{\issue}[2]{\item[Issue:] {\color{red}#1}\\{\item[Soln:] \color{blue}#2}\\[4pt]}

\setbeamercolor{alerted text}{fg=ltgreen}
\setbeamertemplate{description item}[align left]


\newcommand{\lhs}[1]{{\color{blue}#1}}
\newcommand{\rhs}[1]{{\color{red}#1}}
\newcommand{\annotateL}[2]{%
  {\color{blue}\underbrace{\color{blue}#1}_{\color{blue}\mathclap{#2}}}}
\newcommand{\annotateR}[2]{%
  {\color{red}\underbrace{\color{red}#1}_{\color{red}\mathclap{#2}}}}
\newcommand{\eqnannotate}[2]{%
  {\color{blue}%
  \underbrace{\color{black}#1}_{\color{blue}\mathclap{#2}}}}

\newcommand{\trialvec}[1][]{{\vec{\psi}_\mathit{trial}^{#1}}}
\newcommand{\trialscalar}[1][]{{\psi_\mathit{trial}^{#1}}}
\newcommand{\basisvec}[1][]{{\vec{\psi}_\mathit{basis}^{#1}}}
\newcommand{\basisscalar}[1][]{{\psi_\mathit{basis}^{#1}}}

\newcommand{\tensor}[1]{\bm{#1}}
\DeclareMathOperator{\Tr}{Tr}

\usefonttheme[onlymath]{serif}

% minted shortcuts
\newminted{cfg}{bgcolor=ltslate,autogobble,fontsize=\tiny}
\newminted{bash}{bgcolor=ltltgreen,autogobble,fontsize=\tiny}

% PyLith components
\newcommand{\pylith}[1]{{\ttfamily\color{magenta}#1}}



% ========================================================= DOCUMENT
\begin{document}

% ------------------------------------------------------------ SLIDE
\maketitle

\logo{\includegraphics[height=4.5ex]{../../logos/cig_short_pylith}}

% ========================================================== SECTION
\section{{\ttfamily examples/reverse-2d}}
\subsection{Overview}

% ------------------------------------------------------------ SLIDE
\begin{frame}
  \frametitle{Vertical Cross-Section of Reverse Fault (2D): {\ttfamily examples/reverse-2d}}
  \summary{}

  \includefigure[height=6.1cm]{figs/geometry}

  \vfill
  Solve the static boundary elasticity equation with gravitational
  body forces.
  
\end{frame}


% ------------------------------------------------------------ SLIDE
\begin{frame}
  \frametitle{Steps in example}
  \summary{Steps 5-7 are covered in the elasticity with prescribed slip tutorial}

  \begin{description}
    \item[Step 1] \highlight{Gravitational body forces and linear isotropic elasticity}
    \item[Step 2] \highlight{Gravitational body forces and linear isotropic elasticity with a reference stress state}
    \item[Step 3] \highlight{Gravitational body forces and linear isotropic incompressible elasticity}
    \item[Step 4] \highlight{Surface tractions and linear isotropic linear elasticity}
    \item[Step 5] Earthquake rupture on one fault and linear isotropic linear elasticity
    \item[Step 6] Earthquake rupture on two faults and linear isotropic linear elasticity
    \item[Step 7] Same as Step 6 but time-dependent with linear isotropic Maxwell viscoelastic rheology
    \item[Step 8] Same as Step 7 but with linear isotropic power-law viscoelastic rheology
  \end{description}
  
\end{frame}

% ------------------------------------------------------------ SLIDE
\begin{frame}
  \frametitle{Concepts covered}
  \summary{}

  \begin{itemize}
  \item When are gravitational body forces necessary?
  \item Gravitational body forces in 2D
  \item Reference stresses to balance body forces
  \item Incompressible elasticity to achieve a reference state
  \item Traction boundary conditions to represent a surface load
  \end{itemize}
  
\end{frame}


% ------------------------------------------------------------ SLIDE
\begin{frame}
  \frametitle{When Do We Need to Use Gravitational Stresses?}
  \summary{}

  \begin{itemize}
  \item Pressure/stress-dependent rheology
    \begin{itemize}
    \item Pressure-dependent bulk rheology (for example, plasticity)
    \item Stress-dependent fault rheology (for example, friction)
    \end{itemize}
  \item Viscoelastic simulations where we care about vertical deformation
  \item Other simulations where we care about the absolute stress state
  \end{itemize}
  
\end{frame}


% ------------------------------------------------------------ SLIDE
\begin{frame}
  \frametitle{Finite-Element Mesh}
  \summary{See elasticity with prescribed slip tutorial}

  \includefigure[height=7.0cm]{figs/gmsh-tri}
  
\end{frame}


% ========================================================== SECTION
\subsection{Files used for simulations}

% ------------------------------------------------------------ SLIDE
\begin{frame}
  \frametitle{Files used in simulations}
  \summary{Files are in directory {\tt examples/reverse-2d}}

  \begin{description}
  \item[README.md] Brief description of the various examples
  \item[*.cfg] PyLith parameter files
  \item[generate\_gmsh.py] Python script to generate mesh using Gmsh
  \item[*.msh] Finite-element mesh files generated by Gmsh
  \item[*.spatialdb] Spatial database files
  \item[viz] Directory containing ParaView Python scripts
  \item[output] Directory containing simulation output; created automatically when running the simulations
  \end{description}

\end{frame}


% ========================================================== SECTION
\subsection{step01-gravity}

% ------------------------------------------------------------ SLIDE
\begin{frame}
  \frametitle{Step 1: Overview}
  \summary{Gravitational body forces applied to elastic material}

  \includefigure[height=6.5cm]{figs/step01-diagram}
      
\end{frame}


% ------------------------------------------------------------ SLIDE
\begin{frame}
  \frametitle{Step 1: Physics}
  \summary{}

  \begin{minipage}{0.3\textwidth}
    {\scriptsize
    \begin{gather*}
    % Solution
    \vec{s} = \left(\begin{array}{c} \vec{u} \end{array}\right)^T \\
    % Elasticity
    \rho(\vec{x}) \vec{g} + \tensor{\nabla} \cdot \tensor{\sigma}(\vec{u}) = \vec{0} \\
    % Dirichlet
    u_x = 0 \text{ on boundary\_xneg} \\
    u_x = 0 \text{ on boundary\_xpos} \\
    u_y = 0 \text{ on boundary\_yneg} \\
    \end{gather*}}
  \end{minipage}
  \hfill
  \begin{minipage}{0.67\textwidth}
    \includefigure[width=\textwidth]{figs/step01-diagram}
  \end{minipage}
      
\end{frame}


% ------------------------------------------------------------ SLIDE
\begin{frame}[t,fragile]
  \frametitle{Step 1: Physics to simulation parameters}
  \summary{}

  \begin{minipage}[t]{0.3\textwidth}
    {\scriptsize
    \begin{gather*}
    % Solution
    \vec{s} = \left(\begin{array}{c} \vec{u} \end{array}\right)^T \tikzmark{solution1}\\
    % Elasticity
    \rho(\vec{x}) \vec{g} + \tensor{\nabla} \cdot \tensor{\sigma}(\vec{u}) = \vec{0} \tikzmark{material1}\\
    % Dirichlet
    u_x = 0 \text{ on boundary\_xneg} \tikzmark{bc1}\\
    u_x = 0 \text{ on boundary\_xpos} \\
    u_y = 0 \text{ on boundary\_yneg} \\
    \end{gather*}}
  \end{minipage}
  \hfill
  \begin{minipage}[t]{0.67\textwidth}
    % Solution
    \begin{onlyenv}<2>
      \tikzmark{solution1-cfg}
      \begin{cfgcode}
        [pylithapp.problem]
        solution = pylith.problems.SolnDisp
        defaults.quadrature_order = 2
        
        [pylithapp.problem.solution.subfields]
        displacement.basis_order = 2
      \end{cfgcode}
    \end{onlyenv}
    %
    % Governing equations
    \begin{onlyenv}<3>
      \tikzmark{material1-cfg}
      \begin{cfgcode}
        [pylithapp.problem]
        gravity_field = spatialdata.spatialdb.GravityField
        gravity_field.gravity_dir = [0.0, -1.0, 0.0]
      \end{cfgcode}
    \end{onlyenv}
    %
    % Boundary conditions
    \begin{onlyenv}<4>
      \tikzmark{bc1-cfg}
      \begin{cfgcode}
        bc = [bc_xneg, bc_xpos, bc_yneg]
        bc.bc_xneg = pylith.bc.DirichletTimeDependent
        bc.bc_xpos = pylith.bc.DirichletTimeDependent
        bc.bc_yneg = pylith.bc.DirichletTimeDependent
        
        [pylithapp.problem.bc.bc_xpos]
        label = boundary_xpos
        label_value = 11
        constrained_dof = [0]
        db_auxiliary_field = pylith.bc.ZeroDB
        db_auxiliary_field.description = Dirichlet BC +x edge

        auxiliary_subfields.initial_amplitude.basis_order = 0 
      \end{cfgcode}
    \end{onlyenv}
  \end{minipage}

  \begin{tikzpicture}[overlay,remember picture]
    \draw[physics-arrow,visible on=<2>] ($(pic cs:solution1-cfg)-(0,2em)$) to (pic cs:solution1);
    \draw[physics-arrow,visible on=<3>] ($(pic cs:material1-cfg)-(0,2em)$) to (pic cs:material1);
    \draw[physics-arrow,visible on=<4>] ($(pic cs:bc1-cfg)-(0,2em)$) to (pic cs:bc1);
  \end{tikzpicture}
  
\end{frame}


% ------------------------------------------------------------ SLIDE
\begin{frame}
  \frametitle{Step 1: Input files}
  \summary{}

  \begin{description}
  \item[mesh\_tri.msh] Finite-element mesh generated using Gmsh
  \item[pylithapp.cfg] PyLith parameter file common to all steps
  \item[step01\_gravity.cfg] PyLith parameter file
  \item[mat\_elastic.spatialdb] Spatial database for isotropic linear elastic properties
  \end{description}
    
\end{frame}


% ------------------------------------------------------------ SLIDE
\begin{frame}[fragile]
  \frametitle{Step 1: Run the simulation}
  \summary{}

\begin{bashcode}
pylith step01_gravity.cfg

# Output
 >> /software/unix/py39-venv/pylith-debug/lib/python3.9/site-packages/pylith/meshio/MeshIOObj.py:44:read
 -- meshiopetsc(info)
 -- Reading finite-element mesh
 >> /src/cig/pylith/libsrc/pylith/meshio/MeshIO.cc:94:void pylith::meshio::MeshIO::read(pylith::topology::Mesh*)
 -- meshiopetsc(info)
 -- Component 'reader': Domain bounding box:
    (-100000, 100000)
    (-100000, 0)

# -- many lines omitted --

 >> /software/unix/py39-venv/pylith-debug/lib/python3.9/site-packages/pylith/problems/TimeDependent.py:139:run
 -- timedependent(info)
 -- Solving problem.
0 TS dt 0.01 time 0.
    0 SNES Function norm 2.873918352757e-01
    Linear solve converged due to CONVERGED_RTOL iterations 1
    1 SNES Function norm 3.025686251687e-13
  Nonlinear solve converged due to CONVERGED_FNORM_ABS iterations 1
1 TS dt 0.01 time 0.01
 >> /software/unix/py39-venv/pylith-debug/lib/python3.9/site-packages/pylith/problems/Problem.py:201:finalize
 -- timedependent(info)
 -- Finalizing problem.
\end{bashcode}
  
\end{frame}


% ------------------------------------------------------------ SLIDE
\begin{frame}
  \frametitle{Step 1: Visualize results}
  \summary{Run the {\tt viz/plot\_dispwarp.py} Python script from within ParaView}

  \includefigure[height=7.0cm]{figs/step01-solution}
    
\end{frame}


% ========================================================== SECTION
\subsection{step02-gravity-refstate}


% ------------------------------------------------------------ SLIDE
\begin{frame}
  \frametitle{Step 2: Physics}
  \summary{}

  \begin{minipage}{0.3\textwidth}
    {\scriptsize
    \begin{gather*}
    % Solution
    \vec{s} = \left(\begin{array}{c} \vec{u} \end{array}\right)^T \\
    % Elasticity
    \rho(\vec{x}) \vec{g} + \tensor{\nabla} \cdot \tensor{\sigma}(\vec{u}) = \vec{0} \\
    % Dirichlet
    u_x = 0 \text{ on boundary\_xneg} \\
    u_x = 0 \text{ on boundary\_xpos} \\
    u_y = 0 \text{ on boundary\_yneg} \\
    \end{gather*}}
  \end{minipage}
  \hfill
  \begin{minipage}{0.67\textwidth}
    \includefigure[width=\textwidth]{figs/step01-diagram}
  \end{minipage}
      
\end{frame}


% ------------------------------------------------------------ SLIDE
\begin{frame}[t,fragile]
  \frametitle{Step 2: Physics to simulation parameters}
  \summary{}

  \begin{minipage}[t]{0.3\textwidth}
    {\scriptsize
    \begin{gather*}
    % Solution
    \vec{s} = \left(\begin{array}{c} \vec{u} \end{array}\right)^T \tikzmark{solution2}\\
    % Elasticity
    \rho(\vec{x}) \vec{g} + \tensor{\nabla} \cdot \tensor{\sigma}(\vec{u}) = \vec{0} \tikzmark{material2}\\
    \tensor{\sigma} - \tensor{\sigma}^\mathit{ref} = \tensor{C} : \left(\tensor{\epsilon}-\tensor{\epsilon}^\mathit{ref}\right) \\
    % Dirichlet
    u_x = 0 \text{ on boundary\_xneg} \tikzmark{bc2}\\
    u_x = 0 \text{ on boundary\_xpos} \\
    u_y = 0 \text{ on boundary\_yneg} \\
    \end{gather*}}
  \end{minipage}
  \hfill
  \begin{minipage}[t]{0.67\textwidth}
    % Solution
    \begin{onlyenv}<2>
      \tikzmark{solution2-cfg}
      \begin{cfgcode}
        # Same as in Step 1
      \end{cfgcode}
    \end{onlyenv}
    %
    % Governing equations
    \begin{onlyenv}<3>
      \tikzmark{material2-cfg}
      \begin{cfgcode}
        [pylithapp.problem.materials.slab]
        db_auxiliary_field.iohandler.filename = mat_gravity_refstate.spatialdb
        db_auxiliary_field.query_type = linear

        bulk_rheology.use_reference_state = True
      \end{cfgcode}
    \end{onlyenv}
    %
    % Boundary conditions
    \begin{onlyenv}<4>
      \tikzmark{bc2-cfg}
      \begin{cfgcode}
        # Same as Step 1
      \end{cfgcode}
    \end{onlyenv}
  \end{minipage}

  \begin{tikzpicture}[overlay,remember picture]
    \draw[physics-arrow,visible on=<2>] ($(pic cs:solution2-cfg)-(0,2em)$) to (pic cs:solution2);
    \draw[physics-arrow,visible on=<3>] ($(pic cs:material2-cfg)-(0,2em)$) to (pic cs:material2);
    \draw[physics-arrow,visible on=<4>] ($(pic cs:bc2-cfg)-(0,2em)$) to (pic cs:bc2);
  \end{tikzpicture}
  
\end{frame}


% ------------------------------------------------------------ SLIDE
\begin{frame}
  \frametitle{Step 2: Input files}
  \summary{}

  \begin{description}
  \item[mesh\_tri.msh] Finite-element mesh generated using Gmsh
  \item[pylithapp.cfg] PyLith parameter file common to all steps
  \item[step02\_gravity\_refstate.cfg] PyLith parameter file
  \item[mat\_gravity\_refstate.spatialdb] Spatial database with elastic properties and reference stress and strain
  \end{description}

  \vfill
  Reference stress
  \begin{equation}
    \sigma_{xx} = \sigma_{yy} = \sigma_{zz} = \int_0^y \rho g \, dy = \rho g y,
  \end{equation}
  
\end{frame}


% ------------------------------------------------------------ SLIDE
\begin{frame}[fragile]
  \frametitle{Step 2: Run the simulation}
  \summary{}

\begin{bashcode}
pylith step02_gravity_refstate.cfg

# Output
 >> /software/unix/py39-venv/pylith-debug/lib/python3.9/site-packages/pylith/meshio/MeshIOObj.py:44:read
 -- meshiopetsc(info)
 -- Reading finite-element mesh
 >> /src/cig/pylith/libsrc/pylith/meshio/MeshIO.cc:94:void pylith::meshio::MeshIO::read(pylith::topology::Mesh*)
 -- meshiopetsc(info)
 -- Component 'reader': Domain bounding box:
    (-100000, 100000)
    (-100000, 0)

# -- many lines omitted --

 >> /software/unix/py39-venv/pylith-debug/lib/python3.9/site-packages/pylith/problems/TimeDependent.py:139:run
 -- timedependent(info)
 -- Solving problem.
0 TS dt 0.01 time 0.
    0 SNES Function norm 4.578015693966e-15
  Nonlinear solve converged due to CONVERGED_FNORM_ABS iterations 0
1 TS dt 0.01 time 0.01
 >> /software/unix/py39-venv/pylith-debug/lib/python3.9/site-packages/pylith/problems/Problem.py:201:finalize
 -- timedependent(info)
 -- Finalizing problem.
WARNING! There are options you set that were not used!
WARNING! could be spelling mistake, etc!
There is one unused database option. It is:
Option left: name:-ksp_converged_reason (no value)
\end{bashcode}
  
\end{frame}


% ------------------------------------------------------------ SLIDE
\begin{frame}
  \frametitle{Step 2: Visualize results}
  \summary{Run the {\tt viz/plot\_dispwarp.py} Python script from within ParaView}

  \includefigure[height=7.0cm]{figs/step02-solution}
    
\end{frame}


% ========================================================== SECTION
\subsection{step03-gravity-incompressible}

% ------------------------------------------------------------ SLIDE
\begin{frame}
  \frametitle{Step 3: Physics}
  \summary{}

  \begin{minipage}{0.3\textwidth}
    {\scriptsize
    \begin{gather*}
      % Solution
      \vec{s} = \left( \vec{u} \quad \ p \right)^T, \\
      % Incompressible elasticity
      \rho(\vec{x})\vec{g} + \tensor{\nabla} \cdot \left(\tensor{\sigma}^\mathit{dev}(\vec{u}) - p\tensor{I}\right) = \vec{0} \\
      % Pressure
      \vec{\nabla} \cdot \vec{u} + \frac{p}{K} = 0 \\
      % Dirichlet
      u_x = 0 \text{ on boundary\_xneg} \\
      u_x = 0 \text{ on boundary\_xpos} \\
      u_y = 0 \text{ on boundary\_yneg} \\
      p = 0 \text{ on boundary\_ypos}
    \end{gather*}}
  \end{minipage}
  \hfill
  \begin{minipage}{0.67\textwidth}
    \includefigure[width=\textwidth]{figs/step01-diagram}
  \end{minipage}
      
\end{frame}


% ------------------------------------------------------------ SLIDE
\begin{frame}[t,fragile]
  \frametitle{Step 3: Physics to simulation parameters}
  \summary{}

  \begin{minipage}[t]{0.3\textwidth}
    {\scriptsize
    \begin{gather*}
      % Solution
      \vec{s} = \left( \vec{u} \quad \ p \right)^T \tikzmark{solution3}\\
      % Incompressible elasticity
      \rho(\vec{x})\vec{g} + \tensor{\nabla} \cdot \left(\tensor{\sigma}^\mathit{dev}(\vec{u}) - p\tensor{I}\right) = \vec{0} \tikzmark{material3}\\
      % Pressure
      \vec{\nabla} \cdot \vec{u} + \frac{p}{K} = 0 \\
      % Dirichlet
      u_x = 0 \text{ on boundary\_xneg} \tikzmark{bc3}\\
      u_x = 0 \text{ on boundary\_xpos} \\
      u_y = 0 \text{ on boundary\_yneg} \\
      p = 0 \text{ on boundary\_ypos}
    \end{gather*}}
  \end{minipage}
  \hfill
  \begin{minipage}[t]{0.67\textwidth}
    % Solution
    \begin{onlyenv}<2>
      \tikzmark{solution3-cfg}
      \begin{cfgcode}
        [pylithapp.problem]
        solution = pylith.problems.SolnDispPres

        [pylithapp.problem.solution.subfields]
        displacement.basis_order = 1
        pressure.basis_order = 1
      \end{cfgcode}
    \end{onlyenv}
    %
    % Governing equations
    \begin{onlyenv}<3>
      \tikzmark{material3-cfg}
      \begin{cfgcode}
        [pylithapp.problem.materials]
        slab = pylith.materials.IncompressibleElasticity
        crust = pylith.materials.IncompressibleElasticity
        wedge = pylith.materials.IncompressibleElasticity

        [pylithapp.problem.materials.slab]
        db_auxiliary_field.iohandler.filename = mat_elastic_incompressible.spatialdb
      \end{cfgcode}
    \end{onlyenv}
    %
    % Boundary conditions
    \begin{onlyenv}<4>
      \tikzmark{bc3-cfg}
      \begin{cfgcode}
        [pylithapp.problem]
        bc = [bc_xneg, bc_xpos, bc_yneg, bc_ypos]
        bc.bc_ypos = pylith.bc.DirichletTimeDependent

        [pylithapp.problem.bc.bc_ypos]
        label = boundary_ypos
        label_value = 13
        constrained_dof = [0]
        field = pressure
        db_auxiliary_field = pylith.bc.ZeroDB
        db_auxiliary_field.description = Dirichlet BC for pressure on +y edge

        auxiliary_subfields.initial_amplitude.basis_order = 0
      \end{cfgcode}
    \end{onlyenv}
  \end{minipage}

  \begin{tikzpicture}[overlay,remember picture]
    \draw[physics-arrow,visible on=<2>] ($(pic cs:solution3-cfg)-(0,2em)$) to (pic cs:solution3);
    \draw[physics-arrow,visible on=<3>] ($(pic cs:material3-cfg)-(0,2em)$) to (pic cs:material3);
    \draw[physics-arrow,visible on=<4>] ($(pic cs:bc3-cfg)-(0,2em)$) to (pic cs:bc3);
  \end{tikzpicture}
  
\end{frame}


% ------------------------------------------------------------ SLIDE
\begin{frame}
  \frametitle{Step 3: Input files}
  \summary{}

  \begin{description}
  \item[mesh\_tri.msh] Finite-element mesh generated using Gmsh
  \item[pylithapp.cfg] PyLith parameter file common to all steps
  \item[step03\_gravity\_incompressible.cfg] PyLith parameter file
  \item[mat\_elastic\_incompressible.spatialdb] Spatial database with incompressible elastic properties
  \end{description}

\end{frame}


% ------------------------------------------------------------ SLIDE
\begin{frame}[fragile]
  \frametitle{Step 3: Run the simulation}
  \summary{}

\begin{bashcode}
pylith step03_gravity_incompressible.cfg

# Output
 >> /software/unix/py39-venv/pylith-debug/lib/python3.9/site-packages/pylith/meshio/MeshIOObj.py:44:read
 -- meshiopetsc(info)
 -- Reading finite-element mesh
 >> /src/cig/pylith/libsrc/pylith/meshio/MeshIO.cc:94:void pylith::meshio::MeshIO::read(pylith::topology::Mesh*)
 -- meshiopetsc(info)
 -- Component 'reader': Domain bounding box:
    (-100000, 100000)
    (-100000, 0)

# -- many lines omitted --

 >> /software/unix/py39-venv/pylith-debug/lib/python3.9/site-packages/pylith/meshio/MeshIOObj.py:44:read
 -- timedependent(info)
 -- Solving problem.
0 TS dt 0.01 time 0.
    0 SNES Function norm 4.866941773461e-01
    Linear solve converged due to CONVERGED_ATOL iterations 1
    1 SNES Function norm 3.099989574301e-13
  Nonlinear solve converged due to CONVERGED_FNORM_ABS iterations 1
1 TS dt 0.01 time 0.01
 >> /software/unix/py39-venv/pylith-debug/lib/python3.9/site-packages/pylith/problems/Problem.py:201:finalize
 -- timedependent(info)
 -- Finalizing problem.
\end{bashcode}
  
\end{frame}


% ------------------------------------------------------------ SLIDE
\begin{frame}
  \frametitle{Step 3: Visualize results}
  \summary{Run the {\tt viz/plot\_dispwarp.py} Python script from within ParaView}

  \includefigure[height=7.0cm]{figs/step03-solution}
    
\end{frame}


% ========================================================== SECTION
\subsection{step4-surfload}

% ------------------------------------------------------------ SLIDE
\begin{frame}
  \frametitle{Step 4: Overview}
  \summary{Normal tractions applied to simulate a surface load}

  \includefigure[height=6.5cm]{figs/step04-diagram}
      
\end{frame}


% ------------------------------------------------------------ SLIDE
\begin{frame}
  \frametitle{Step 4: Physics}
  \summary{}

  \begin{minipage}{0.3\textwidth}
    {\scriptsize
      \begin{gather*}
        % Solution
        \vec{s} = \left(\begin{array}{c} \vec{u} \end{array}\right)^T \\
        % Elasticity
        \tensor{\nabla} \cdot \tensor{\sigma}(\vec{u}) = \vec{0} \\
        % Dirichlet
        u_x = -u_0 \text{ on boundary\_xneg} \\
        u_x = +u_0 \text{ on boundary\_xpos} \\
        u_y = 0 \text{ on boundary\_yneg} \\
        % Neumann
        \tau_n = \tau_0(\vec{x} \text{ on boundary\_ypos}
    \end{gather*}}
  \end{minipage}
  \hfill
  \begin{minipage}{0.67\textwidth}
    \includefigure[width=\textwidth]{figs/step04-diagram}
  \end{minipage}
      
\end{frame}


% ------------------------------------------------------------ SLIDE
\begin{frame}[t,fragile]
  \frametitle{Step 4: Physics to simulation parameters}
  \summary{}

  \begin{minipage}[t]{0.3\textwidth}
    {\scriptsize
    \begin{gather*}
        % Solution
        \vec{s} = \left(\begin{array}{c} \vec{u} \end{array}\right)^T \tikzmark{solution4}\\
        % Elasticity
        \tensor{\nabla} \cdot \tensor{\sigma}(\vec{u}) = \vec{0} \tikzmark{material4}\\
        % Dirichlet
        u_x = -u_0 \text{ on boundary\_xneg} \tikzmark{bc4}\\
        u_x = +u_0 \text{ on boundary\_xpos} \\
        u_y = 0 \text{ on boundary\_yneg} \\
        % Neumann
        \tau_n = \tau_0(\vec{x}) \text{ on boundary\_ypos}
    \end{gather*}}
  \end{minipage}
  \hfill
  \begin{minipage}[t]{0.67\textwidth}
    % Solution
    \begin{onlyenv}<2>
      \tikzmark{solution4-cfg}
      \begin{cfgcode}
        # Same as Steps 1 and 2
      \end{cfgcode}
    \end{onlyenv}
    %
    % Governing equations
    \begin{onlyenv}<3>
      \tikzmark{material4-cfg}
      \begin{cfgcode}
        # Same as Steps 1 and 2
      \end{cfgcode}
    \end{onlyenv}
    %
    % Boundary conditions
    \begin{onlyenv}<4>
      \tikzmark{bc4-cfg}
      \begin{cfgcode}
        bc = [bc_xneg, bc_xpos, bc_yneg, bc_ypos]
        bc.bc_ypos = pylith.bc.NeumannTimeDependent

        [pylithapp.problem.bc.bc_ypos]
        label = boundary_ypos
        label_value = 13

        db_auxiliary_field = spatialdata.spatialdb.SimpleDB
        db_auxiliary_field.description = Neumann BC +y edge
        db_auxiliary_field.iohandler.filename = traction_surfload.spatialdb
        db_auxiliary_field.query_type = linear

        auxiliary_subfields.initial_amplitude.basis_order = 1
      \end{cfgcode}
    \end{onlyenv}
  \end{minipage}

  \begin{tikzpicture}[overlay,remember picture]
    \draw[physics-arrow,visible on=<2>] ($(pic cs:solution4-cfg)-(0,2em)$) to (pic cs:solution4);
    \draw[physics-arrow,visible on=<3>] ($(pic cs:material4-cfg)-(0,2em)$) to (pic cs:material4);
    \draw[physics-arrow,visible on=<4>] ($(pic cs:bc4-cfg)-(0,2em)$) to (pic cs:bc4);
  \end{tikzpicture}
  
\end{frame}


% ------------------------------------------------------------ SLIDE
\begin{frame}
  \frametitle{Step 4: Input files}
  \summary{}

  \begin{description}
  \item[mesh\_tri.msh] Finite-element mesh generated using Gmsh
  \item[pylithapp.cfg] PyLith parameter file common to all steps
  \item[step04\_surfload.cfg] PyLith parameter file
  \item[mat\_elastic.spatialdb] Spatial database with elastic properties
  \item[traction\_surfload.spatialdb] Spatial database with traction distribution for surface load
  \end{description}

\end{frame}


% ------------------------------------------------------------ SLIDE
\begin{frame}[fragile]
  \frametitle{Step 4: Run the simulation}
  \summary{}

\begin{bashcode}
pylith step04_surfload.cfg

# Output
 >> /software/unix/py39-venv/pylith-debug/lib/python3.9/site-packages/pylith/meshio/MeshIOObj.py:44:read
 -- meshiopetsc(info)
 -- Reading finite-element mesh
 >> /src/cig/pylith/libsrc/pylith/meshio/MeshIO.cc:94:void pylith::meshio::MeshIO::read(pylith::topology::Mesh*)
 -- meshiopetsc(info)
 -- Component 'reader': Domain bounding box:
    (-100000, 100000)
    (-100000, 0)

# -- many lines omitted --

 >> /software/unix/py39-venv/pylith-debug/lib/python3.9/site-packages/pylith/problems/TimeDependent.py:139:run
 -- timedependent(info)
 -- Solving problem.
0 TS dt 0.01 time 0.
    0 SNES Function norm 1.213351093160e-02
    Linear solve converged due to CONVERGED_ATOL iterations 1
    1 SNES Function norm 1.038106792811e-15
  Nonlinear solve converged due to CONVERGED_FNORM_ABS iterations 1
1 TS dt 0.01 time 0.01
 >> /software/unix/py39-venv/pylith-debug/lib/python3.9/site-packages/pylith/problems/Problem.py:201:finalize
 -- timedependent(info)
 -- Finalizing problem.
\end{bashcode}
  
\end{frame}


% ------------------------------------------------------------ SLIDE
\begin{frame}
  \frametitle{Step 4: Visualize results}
  \summary{Run the {\tt viz/plot\_dispwarp.py} Python script from within ParaView}

  \includefigure[height=7.0cm]{figs/step04-solution}
    
\end{frame}


% ======================================================================
\end{document}


% End of file
