% -*- TeX -*-
\documentclass[aspectratio=169]{beamer}

\usepackage{tikz}
\usetikzlibrary{shapes,arrows,calc}
\usetikzlibrary{decorations.pathreplacing}
\usetikzlibrary{fit,matrix}

\usepackage{listings}
\lstset{basicstyle=\tiny}

\title{PyLith Modeling Tutorial}
\subtitle{Troubleshooting PyLith Simulations}
\author{Brad Aagaard}
\institute{\includegraphics[scale=0.4]{../../logos/cig_blackfg}}
\date{June 10, 2019}


% ---------------------------------------------------- CUSTOMIZATION
%\renewcommand{\thispdfpagelabel}[1]{}
\usetheme{CIG}

% Local macros
% Style information for PyLith presentations.

% Colors
\definecolor{ltorange}{rgb}{1.0, 0.74, 0.41} % 255/188/105
\definecolor{orange}{rgb}{0.96, 0.50, 0.0} % 246/127/0

\definecolor{ltred}{rgb}{1.0, 0.25, 0.25} % 255/64/64
\definecolor{red}{rgb}{0.79, 0.00, 0.01} % 201/0/3

\definecolor{ltpurple}{rgb}{0.81, 0.57, 1.00} % 206/145/255
\definecolor{purple}{rgb}{0.38, 0.00, 0.68} % 97/1/175

\definecolor{ltblue}{rgb}{0.2, 0.73, 1.0} % 51/187/255
\definecolor{mdblue}{rgb}{0.28, 0.50, 0.80} % 72/128/205
\definecolor{blue}{rgb}{0.12, 0.43, 0.59} % 30/110/150

\definecolor{ltltgreen}{rgb}{0.7, 1.00, 0.7} % 96/204/14
\definecolor{ltgreen}{rgb}{0.37, 0.80, 0.05} % 96/204/14
\definecolor{green}{rgb}{0.23, 0.49, 0.03} % 59/125/8
  
\definecolor{dkslate}{rgb}{0.18, 0.21, 0.28} % 47/53/72
\definecolor{mdslate}{rgb}{0.45, 0.50, 0.68} % 114/127/173
\definecolor{ltslate}{rgb}{0.85, 0.88, 0.95} % 216/225/229


\newcommand{\includefigure}[2][]{{\centering\includegraphics[#1]{#2}\par}}
\newcommand{\highlight}[1]{{\bf\usebeamercolor[fg]{structure}#1}}
\newcommand{\important}[1]{{\color{red}#1}}
\newcommand{\issue}[2]{\item[Issue:] {\color{red}#1}\\{\item[Soln:] \color{blue}#2}\\[4pt]}

\setbeamercolor{alerted text}{fg=ltgreen}
\setbeamertemplate{description item}[align left]


\newcommand{\lhs}[1]{{\color{blue}#1}}
\newcommand{\rhs}[1]{{\color{red}#1}}
\newcommand{\annotateL}[2]{%
  {\color{blue}\underbrace{\color{blue}#1}_{\color{blue}\mathclap{#2}}}}
\newcommand{\annotateR}[2]{%
  {\color{red}\underbrace{\color{red}#1}_{\color{red}\mathclap{#2}}}}
\newcommand{\eqnannotate}[2]{%
  {\color{blue}%
  \underbrace{\color{black}#1}_{\color{blue}\mathclap{#2}}}}

\newcommand{\trialvec}[1][]{{\vec{\psi}_\mathit{trial}^{#1}}}
\newcommand{\trialscalar}[1][]{{\psi_\mathit{trial}^{#1}}}
\newcommand{\basisvec}[1][]{{\vec{\psi}_\mathit{basis}^{#1}}}
\newcommand{\basisscalar}[1][]{{\psi_\mathit{basis}^{#1}}}

\newcommand{\tensor}[1]{\bm{#1}}
\DeclareMathOperator{\Tr}{Tr}

\usefonttheme[onlymath]{serif}

% minted shortcuts
\newminted{cfg}{bgcolor=ltslate,autogobble,fontsize=\tiny}
\newminted{bash}{bgcolor=ltltgreen,autogobble,fontsize=\tiny}

% PyLith components
\newcommand{\pylith}[1]{{\ttfamily\color{magenta}#1}}




% ========================================================= DOCUMENT
\begin{document}

% ------------------------------------------------------------ SLIDE
\maketitle

% ------------------------------------------------------------ SLIDE
\logo{\includegraphics[height=4.5ex]{../../logos/cig_blackfg}}

% ========================================================== SECTION
\section{Troubleshooting}
\subsection{Parameters}

% ------------------------------------------------------------ SLIDE
\begin{frame}
  \frametitle{What parameters are available?}
  \summary{Parameters are specified as a hierarchy of components and properties}

  \begin{itemize}
  \item Components (Facilities) are the object building blocks\\
    \important{Appendix B of the PyLith manual lists all of the components}
    \begin{itemize}
    \item Problem \component{TimeDependent}
    \item Boundary conditions \component{DirichletTimeDependent}
    \item Faults \component{FaultCohesiveKin}
    \item Materials \component{Elasticity}
    \item Solution observers \component{OutputSolnBoundary}
    \item Readers \component{MeshIOCubit}
    \end{itemize}
  \item Properties are the basic types
    \begin{itemize}
    \item String \property{mat\_viscoelastic.spatialdb}
    \item Integer \property{4}
    \item Float \property{2.3}
    \item Dimensioned quantity \property{2.5*year}
    \item Array of Strings, Integers, or Floats \property{[0, 0, 1]}
    \end{itemize}
  \end{itemize}
  
\end{frame}


% ------------------------------------------------------------ SLIDE
\begin{frame}[fragile]
  \frametitle{Parameter Files}
  \summary{Simple syntax for specifying parameters for properties and components}

\begin{cfg}
# Syntax
<h>[pylithapp.COMPONENT.SUBCOMPONENT]</h> ; Inline comment
<f>COMPONENT</f> = OBJECT
<p>PARAMETER</p> = VALUE

# Example
<h>[pylithapp.mesh_generator]</h> ; Header indicates path of mesh_generator in hierarchy
<f>reader</f> = pylith.meshio.MeshIOCubit ; Use mesh from CUBIT/Trelis
<p>reader.filename</p> = mesh_quad4.exo ; Set filename of mesh.
<p>reader.coordsys.space_dim</p> = 2 ; Set coordinate system of mesh.

<h>[pylithapp.problem.solution_outputs.output]</h> ; Set output format
<f>writer</f> = pylith.meshio.DataWriterHDF5
<p>writer.filename</p> = axialdisp.h5

<h>[pylithapp.problem]</h>
<f>bc</f> = [x_neg, x_pos, y_neg] ; Create array of boundary conditions
<f>bc.x_neg</f> = pylith.bc.DirichletTimeDependent ; Set type of boundary condition
<f>bc.x_pos</f> = pylith.bc.DirichletTimeDependent
<f>bc.y_neg</f> = pylith.bc.DirichletTimeDependent

<h>[pylithapp.problem.bc.x_pos]</h> ; Boundary condition for +x
<p>constrained_dof</p> = [0] ; Constrain x DOF
<p>label</p> = edge_xpos ; Name of nodeset from CUBIT/Trelis
<f>db_auxiliary_field</f> = spatialdata.spatialdb.SimpleDB ; Set type of spatial database
<p>db_auxiliary_field.label</p> = Dirichlet BC +x edge
<p>db_auxiliary_field.iohandler.filename</p> = axial_disp.spatialdb ; Filename for database
\end{cfg}

\end{frame}

% ------------------------------------------------------------ SLIDE
\begin{frame}
  \frametitle{Parameters Graphical User-Interface}
  \summary{{\tt cd parametersgui; ./pylith\_paramviewer}}

  \includefigure[height=8.5cm]{figs/paramgui_snapshot}

\end{frame}

% ------------------------------------------------------------ SLIDE
\begin{frame}[fragile]
  \frametitle{Parameters Graphical User-Interface}
  \summary{Case study: {\tt examples/2d/box/step02\_sheardisp}}

\begin{enumerate}
\item Generate the JSON file with the parameters
\begin{lstlisting}
cd examples/2d/box
pylithinfo step02_sheardisp.cfg
\end{lstlisting}
\item Start the web-server (start at your top-level PyLith directory)
\begin{lstlisting}
cd parametersgui
./pylith_paramviewer
\end{lstlisting}
\item Point your web browser to {\tt http://127.0.0.1:9000}
\item Load the parameter file
\end{enumerate}

\end{frame}


% ------------------------------------------------------------ SLIDE
\begin{frame}
  \frametitle{Show values of parameters using the command line}
  \summary{Case study: {\tt examples/2d/box/step02\_sheardisp}}

  \begin{itemize}
  \item Components and properties for given component \cmdblurb{--help}
    \begin{description}
    \item[step02\_sheardisp.cfg] \cfgblurb{[pylithapp.problem.bc.y\_neg]}
    \item[shell] \cmdblurb{pylith step02.cfg --problem.bc.y\_neg.help}
    \end{description}
  \item Current components of a given component \cmdblurb{--help-components}
    \begin{description}
    \item[step02\_sheardisp.cfg] \cfgblurb{[pylithapp.problem.bc.y\_neg]}
    \item[shell] \cmdblurb{pylith step02\_sheardisp.cfg --problem.bc.y\_neg.help-components}
    \end{description}
  \item Current properties of a given component \cmdblurb{--help-properties}
    \begin{description}
    \item[step02\_sheardisp.cfg] \cfgblurb{[pylithapp.problem.bc.y\_neg]}
    \item[shell] \cmdblurb{pylith step02\_sheardisp.cfg --problem.bc.y\_neg.help-properties}
    \end{description}
  \end{itemize}

\end{frame}


% ------------------------------------------------------------ SLIDE
\begin{frame}
  \frametitle{What about a GUI for editing parameters?}
  \summary{On the wish list but will require time or a developer}

  \begin{itemize}
  \item Parameter viewer $\rightarrow$ editor
    \begin{itemize}
    \item See possible choices for components and properties
    \item Basic validation of parameters
    \item $\Rightarrow$ Generate JSON schema from component
      specifications
    \item $\Rightarrow$ Translate JSON schema into GUI
    \end{itemize}
  \item Export parameters to single file\\
    Facilitates archiving parameters used in given simulation
  \end{itemize}

\end{frame}


% ========================================================== SECTION
\section{Error Messages}
\subsection{}

% ------------------------------------------------------------ SLIDE
\begin{frame}
  \frametitle{Troubleshooting Examples}
  \summary{See {\tt examples/troubleshooting/nofaults-2d}}

  \highlight{Introduce common (and a few uncommon) errors into 2d/box
    input files}
  
\end{frame}


% ======================================================================
\end{document}


% End of file
