% -*- TeX -*-
\documentclass[aspectratio=169,hyperref=colorlinks]{beamer}

\usepackage{mathtools}
\usepackage{bm}
\usepackage{listings}

\usepackage{tikz}
\usetikzlibrary{shapes,arrows,calc}
\usetikzlibrary{decorations.pathreplacing}
\usetikzlibrary{fit,matrix}

\title{Crustal Deformation Modeling Tutorial}
\subtitle{PyLith Version 3}
\author{Brad Aagaard \\
  Matthew Knepley \\
  Charles Williams}
\institute{\includegraphics[scale=0.4]{../../logos/cig_blackfg}}
\date{June 27, 2017}

% ---------------------------------------------------- CUSTOMIZATION
%\renewcommand{\thispdfpagelabel}[1]{}
\usetheme{CIG}

% Style information for PyLith presentations.

% Colors
\definecolor{ltorange}{rgb}{1.0, 0.74, 0.41} % 255/188/105
\definecolor{orange}{rgb}{0.96, 0.50, 0.0} % 246/127/0

\definecolor{ltred}{rgb}{1.0, 0.25, 0.25} % 255/64/64
\definecolor{red}{rgb}{0.79, 0.00, 0.01} % 201/0/3

\definecolor{ltpurple}{rgb}{0.81, 0.57, 1.00} % 206/145/255
\definecolor{purple}{rgb}{0.38, 0.00, 0.68} % 97/1/175

\definecolor{ltblue}{rgb}{0.2, 0.73, 1.0} % 51/187/255
\definecolor{mdblue}{rgb}{0.28, 0.50, 0.80} % 72/128/205
\definecolor{blue}{rgb}{0.12, 0.43, 0.59} % 30/110/150

\definecolor{ltltgreen}{rgb}{0.7, 1.00, 0.7} % 96/204/14
\definecolor{ltgreen}{rgb}{0.37, 0.80, 0.05} % 96/204/14
\definecolor{green}{rgb}{0.23, 0.49, 0.03} % 59/125/8
  
\definecolor{dkslate}{rgb}{0.18, 0.21, 0.28} % 47/53/72
\definecolor{mdslate}{rgb}{0.45, 0.50, 0.68} % 114/127/173
\definecolor{ltslate}{rgb}{0.85, 0.88, 0.95} % 216/225/229


\newcommand{\includefigure}[2][]{{\centering\includegraphics[#1]{#2}\par}}
\newcommand{\highlight}[1]{{\bf\usebeamercolor[fg]{structure}#1}}
\newcommand{\important}[1]{{\color{red}#1}}
\newcommand{\issue}[2]{\item[Issue:] {\color{red}#1}\\{\item[Soln:] \color{blue}#2}\\[4pt]}

\setbeamercolor{alerted text}{fg=ltgreen}
\setbeamertemplate{description item}[align left]


\newcommand{\lhs}[1]{{\color{blue}#1}}
\newcommand{\rhs}[1]{{\color{red}#1}}
\newcommand{\annotateL}[2]{%
  {\color{blue}\underbrace{\color{blue}#1}_{\color{blue}\mathclap{#2}}}}
\newcommand{\annotateR}[2]{%
  {\color{red}\underbrace{\color{red}#1}_{\color{red}\mathclap{#2}}}}
\newcommand{\eqnannotate}[2]{%
  {\color{blue}%
  \underbrace{\color{black}#1}_{\color{blue}\mathclap{#2}}}}

\newcommand{\trialvec}[1][]{{\vec{\psi}_\mathit{trial}^{#1}}}
\newcommand{\trialscalar}[1][]{{\psi_\mathit{trial}^{#1}}}
\newcommand{\basisvec}[1][]{{\vec{\psi}_\mathit{basis}^{#1}}}
\newcommand{\basisscalar}[1][]{{\psi_\mathit{basis}^{#1}}}

\newcommand{\tensor}[1]{\bm{#1}}
\DeclareMathOperator{\Tr}{Tr}

\usefonttheme[onlymath]{serif}

% minted shortcuts
\newminted{cfg}{bgcolor=ltslate,autogobble,fontsize=\tiny}
\newminted{bash}{bgcolor=ltltgreen,autogobble,fontsize=\tiny}

% PyLith components
\newcommand{\pylith}[1]{{\ttfamily\color{magenta}#1}}



% --------------------------------------------------------- DOCUMENT
\begin{document}

% ------------------------------------------------------------ SLIDE
\maketitle

\logo{\includegraphics[height=4.5ex]{../../logos/cig_blackfg}}

% ========================================================== SECTION
\section{Introduction}

% ------------------------------------------------------------ SLIDE
\begin{frame}
  \frametitle{PyLith v3.0}
  \summary{}

  \begin{itemize}
  \item Multiphysics through pointwise integration kernels
  \item Higher order spatial and temporal discretizations
  \item Adaptive time stepping via PETSc TS
  \item Improved fault formulation for spontaneous rupture (v3.1)
  \end{itemize}

\end{frame}

\abovedisplayskip=2pt
\abovedisplayshortskip=2pt
\belowdisplayskip=2pt
\belowdisplayshortskip=2pt

% ------------------------------------------------------------ SLIDE
\begin{frame}
  \frametitle{Aside: Finite-Element Method}
  \summary{Strong form to weak form}

  Solve governing equation in integrated sense:
  \begin{equation}
    \int_\Omega \trialscalar \cdot \mathit{PDE} \, d\Omega = 0,
  \end{equation}
  by minimizing the error with respect to the unknown coefficients.
  \vfill
  This leads to equations of the form:
  \begin{equation}
    \int_\Omega \trialscalar \cdot f_0(x,t) + \nabla \trialscalar \cdot f_1(x,t) \, d\Omega = 0.
  \end{equation}

\end{frame}

% ------------------------------------------------------------ SLIDE
\begin{frame}
  \frametitle{Governing Equations}
  \summary{}

  We want to solve equations in which the weak form can be expressed
  as
  \begin{gather}
    \lhs{F(t,s,\dot{s})} = \rhs{G(t,s)} \\
    s(t_0) = s_0
  \end{gather}
  where $\lhs{F}$ and $\rhs{G}$ are vector functions, $t$ is time, and $s$ is the solution vector.
  \vfill
  Using the finite-element method and divergence theorem, we cast the weak form into
  \begin{multline}
    \int_\Omega \trialvec \cdot \lhs{\vec{f}_0(t,s,\dot{s})} + \nabla \trialvec : \lhs{\tensor{f}
    _1(t,s,\dot{s})} \, 
    d\Omega = \\
    \int_\Omega \trialvec \cdot \rhs{\vec{g}_0(t,s)} + \nabla \trialvec : \rhs{\tensor{g}_1(t,s)} \, 
    d\Omega,
  \end{multline}
  where $\vec{f}_0$ and $\vec{g}_0$ are vectors, and $\tensor{f}_1$ and
  $\tensor{g}_1$ are tensors.

\end{frame}

% ------------------------------------------------------------ SLIDE
\begin{frame}
  \frametitle{Explicit Time Stepping}
  \summary{}

  Explicit time stepping with the PETSc TS requires $\lhs{F(t,s,\dot{s})} = \dot{s}$.
  \vfill
  Normally $\lhs{F(t,s,\dot{s})}$ contains the inertial term ($\rho \ddot{u}$).
  \vfill
  Therefore, when using explicit time stepping we transform our equation into the form:
  \begin{align}
    \lhs{F^*(t,s,\dot{s})} = \lhs{\dot{s}} &= \rhs{G^*(t,s)} \\
    \lhs{\dot{s}} &= \lhs{M^{-1}}\rhs{G(t,s)}.
  \end{align}  

\end{frame}

% ------------------------------------------------------------ SLIDE
\begin{frame}
  \frametitle{Solving the Equations}
  \summary{Explicit time stepping requires a subset of the terms used in implicit time stepping.}

  \begin{itemize}
  \item PETSc TS object provides time-stepping and solver implementations
    \begin{itemize}
    \item Application code provides functions for computing RHS and LHS residuals and Jacobians
    \end{itemize}
  \item Explicit time stepping
    \begin{itemize}
    \item Compute RHS residual, $\rhs{G(t,s)}$
    \item Compute lumped inverse of LHS, $\lhs{M^{-1}}$
    \item No need to compute LHS residual, because $\lhs{F(t,s,\dot{s})} = \dot{s}$
    \end{itemize}
  \item Implicit time stepping (Krylov solvers)
    \begin{itemize}
    \item Compute RHS residual, $\rhs{G(t,s)}$
    \item Compute LHS residual, $\lhs{F(t,s,\dot{s})}$
    \item Compute RHS Jacobian, $\rhs{J_G}$ = $\rhs{\frac{\partial G}{\partial s}}$
    \item Compute LHS Jacobian, $\lhs{J_F}$ = $\lhs{\frac{\partial F}{\partial s}} + t_\mathit{shift}\lhs{\frac{\partial F}{\partial \dot{s}}}$
    \end{itemize}
  \end{itemize}

\end{frame}

% ------------------------------------------------------------ SLIDE
\begin{frame}
  \frametitle{Example: Elasticity}
  \summary{}

      \begin{align}
        \lhs{\rho \frac{\partial^2\vec{u}}{\partial t^2}} &= \rhs{\vec{f}(\vec{x},t) + \tensor{\nabla} \cdot \tensor{\sigma}(\vec{u})} \text{ in } \Omega, \\
        %
        \tensor{\sigma} \cdot \vec{n} &= \vec{\tau}(\vec{x},t) \text{ on } \Omega_\tau, \\
        %
        \vec{u} &= \vec{u}_0(\vec{x},t) \text{ on } \Omega_u,
      \end{align}

      \vfill

      {\bf Implicit Time Stepping without Inertia}

      Displacement $\vec{u}$ is the unknown, $\vec{s}= \vec{u}$.
      \begin{align}
        \annotateL{0}{\vec{f}_0^u} &=
  \int_\Omega \trialvec[u] \cdot \annotateR{\vec{f}(\vec{x},t)}{\vec{g}_0^u} + \nabla \trialvec[u] : \annotateR{-\tensor{\sigma}(\vec{u})}{\tensor{g}_1^u} \, d\Omega + \int_{\Omega_\tau} \trialvec[u] \cdot \annotateR{\vec{\tau}(\vec{x},t)}{\vec{g}_0^u} \, d\Omega_\tau
      \end{align}

      \vfill
      Different constitutive models are encapsulated in alternative
      kernels for $\tensor{\sigma}(\vec{u})$.

\end{frame}

% ------------------------------------------------------------ SLIDE
\begin{frame}
  \frametitle{Example: Elasticity (continued)}
  \summary{}

      {\bf Explicit Time Stepping with Inertia}

      Form a first order equation using displacement $\vec{u}$ and velocity $\vec{v}$ as
      unknowns, $\vec{s}^T = \left( \begin{array}{cc} \vec{u} & \vec{v} \end{array} \right)^T$
      \begin{align}
        \int_\Omega \trialvec[u] \cdot \annotateL{\frac{\partial \vec{u}}{\partial t}}{\dot{u}} \, d\Omega &= 
        \int_\Omega \trialvec[u] \cdot \annotateR{\vec{v}}{\vec{g}_0^u} \, d\Omega, \\
        %
        \int_\Omega \trialvec[v] \cdot \annotateL{\frac{\partial \vec{v}}{\partial t}}{\dot{v}} \, d\Omega &=
\frac{1}{\vec{m}} \left(
  \int_\Omega \trialvec[v] \cdot \annotateR{\vec{f}(\vec{x},t)}{\vec{g}_0^v} + \nabla \trialvec[v] : \annotateR{-\tensor{\sigma}(\vec{u})}{\tensor{g}_1^v} \, d\Omega + \int_{\Omega_\tau} \trialvec[v] \cdot \annotateR{\vec{\tau}(\vec{x},t)}{\vec{g}_0^v} \, d\Omega_\tau \right), \\
\vec{m} &= \int_\Omega \trialvec[v] \cdot {\annotateL{\rho}{J_{f0}^{vv}}} \basisvec[v] \, d\Omega
      \end{align}


\end{frame}

% ------------------------------------------------------------ SLIDE
\begin{frame}
  \frametitle{Example: Elasticity (continued)}
  \summary{Implementing the governing equations involves a small set of simple kernels.}

  {\centering
  \begin{tabular}{ccc}
    & {\bf Implicit} & {\bf Explicit} \\ \hline
    $\rhs{\vec{v}}$   &   --   &   $\rhs{\vec{g}_0^v}$ \\
    $\rhs{\vec{f}(\vec{x},t)}$   &   $\rhs{\vec{g}_0^u}$   &   $\rhs{\vec{g}_0^v}$ \\
    $\rhs{-\tensor{\sigma}(\vec{u})}$   & $\rhs{\tensor{g}_1^u}$   &   $\rhs{\tensor{g}_1^v}$ \\
    $\rhs{\vec{\tau}(\vec{x},t)}$   &   $\rhs{\vec{g}_0^u}$   &   $\rhs{\vec{g}_0^v}$ \\
    $\lhs{\vec{0}}$   &   $\lhs{\vec{f}_0^u}$ & -- \\
    $\lhs{\rho}$ & -- & $\lhs{J_{f_0}^{uv}}$ \\
    \hline
  \end{tabular}\par}
  \vfill
  We also have simple kernels for the Jacobians needed in implicit time stepping.

\end{frame}

% ------------------------------------------------------------ SLIDE
\begin{frame}[fragile]
  \frametitle{Example: Elasticity Stress Kernels for Residual}
  \summary{}

\begin{itemize}
\item Volumetric Stress
\lstset{language=C}
\begin{lstlisting}[basicstyle=\tiny\ttfamily]
for (i=0; i < _dim; ++i) {
   trace += disp_x[i] - initialstrain[i*_dim+i];
   meanistress += initialstress[i*_dim+i];
}
meanistress /= (PylithReal) _dim;
for (i = 0; i < _dim; ++i) {
   stress[i*_dim+i] += lambda * trace + meanistress;
}
\end{lstlisting}
\item Deviatoric Stress
\begin{lstlisting}[basicstyle=\tiny\ttfamily]
for (i=0; i < _dim; ++i) {
    meanistress += initialstress[i*_dim+i];
}
meanistress /= (PylithReal) _dim;
    for (i=0; i < _dim; ++i) {
      for (j=0; j < _dim; ++j) {
          stress[i*_dim+j] += mu * (disp_x[i*_dim+j] + disp_x[j*_dim+i]
              - initialstrain[i*_dim+j]) + initialstress[i*_dim+j];
    }
    stress[i*_dim+i] -= meanistress;
}
\end{lstlisting}
\end{itemize}

\end{frame}

% ------------------------------------------------------------ SLIDE
\begin{frame}
  \frametitle{Example: Poroelasticity Neglecting Inertia}
  \summary{}

        We assume a compressible fluid completely saturates a porous
        solid undergoing infinitesimal strain.

        Elasticity equilibrium equation neglecting inertia:
        \begin{alignat}{2}
          0 &= \rhs{\vec{f}(\vec{x},t) + \tensor{\nabla} \cdot \tensor{\sigma}(\vec{u},p_f)} \text{ in } \Omega, 
          & \quad
          \tensor{\sigma} \cdot \vec{n} = \vec{\tau}(\vec{x},t) \text{ on } \Omega_\tau, 
          %
          \vec{u} = \vec{u}_0(\vec{x},t) \text{ on } \Omega_u,
        \end{alignat}
        Mass balance of the fluid:
        \begin{alignat}{2}
          \lhs{\frac{\partial \zeta(\vec{u},p_f)}{\partial t}} = \rhs{\gamma(\vec{x},t) - \nabla \cdot \vec{q}(p_f)} \text{ in }\Omega,
          & \quad
          \vec{q} \cdot \vec{n} = q_0(\vec{x},t) \text{ on }\Omega_q,
          %
          p_f = p_0(\vec{x},t) \text{ on }\Omega_p,
        \end{alignat}
        Darcy's law:
        \begin{alignat}{2}
          \vec{q}(p_f) = -\kappa (\nabla p_f - \vec{f}_f), 
          & \quad
          \kappa = \frac{k}{\eta_f}
        \end{alignat}
        Constitutive behavior of the fluid:
        \begin{alignat}{2}
          \zeta(\vec{u},p_f) = \alpha (\nabla \cdot \vec{u}) + \frac{p_f}{M}, 
          & \quad
          \frac{1}{M} = \frac{\alpha-\phi}{K_s} + \frac{\phi}{K_f},
        \end{alignat}
        Constitutive behavior of the solid (linear elasticity):
        \begin{equation}
          \tensor{\sigma}(\vec{u},p_f) = \tensor{C} : \tensor{\epsilon} - \alpha p_f \tensor{I}
        \end{equation}


\end{frame}

% ------------------------------------------------------------ SLIDE
\begin{frame}
  \frametitle{Example: Poroelasticity Neglecting Inertia}
  \summary{}

      Consider displacement $\vec{u}$ and fluid pressure $p_f$ as
      unknowns, $\vec{s}^T = \left( \begin{array}{cc} \vec{u} & p_f \end{array} \right)^T$
      \begin{align}
        \annotateL{0}{\vec{f}_0^u} &= \int_\Omega \trialvec[u] \cdot \annotateR{\vec{f}(\vec{x},t)}{\vec{g}^u_0} + \nabla \trialvec[u] : \annotateR{-\tensor{\sigma}(\vec{u},p_f)}{\tensor{g}^u_1} \, d\Omega + \int_{\Omega_\tau} \trialvec[u] \cdot \annotateR{\vec{\tau}(\vec{x},t)}{\vec{g}^u_0} \, d\Omega_\tau, \\
%
 \int_\Omega  \trialscalar[p] \annotateL{\frac{\partial \zeta(\vec{u},p_f)}{\partial t}}{f^p_0} \, d\Omega &= 
 \int_\Omega \trialscalar[p] \annotateR{\gamma(\vec{x},t)}{g^p_0} + \nabla \trialscalar[p] \cdot \annotateR{\vec{q}(p_f)}{\vec{f}^p_1} \, d\Omega
 + \int_{\Omega_q} \trialscalar[p] \annotateR{(-q_0(\vec{x},t))}{g^p_0} \, d\Omega_q.
      \end{align}

      Poroelasticity involves many of the same kernels as elasticity plus a few additional ones.

\end{frame}

% ------------------------------------------------------------ SLIDE
\begin{frame}
  \frametitle{Finite-Element Discretization}
  \summary{Specify discretizations for solution fields and auxiliary fields}

  \begin{itemize}
  \item Solution Fields \\
    Specify basis functions and quadrature for each field in solution.
  \item Auxiliary Fields 
    \begin{itemize}
    \item Fields associated with parameters and state variables for constitutive models \& boundary conditions.
    \item Populated from spatial databases.
    \item Specify basis functions for each subfield in the auxiliary fields.
    \end{itemize}
  \item PETSc DMPlex infrastructure unpacks/packs
    information to/from solution and auxiliary fields and calling
    finite-element kernels.
  \end{itemize}


\end{frame}

% ------------------------------------------------------------ SLIDE
\begin{frame}
  \frametitle{Summary of Multiphysics Implementation}
  \summary{}

  We decouple the element definition from the fully-coupled equation,
  using pointwise kernels that look like the PDE.

  \vfill    
  \begin{description}
  \item[Flexibility] The cell traversal, handled by the library,
    accommodates arbitrary cell shapes. The problem can be posed
    in any spatial dimension with an arbitrary number of
    physical fields.
  \item[Extensibility] The library developer needs to maintain only a
    single method, easing language transitions (CUDA, OpenCL). A new
    discretization scheme could be enabled in a single place in the
    code.
  \item[Efficiency] Only a single routine needs to be
    optimized. The application scientist is no longer
    responsible for proper vectorization, tiling, and other
    traversal optimization.
  \end{description}

\end{frame}


% --------------------------------------------------------- DOCUMENT
\end{document}

% End of file
