% -*- TeX -*-
%
% ----------------------------------------------------------------------
%
%                           Brad T. Aagaard
%                        U.S. Geological Survey
%
% ----------------------------------------------------------------------
%
\documentclass[aspectratio=169]{beamer}

\usepackage{listings}

\title{Crustal Deformation Modeling Tutorial}
\subtitle{Prescribed Fault Slip}
\author{Brad Aagaard \\
  Charles Williams \\
  Matthew Knepley}
\institute{\includegraphics[scale=0.4]{../../logos/cig_blackfg}}
\date{June 26, 2016}


% ---------------------------------------------------- CUSTOMIZATION
%\newcommand{\thispdfpagelabel}[1]{}
\usetheme{CIG}

% Style information for PyLith presentations.

% Colors
\definecolor{ltorange}{rgb}{1.0, 0.74, 0.41} % 255/188/105
\definecolor{orange}{rgb}{0.96, 0.50, 0.0} % 246/127/0

\definecolor{ltred}{rgb}{1.0, 0.25, 0.25} % 255/64/64
\definecolor{red}{rgb}{0.79, 0.00, 0.01} % 201/0/3

\definecolor{ltpurple}{rgb}{0.81, 0.57, 1.00} % 206/145/255
\definecolor{purple}{rgb}{0.38, 0.00, 0.68} % 97/1/175

\definecolor{ltblue}{rgb}{0.2, 0.73, 1.0} % 51/187/255
\definecolor{mdblue}{rgb}{0.28, 0.50, 0.80} % 72/128/205
\definecolor{blue}{rgb}{0.12, 0.43, 0.59} % 30/110/150

\definecolor{ltltgreen}{rgb}{0.7, 1.00, 0.7} % 96/204/14
\definecolor{ltgreen}{rgb}{0.37, 0.80, 0.05} % 96/204/14
\definecolor{green}{rgb}{0.23, 0.49, 0.03} % 59/125/8
  
\definecolor{dkslate}{rgb}{0.18, 0.21, 0.28} % 47/53/72
\definecolor{mdslate}{rgb}{0.45, 0.50, 0.68} % 114/127/173
\definecolor{ltslate}{rgb}{0.85, 0.88, 0.95} % 216/225/229


\newcommand{\includefigure}[2][]{{\centering\includegraphics[#1]{#2}\par}}
\newcommand{\highlight}[1]{{\bf\usebeamercolor[fg]{structure}#1}}
\newcommand{\important}[1]{{\color{red}#1}}
\newcommand{\issue}[2]{\item[Issue:] {\color{red}#1}\\{\item[Soln:] \color{blue}#2}\\[4pt]}

\setbeamercolor{alerted text}{fg=ltgreen}
\setbeamertemplate{description item}[align left]


\newcommand{\lhs}[1]{{\color{blue}#1}}
\newcommand{\rhs}[1]{{\color{red}#1}}
\newcommand{\annotateL}[2]{%
  {\color{blue}\underbrace{\color{blue}#1}_{\color{blue}\mathclap{#2}}}}
\newcommand{\annotateR}[2]{%
  {\color{red}\underbrace{\color{red}#1}_{\color{red}\mathclap{#2}}}}
\newcommand{\eqnannotate}[2]{%
  {\color{blue}%
  \underbrace{\color{black}#1}_{\color{blue}\mathclap{#2}}}}

\newcommand{\trialvec}[1][]{{\vec{\psi}_\mathit{trial}^{#1}}}
\newcommand{\trialscalar}[1][]{{\psi_\mathit{trial}^{#1}}}
\newcommand{\basisvec}[1][]{{\vec{\psi}_\mathit{basis}^{#1}}}
\newcommand{\basisscalar}[1][]{{\psi_\mathit{basis}^{#1}}}

\newcommand{\tensor}[1]{\bm{#1}}
\DeclareMathOperator{\Tr}{Tr}

\usefonttheme[onlymath]{serif}

% minted shortcuts
\newminted{cfg}{bgcolor=ltslate,autogobble,fontsize=\tiny}
\newminted{bash}{bgcolor=ltltgreen,autogobble,fontsize=\tiny}

% PyLith components
\newcommand{\pylith}[1]{{\ttfamily\color{magenta}#1}}



% ========================================================= DOCUMENT
\begin{document}

% ------------------------------------------------------------ SLIDE
\maketitle

% ------------------------------------------------------------ SLIDE
\logo{\includegraphics[height=4.5ex]{../../logos/cig_blackfg}}

% ========================================================== SECTION
\section{Overview}

% ------------------------------------------------------------ SLIDE
\begin{frame}
  \frametitle{3-D Subduction Zone: Steps 1--4}
  \summary{Dirichlet boundary conditions and prescribed slip}

  \begin{description}
  \item[Step 1] Axial compression (discussed in manual)
  \item[Step 2] Viscoelastic relaxation from coseismic slip on central
    fault patch
  \item[Step 3] Interseismic deformation with prescribed creep \& viscoelastic materials
  \item[Step 4] Earthquake cycle with prescribed creep and
    earthquake ruptures
  \end{description}
  
\end{frame}

% ========================================================== SECTION
\section{Step 2}

% ------------------------------------------------------------ SLIDE
\begin{frame}
  \frametitle{Step 2: Uniform Slip on Central Patch}
  \summary{Viscoelastic response to coseismic slip on subduction interface}

  \includefigure[height=6.9cm]{figs/subduction3d_step02_diagram}
  
\end{frame}

% ------------------------------------------------------------ SLIDE
\begin{frame}
  \frametitle{Step 2: Boundary Conditions}
  \summary{}

  \begin{itemize}
  \item \highlight{How do we want the model to behave at the boundaries?}\pause
    \begin{itemize}
    \item We expect displacements to go to zero as distance increases.
    \item Might have some tangential motion at the boundaries.
    \item Expect motion perpendicular to the boundaries to be smaller.\pause
    \item $\Rightarrow$ Impose zero displacements perpendicular to the boundary.\pause
    \end{itemize}
  \item \highlight{How many boundary conditions do we need?}\pause
    \begin{itemize}
    \item $\Rightarrow$ Five: east, west, north, south, bottom
    \end{itemize}
  \end{itemize}  
  
\end{frame}


% ------------------------------------------------------------ SLIDE
\begin{frame}
  \frametitle{Step 2: Material Properties}
  \summary{}

  \begin{itemize}
  \item \highlight{How many materials do we need?}\pause
    \begin{itemize}
    \item $\Rightarrow$ Four: slab, mantle, crust, wedge\pause
    \end{itemize}
  \item \highlight{Which materials should we make viscoelastic?}\pause
    \begin{itemize}
    \item $\Rightarrow$ Linear Maxwell model w/depth dependent viscosity: mantle, slab
    \item $\Rightarrow$ Elastic model: crust, wedge
    \end{itemize}
  \end{itemize}
  
\end{frame}


% ------------------------------------------------------------ SLIDE
\begin{frame}
  \frametitle{Step 2: Prescribed Slip on Subduction Interface Patch}
  \summary{}

  \begin{itemize}
  \item \highlight{How many faults do we need?}\pause
    \begin{itemize}
    \item $\Rightarrow$ One: central patch of subduction interface\pause
    \end{itemize}
  \item \highlight{What type of slip time function should we use?}\pause
    \begin{itemize}
    \item $\Rightarrow$ Step slip-time function
    \item $\Rightarrow$ Impose slip at 10 years
    \end{itemize}
  \end{itemize}
  
\end{frame}


% ------------------------------------------------------------ SLIDE
\begin{frame}
  \frametitle{Step 2: Time Stepping}
  \summary{}

  \begin{itemize}
  \item \highlight{What duration should we use for our simulation?}\pause
    \begin{itemize}
    \item $\Rightarrow$ 200 years (roughly Maxwell relaxation time)\pause
    \end{itemize}
  \item \highlight{What should we use for a time step?}\pause
    \begin{itemize}
    \item $\Rightarrow$ 10 year (0.05--0.1 of the shortest Maxwell time)
    \end{itemize}
  \end{itemize}
  
\end{frame}


% ------------------------------------------------------------ SLIDE
\begin{frame}[fragile]
  \frametitle{Step 2: Tour of Input Files}
  \summary{Mesh is georeferenced, so georeference our parameters as well}

  \begin{description}
  \item[{\tt pylithapp.cfg}] Parameters (mostly) common to Steps 1--8
  \item[{\tt step02.cfg}] Parameters specific to Step 2
  \item[{\tt mat\_viscoelastic.cfg}] Material settings
  \item[{\tt solver\_fieldsplit.cfg}] Solver settings
  \item[{\tt spatialdb/mat\_viscosity.spatialdb}] Viscosity
    spatial database
  \item[{\tt spatialdb/fault\_slabtop\_coseismic.spatialdb}] Fault
    slip spatial database
  \end{description}

  \vfill
  Note: See the {\tt step02.cfg} file for a list of all {\tt .cfg}
  files used in this simulation.
  \vfill

  Run the simulation:\\
  \important{\tt pylith step02.cfg mat\_viscoelastic.cfg solver\_fieldsplit.cfg}
  
\end{frame}


% ========================================================== SECTION
\section{Step 3}

% ------------------------------------------------------------ SLIDE
\begin{frame}
  \frametitle{Step 3:  Interseismic Deformation}
  \summary{}

  \includefigure[height=6.9cm]{figs/subduction3d_step03_diagram}
  
\end{frame}

% ------------------------------------------------------------ SLIDE
\begin{frame}
  \frametitle{Step 3: Boundary Conditions}
  \summary{}

  \begin{itemize}
  \item \highlight{How do we want the model to behave at the boundaries?}\pause
    \begin{itemize}
    \item Expect displacements in mantle to go to zero as distance increases.
    \item Want slab to move to the east and downward.
    \item Expect north-south motion on north and south boundaries to be small.\pause
    \item $\Rightarrow$ Impose zero displacements perpendicular to the
      boundary, except on slab (free).\pause
    \end{itemize}
  \item \highlight{How many boundary conditions do we need?}\pause
    \begin{itemize}
    \item $\Rightarrow$ Five: east, west, north, south, bottom
    \end{itemize}
  \end{itemize}  
  
\end{frame}


% ------------------------------------------------------------ SLIDE
\begin{frame}
  \frametitle{Step 3: Material Properties}
  \summary{}

  \begin{center}
    \highlight{Same as Step 2.}
  \end{center}
  
  \vfill
  
  \begin{itemize}
  \item Linear Maxwell model w/depth dependent viscosity: mantle, slab
  \item Elastic model: crust, wedge
  \end{itemize}
  
\end{frame}


% ------------------------------------------------------------ SLIDE
\begin{frame}
  \frametitle{Step 3: Prescribed Creep on Slab}
  \summary{}

  \begin{itemize}
  \item \highlight{How many faults do we need?}\pause
    \begin{itemize}
    \item $\Rightarrow$ Two: Top and bottom of the slab\pause
    \end{itemize}
  \item \highlight{What type of slip time function should we use?}\pause
    \begin{itemize}
    \item $\Rightarrow$ Constant slip rate slip-time function\pause
    \end{itemize}
  \item \highlight{What sense of slip do we impose on the faults?}\pause
    \begin{itemize}
    \item $\Rightarrow$ Subduction interface: reverse w/right-lateral\pause
    \item $\Rightarrow$ Subduction interface: normal w/left-lateral
    \end{itemize}
  \end{itemize}
  
\end{frame}


% ------------------------------------------------------------ SLIDE
\begin{frame}
  \frametitle{Step 3: Tour of Input Files}
  \summary{}

  \begin{description}
  \item[{\tt step03.cfg}] Parameters specific to Step 3
  \item[{\tt spatialdb/fault\_slabtop\_creep.cfg}] Fault slip spatial database
  \end{description}
  
  \vfill
  Note: See the {\tt step03.cfg} file for a list of all {\tt .cfg}
  files used in this simulation.
  \vfill

  Run the simulation:\\
  \important{\tt pylith step03.cfg mat\_viscoelastic.cfg solver\_fieldsplit.cfg}

\end{frame}


% ========================================================== SECTION
\section{Step 4}

% ------------------------------------------------------------ SLIDE
\begin{frame}
  \frametitle{Step 4: Earthquake Cycle w/Prescribed Earthquakes}
  \summary{}

  \includefigure[height=6.9cm]{figs/subduction3d_step04_diagram}
  
\end{frame}

% ------------------------------------------------------------ SLIDE
\begin{frame}
  \frametitle{Step 4: Boundary Conditions and Materials}
  \summary{}

  \begin{center}
    \highlight{Same as Step 3.}
  \end{center}
  
\end{frame}


% ------------------------------------------------------------ SLIDE
\begin{frame}
  \frametitle{Step 4: Earthquake Cycle}
  \summary{}

  \begin{itemize}
  \item Creep on deep portion of subduction interface
  \item Creep on bottom of slab
  \item Earthquake at 100 years and 200 years on subduction interface
  \item Earthquake at 150 years on splay fault
  \end{itemize}
  
  \begin{itemize}
  \item \highlight{How many faults do we need?}\pause
    \begin{itemize}
    \item $\Rightarrow$ Three: Top and bottom of the slab plus splay fault
    \end{itemize}
  \item \highlight{How many earthquake sources do we need?}\pause
    \begin{itemize}
    \item Splay fault?\pause \quad $\Rightarrow$ 1\pause
    \item Bottom of slab?\pause \quad $\Rightarrow$ 1\pause
    \item Top of slab?\pause \quad $\Rightarrow$ 3
    \end{itemize}
  \end{itemize}
  
\end{frame}


% ------------------------------------------------------------ SLIDE
\begin{frame}
  \frametitle{Step 4: Tour of Input Files}
  \summary{}

  \begin{description}
  \item[{\tt step04.cfg}] Parameters specific to Step 4
  \item[{\tt spatialdb/fault\_slabtop\_creep.cfg}] Fault slip spatial database
  \end{description}
  
  \vfill
  Note: See the {\tt step04.cfg} file for a list of all {\tt .cfg}
  files used in this simulation.
  \vfill

  Run the simulation:\\
  \important{\tt pylith step04.cfg mat\_viscoelastic.cfg solver\_fieldsplit.cfg}

\end{frame}




% ------------------------------------------------------------ SLIDE
\begin{frame}
  \frametitle{Group Exercise}
  \summary{}
 
  \highlight{Work in groups of 3--4 to complete some of the exercises listed
  in the manual.}

  \begin{itemize}
  \item Easy
    \begin{itemize}
    \item Adjust values for material properties and faults
    \item Change the slip in Step 2 to the splay fault
    \end{itemize}
  \item Intermediate
    \begin{itemize}
    \item Step 2: Create simultaneous rupture on the subduction
      interface rupture patch and the splay fault rupture patch.
    \item Prescribe coseismic slip on the central patch for splay fault
      and the subduction interface below the intersection with the splay fault.
    \item Add additional earthquakes with different amplitudes and
      depth variations in slip, keeping the total equal to the overall
      slip rate.
    \end{itemize}
  \item Advanced
    \begin{itemize}
    \item Make the splay fault and the deeper portion of the
      subduction interface form the through-going fault and the upper
      portion of the subduction interface is the secondary fault.
    \end{itemize}
  \end{itemize}

\end{frame}


% ======================================================================
\end{document}


% End of file
