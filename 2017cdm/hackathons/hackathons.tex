% -*- TeX -*-
\documentclass[aspectratio=169]{beamer}
%\documentclass{beamer}

\usepackage{listings}
\usepackage{amsmath}
\usepackage{tikz}
\usetikzlibrary{shapes,calc}

\title{Crustal Deformation Modeling Tutorial}
\subtitle{PyLith Hackathons}
\author{Brad Aagaard, Charles Williams, and Matthew Knepley}
\institute{\includegraphics[scale=0.4]{../../logos/cig_blackfg}}
\date{June 27, 2017}


% ---------------------------------------------------- CUSTOMIZATION
\usetheme{CIG}

% Style information for PyLith presentations.

% Colors
\definecolor{ltorange}{rgb}{1.0, 0.74, 0.41} % 255/188/105
\definecolor{orange}{rgb}{0.96, 0.50, 0.0} % 246/127/0

\definecolor{ltred}{rgb}{1.0, 0.25, 0.25} % 255/64/64
\definecolor{red}{rgb}{0.79, 0.00, 0.01} % 201/0/3

\definecolor{ltpurple}{rgb}{0.81, 0.57, 1.00} % 206/145/255
\definecolor{purple}{rgb}{0.38, 0.00, 0.68} % 97/1/175

\definecolor{ltblue}{rgb}{0.2, 0.73, 1.0} % 51/187/255
\definecolor{mdblue}{rgb}{0.28, 0.50, 0.80} % 72/128/205
\definecolor{blue}{rgb}{0.12, 0.43, 0.59} % 30/110/150

\definecolor{ltltgreen}{rgb}{0.7, 1.00, 0.7} % 96/204/14
\definecolor{ltgreen}{rgb}{0.37, 0.80, 0.05} % 96/204/14
\definecolor{green}{rgb}{0.23, 0.49, 0.03} % 59/125/8
  
\definecolor{dkslate}{rgb}{0.18, 0.21, 0.28} % 47/53/72
\definecolor{mdslate}{rgb}{0.45, 0.50, 0.68} % 114/127/173
\definecolor{ltslate}{rgb}{0.85, 0.88, 0.95} % 216/225/229


\newcommand{\includefigure}[2][]{{\centering\includegraphics[#1]{#2}\par}}
\newcommand{\highlight}[1]{{\bf\usebeamercolor[fg]{structure}#1}}
\newcommand{\important}[1]{{\color{red}#1}}
\newcommand{\issue}[2]{\item[Issue:] {\color{red}#1}\\{\item[Soln:] \color{blue}#2}\\[4pt]}

\setbeamercolor{alerted text}{fg=ltgreen}
\setbeamertemplate{description item}[align left]


\newcommand{\lhs}[1]{{\color{blue}#1}}
\newcommand{\rhs}[1]{{\color{red}#1}}
\newcommand{\annotateL}[2]{%
  {\color{blue}\underbrace{\color{blue}#1}_{\color{blue}\mathclap{#2}}}}
\newcommand{\annotateR}[2]{%
  {\color{red}\underbrace{\color{red}#1}_{\color{red}\mathclap{#2}}}}
\newcommand{\eqnannotate}[2]{%
  {\color{blue}%
  \underbrace{\color{black}#1}_{\color{blue}\mathclap{#2}}}}

\newcommand{\trialvec}[1][]{{\vec{\psi}_\mathit{trial}^{#1}}}
\newcommand{\trialscalar}[1][]{{\psi_\mathit{trial}^{#1}}}
\newcommand{\basisvec}[1][]{{\vec{\psi}_\mathit{basis}^{#1}}}
\newcommand{\basisscalar}[1][]{{\psi_\mathit{basis}^{#1}}}

\newcommand{\tensor}[1]{\bm{#1}}
\DeclareMathOperator{\Tr}{Tr}

\usefonttheme[onlymath]{serif}

% minted shortcuts
\newminted{cfg}{bgcolor=ltslate,autogobble,fontsize=\tiny}
\newminted{bash}{bgcolor=ltltgreen,autogobble,fontsize=\tiny}

% PyLith components
\newcommand{\pylith}[1]{{\ttfamily\color{magenta}#1}}



% --------------------------------------------------------- DOCUMENT
\begin{document}

% ------------------------------------------------------------ SLIDE
\maketitle

% ------------------------------------------------------------- LOGO
\logo{\includegraphics[height=4.5ex]{../../logos/cig_blackfg}}

% ------------------------------------------------------------ SLIDE
\begin{frame}
  \frametitle{Community Tools for Crustal Deformation}
  \summary{}
  
  \important{Objective: Increase number of developers contributing to
    a coherent crustal deformation modeling workflow}

  \vfill
  \begin{itemize}
  \item Hackathon: Informal meetings ($\approx$week) focused on code
    development. Participants focus on particular topic of research
    interest.
  \item Help improve manual
    \begin{itemize}
    \item Catch mistakes, suggest better wording, etc.
    \item Improve/contribute examples in collaboration with PyLith developers.
    \end{itemize}
  \end{itemize}
\end{frame}


% ------------------------------------------------------------ SLIDE
\begin{frame}
  \frametitle{Hackathon}
  \summary{}
  
  \begin{itemize}
  \item PyLith development
    \begin{itemize}
    \item Particle advection for material properties or fault evolution
    \item Adjoint formulations for large data assimilation
    \item Import MED mesh files from Gmsh/Salome/OpenCASCADE
    \item Fault influence in adaptive time stepping
    \end{itemize}
  \item Integrated workflow
    \begin{itemize}
    \item Fault surface construction from Slab 1.0 contours (via Python)
    \item Fault surface construction from UCERF3 fault model (via Python)
    \end{itemize}
  \item Spatial databases
    \begin{itemize}
    \item Arbitrary points interpolated via radial basis functions
    \item Efficient spatial database from output
    \item Interfaces to common community seismic velocity models
    \end{itemize}
  \end{itemize}

\end{frame}


 
% ======================================================================
\end{document}


% End of file
