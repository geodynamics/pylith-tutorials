% -*- TeX -*-
\documentclass[aspectratio=169,hyperref=colorlinks]{beamer}

\usepackage{mathtools}
\usepackage{bm}
\usepackage{listings}
\usepackage{booktabs}
%\usepackage{enumitem}
\usepackage{cancel}

\usepackage{tikz}
\usetikzlibrary{shapes,arrows,calc}
\usetikzlibrary{decorations.pathreplacing}
\usetikzlibrary{fit,matrix}



\title{Poroelasticity Implementation}
\subtitle{PyLith Version 3}
\author{Robert Walker}
\institute{\includegraphics[scale=0.4]{../../logos/cig_blackfg}}
\date{16 March 2021}

% ---------------------------------------------------- CUSTOMIZATION
%\renewcommand{\thispdfpagelabel}[1]{}
\usetheme{CIG}

% Style information for PyLith presentations.

% Colors
\definecolor{ltorange}{rgb}{1.0, 0.74, 0.41} % 255/188/105
\definecolor{orange}{rgb}{0.96, 0.50, 0.0} % 246/127/0

\definecolor{ltred}{rgb}{1.0, 0.25, 0.25} % 255/64/64
\definecolor{red}{rgb}{0.79, 0.00, 0.01} % 201/0/3

\definecolor{ltpurple}{rgb}{0.81, 0.57, 1.00} % 206/145/255
\definecolor{purple}{rgb}{0.38, 0.00, 0.68} % 97/1/175

\definecolor{ltblue}{rgb}{0.2, 0.73, 1.0} % 51/187/255
\definecolor{mdblue}{rgb}{0.28, 0.50, 0.80} % 72/128/205
\definecolor{blue}{rgb}{0.12, 0.43, 0.59} % 30/110/150

\definecolor{ltltgreen}{rgb}{0.7, 1.00, 0.7} % 96/204/14
\definecolor{ltgreen}{rgb}{0.37, 0.80, 0.05} % 96/204/14
\definecolor{green}{rgb}{0.23, 0.49, 0.03} % 59/125/8
  
\definecolor{dkslate}{rgb}{0.18, 0.21, 0.28} % 47/53/72
\definecolor{mdslate}{rgb}{0.45, 0.50, 0.68} % 114/127/173
\definecolor{ltslate}{rgb}{0.85, 0.88, 0.95} % 216/225/229


\newcommand{\includefigure}[2][]{{\centering\includegraphics[#1]{#2}\par}}
\newcommand{\highlight}[1]{{\bf\usebeamercolor[fg]{structure}#1}}
\newcommand{\important}[1]{{\color{red}#1}}
\newcommand{\issue}[2]{\item[Issue:] {\color{red}#1}\\{\item[Soln:] \color{blue}#2}\\[4pt]}

\setbeamercolor{alerted text}{fg=ltgreen}
\setbeamertemplate{description item}[align left]


\newcommand{\lhs}[1]{{\color{blue}#1}}
\newcommand{\rhs}[1]{{\color{red}#1}}
\newcommand{\annotateL}[2]{%
  {\color{blue}\underbrace{\color{blue}#1}_{\color{blue}\mathclap{#2}}}}
\newcommand{\annotateR}[2]{%
  {\color{red}\underbrace{\color{red}#1}_{\color{red}\mathclap{#2}}}}
\newcommand{\eqnannotate}[2]{%
  {\color{blue}%
  \underbrace{\color{black}#1}_{\color{blue}\mathclap{#2}}}}

\newcommand{\trialvec}[1][]{{\vec{\psi}_\mathit{trial}^{#1}}}
\newcommand{\trialscalar}[1][]{{\psi_\mathit{trial}^{#1}}}
\newcommand{\basisvec}[1][]{{\vec{\psi}_\mathit{basis}^{#1}}}
\newcommand{\basisscalar}[1][]{{\psi_\mathit{basis}^{#1}}}

\newcommand{\tensor}[1]{\bm{#1}}
\DeclareMathOperator{\Tr}{Tr}

\usefonttheme[onlymath]{serif}

% minted shortcuts
\newminted{cfg}{bgcolor=ltslate,autogobble,fontsize=\tiny}
\newminted{bash}{bgcolor=ltltgreen,autogobble,fontsize=\tiny}

% PyLith components
\newcommand{\pylith}[1]{{\ttfamily\color{magenta}#1}}



% --------------------------------------------------------- DOCUMENT
\begin{document}

% ------------------------------------------------------------ SLIDE
\maketitle

\logo{\includegraphics[height=4.5ex]{../../logos/cig_blackfg}}

% ========================================================== SECTION
\section{Introduction}
\setbeamertemplate{footline}{}
% ------------------------------------------------------------ SLIDE
\begin{frame}
  \frametitle{What is Poroelasticity?}
  \summary{Poroelasticity is the study of the interaction between fluid flow and solid deformation in a porous
    medium.}
    \begin{minipage}{0.60\textwidth}
    \includefigure[height=0.9\textheight]{figs/chang-poro}
    \end{minipage}
    \begin{minipage}{0.38\textwidth}
            \begin{enumerate} %[label=(\alph*)]
                \item Sand
                \item Sandstone
                \item Volcanic rock
                \item Fractured rock
                \item Pervious concrete
                \item Polyurethane doam
                \item Metal foam
                \item Bone
                \item Articular cartilage
                \item Nanoporous alumnina
            \end{enumerate}

    \end{minipage}
\end{frame}

% ------------------------------------------------------------ SLIDE
\begin{frame}
  \frametitle{Biot Formulation}
  \begin{enumerate}
   \item Conservation of Momentum
    \begin{equation*}
      \rho_{b} \ddot{\vec{u}} = \vec{f}\left(t\right) + \nabla \cdot \tensor{\sigma}\left(\vec{u},p\right)
    \end{equation*}
        {\tiny
        \begin{itemize}
          \item  Drawn directly from linear elasticity, with stress tensor modified to account for fluid pressure: $\tensor{\sigma} = \tensor{C}:\tensor{\epsilon} - \alpha \tensor{I} p$
          \item Bulk density defined as $\rho_{b} = \left(1 - \phi\right) \rho_{s} + \phi \rho_{f}$
        \end{itemize}}
    \item Conservation of Mass
        \begin{equation*}
          \dot{\zeta} + \nabla \cdot \vec{q}\left(p\right) = \gamma\left(\vec{x},t\right)
        \end{equation*}
          {\tiny
          \begin{itemize}
            \item Specific discharge is defined by Darcy's Law: $\vec{q} = -\frac{\tensor{k}}{\mu_{f}} \cdot \left(\nabla p - \vec{f}_{f} \right)$.
            \item Variation of fluid content is defined as $\zeta = \alpha \epsilon_{v} + \frac{p}{M}$
            \item Biot Modulus is defined as: $M = \frac{K_{f}}{\phi} + \frac{K_{s}}{\alpha - \phi}$
          \end{itemize}}
  \end{enumerate}
\end{frame}

% ------------------------------------------------------------ SLIDE
\begin{frame}
  \frametitle{Governing Assumptions}
  \summary{for Quasistatic, Isotropic Linear Poroelasticity}
  \begin{itemize}
    \item Infinitesimal strain formulation
    \item Undrained condition
    \item Linear elastic solid matrix
    \item Slightly compressible fluid
    \item Inertial term is negligible ($\rho_{b}\ddot{\vec{u}}$ = 0)
    \item Solid phase mass is constant, fluid phase mass is conserved.
  \end{itemize}
\end{frame}

% ------------------------------------------------------------ SLIDE
\begin{frame}
  \frametitle{Quasistatic Formulation}
  \summary{Implicit Time Stepping}
  \begin{itemize}
    \item   We want to solve equations in which the weak form can be expressed
  as:
    \begin{gather*}
      \lhs{F(t,s,\dot{s})} = \rhs{G(t,s)} \\
      \lhs{F(t,s,\dot{s})} = \vec{0} \\
      \lhs{F(t,s,\dot{s})} = \cancelto{\vec{0}}{\rhs{G(t,s)}}
    \end{gather*} 
    Thus, all terms are shifted to the Left Hand Side.
    \item We include a volumetric strain relation,
    \begin{equation*}
      \nabla \cdot \vec{u} = \epsilon_{v}
    \end{equation*}
    and the volumetric strain term, $\epsilon_{v}$, to the solution vector to 
    aid with stability close to incompressibility.
  \end{itemize}
\end{frame}

% ------------------------------------------------------------ SLIDE
\begin{frame}
  \frametitle{Rheology Concept}
  \summary{Elasticity and Rheologies}
  
    \begin{table}[htbp]
      \resizebox{\textwidth}{!}{\begin{tabular}{lll}
        \toprule
        {\bf Material} & {\bf Bulk Rheology} & {\bf Description} \\
        \midrule
        Elasticity & IsotropicLinearElasticity & Isotropic, linear elasticity \\
                   & IsotropicLinearMaxwell & Isotropic, linear Maxwell viscoelasticity \\
                   & IsotropicLinearGenMaxwell & Isotropic, generalized Maxwell viscoelasticity \\
                   & IsotropicPowerLaw & Isotropic, power-law viscoelasticity \\
                   & IsotropicDruckerPrager & Isotropic, Drucker-Prager elastoplasticity \\
        \bottomrule
    \end{tabular}}
    \end{table}
    Poroelasticity has ONE rheology (at the moment).
\end{frame}

% ------------------------------------------------------------ SLIDE
\begin{frame}
  \frametitle{Auxiliary Fields}
  \summary{for Quasistatic Linear Isotropic Poroelasticity}
  
      \begin{table}[htbp]
      \resizebox{\textheight}{!}{\begin{tabular}{lllll}
        \toprule
        {\bf Origin} &{\bf Variable} & {\bf Description} & {\bf Position} & {\bf Notes} \\
        \midrule
        Material & $\rho_b$ & Rock Density    & $0$ &\\
                 & $\rho_f$ & Fluid Density   & $1$ &\\
                 & $\mu_f$  & Fluid Viscosity & $2$ &\\
                 & $\phi$   & Porosity        & $3$ &\\
                 & $\vec{f}_{b}$ & Body Force  & $+1$ &\\
                 & $\vec{g}$ & Gravity        & $+1$ &\\
                 & $\gamma$ & Fluid Source  & $+1$ &\\
        \midrule
        Rheology & $\tensor{\sigma}_{R}$ & Reference Stress & NumAux - 7 &\\
                 & $\tensor{\epsilon}_{R}$ & Reference Strain & NumAux - 6 &\\
                 & $G$ & Shear Modulus & NumAux - 5 &\\
                 & $K_d$ & Drained Bulk Modulus & NumAux - 4 &\\
                 & $\alpha$ & Biot Coefficient & NumAux - 3 &\\
                 & $M$ & Biot Modulus & NumAux - 2 & $\frac{K_{f}}{\phi} + \frac{K_{s}}{\alpha - \phi}$ \\
                 & $\tensor{k}$ & Permeability & NumAux - 1 & \\
        \midrule
         Input   & $K_s$ & Solid Grain Bulk Modulus & - & \\
                         & $K_f$ & Fluid Bulk Modulus & - & \\            
        \bottomrule
    \end{tabular}}
    \end{table}
\end{frame}


% ------------------------------------------------------------ SLIDE
\begin{frame}
  \frametitle{Residuals}
  \summary{for Quasistatic, Isotropic Linear Poroelasticity}
  {\small
  \begin{align*}
    % Displacement
  F^u(t,s,\dot{s}) &= \int_\Omega \trialvec[u] \cdot \eqnannotatemat{\vec{f}(\vec{x},t)}{\vec{f}^u_0}
                     + \nabla \trialvec[u] : \eqnannotaterhe{-\tensor{\sigma}(\vec{u},p_f)}{\tensor{f}^u_1} \, d\Omega
                     + \int_{\Gamma_\tau} \trialvec[u] \cdot \eqnannotate{\vec{\tau}(\vec{x},t)}{\vec{f}^u_0} \, d\Gamma, \\
% Pressure
  F^p(t,s,\dot{s}) &= \int_\Omega  \trialscalar[p] \eqnannotaterhe{\left[\frac{\partial \zeta(\vec{u},p_f)}{\partial t} - \gamma(\vec{x},t) \right]}{f^p_0}
                     + \nabla \trialscalar[p] \cdot \eqnannotaterhe{-\vec{q}(p_f)}{\vec{f}^p_1} \, d\Omega
                     + \int_{\Gamma_q} \trialscalar[p] (\eqnannotate{q_0(\vec{x},t)}{f^p_0}) \, d\Gamma, \\
 % Volumetric Strain
  F^{\epsilon_{v}}(t,s,\dot{s}) &= \int_{\Omega} \trialscalar[\epsilon_{v}] \cdot \eqnannotatemat{\left(\nabla \cdot \vec{u} - \epsilon_{v} \right)}{f^{\epsilon_{v}}_{0}} \, d\Omega. 
\end{align*}}
Brown refers to Material, Green to Rheology

\end{frame}

% ------------------------------------------------------------ SLIDE
\begin{frame}
  \frametitle{Jacobians}
  \summary{for Quasistatic, Isotropic Linear Poroelasticity}
  \centering
\resizebox{1.3\textheight}{!}{
  \begin{minipage}{\linewidth}
  \begin{align*}
%
% JF_UU
% Jf3uu
J_F^{uu} &= \frac{\partial F^u}{\partial u} + t_{shift} \frac{\partial F^u}{\partial \dot{u}} = \int_{\Omega} \trialscalar[u]_{i,k}
\eqnannotaterhe{\left(-C_{ikjl}\right)}{J_{f3}^{uu}} \basisscalar[u]_{j,l} \ d\Omega \\
%
% JF_UP
% Jf2up
J_F^{up} &= \frac{\partial F^u}{\partial p} + t_{shift} \frac{\partial F^u}{\partial \dot{p}} = 
\int_{\Omega} \trialscalar[u]_{i,j} \eqnannotatemat{\left(\alpha \delta_{ij}\right)}{J_{f2}^{up}} \basisscalar[p] \ d\Omega \\
%
% JF_UE
% Jf2ue
J_F^{u \epsilon_{v}} &= \frac{\partial F^u}{\partial \epsilon_{v}} + t_{shift} \frac{\partial F^u}{\partial \dot{\epsilon_{v}}} = \int_{\Omega} \nabla \trialvec[u] : \frac{\partial}{\partial \epsilon_{v}} \
\left[-\left(2 \mu \tensor{\epsilon} + \lambda \tensor{I} \epsilon_{v} - \alpha \tensor{I} p \right) \right] d\Omega =
\int_{\Omega} \trialscalar[u]_{i,j} \eqnannotaterhe{\left(-\lambda \delta_{ij} \right)}{J_{f2}^{u \epsilon_{v}}} \basisscalar[\epsilon_{v}] d\Omega  \\
%
% JF_PU
%
%J_F^{pu} &= \frac{\partial F^p}{\partial u} + t_{shift} \frac{\partial F^p}{\partial \dot{u}} = 0 \\
%
% JF_PP
% Jf0pp
J_F^{pp} &= \frac{\partial F^p}{\partial p} + t_{shift} \frac{\partial F^p}{\partial \dot{p}} = \int_{\Omega} \psi_{trial,k}^p \eqnannotaterhe{\left(-\frac{\tensor{k}}{\mu_{f}} \delta_{kl}\right)}{J_{f3}^{pp}} \psi_{basis,l}^p \ d\Omega +
\int_{\Omega} \trialscalar[p] \eqnannotaterhe{\left(t_{shift} \frac{1}{M}\right)}{J_{f0}^{pp}} \basisscalar[p] \ d\Omega \\
%
% JF_PE
% Jf0pe
J_F^{p\epsilon_{v}} &= \frac{\partial F^p}{\partial \epsilon_{v}} + t_{shift} \frac{\partial
F^p}{\partial \dot{\epsilon_{v}}} = \int_{\Omega} \trialscalar[p] \eqnannotaterhe{\left(t_{shift} \alpha \right)}{J_{f0}^{p\epsilon_{v}}}
\basisscalar[\epsilon_{v}] \ d\Omega \\
%
% JF_EU
% Jf1eu
J_F^{\epsilon_{v}u} &= \frac{\partial F^{\epsilon_{v}}}{\partial u} + t_{shift} \frac{\partial F^{\epsilon_{v}}}{\partial \dot{u}} =
\int_{\Omega} \psi_{trial}^{\epsilon_{v}} \nabla \cdot \vec{\psi}_{basis}^u \ d\Omega = \int_{\Omega}
\basisscalar[\epsilon_{v}] \eqnannotatemat{\left(\delta_{ij}\right)}{J_{f1}^{\epsilon_{v}u}}
\basisscalar[u]_{i,j} \ d\Omega\\
%
% JF_EP
%
%J_F^{\epsilon_{v}p} &= \frac{\partial F^{\epsilon_{v}}}{\partial p} + t_{shift} \frac{\partial F^{\epsilon_{v}}}{\partial \dot{p}} = 0 \\
%
% JF_EE
%
J_F^{\epsilon_{v}\epsilon_{v}} &= \frac{\partial F^\epsilon_{v}}{\epsilon_{v}} + t_{shift} \frac{\partial F^{\epsilon_{v}}}{\partial \dot{\epsilon_{v}}} =
\int_{\Omega} \basisscalar[\epsilon_{v}] \eqnannotatemat{\left(-1\right)}{J_{f0}^{\epsilon_{v}\epsilon_{v}}} \basisscalar[\epsilon_{v}] \ d\Omega
\end{align*}
  \end{minipage}
}
\end{frame}

% ------------------------------------------------------------ SLIDE
\begin{frame}
  \frametitle{Parameters}
  \summary{for Quasistatic Isotropic Linear Poroelasticity}
  \begin{table}[ht]
  \resizebox{1.1\textheight}{!}{\begin{tabular}{lcp{3.5in}}
    \hline
    {\bf Category} & {\bf Symbol} & {\bf Description} \\
    \hline
    Unknowns           & $\vec{u}$ & Displacement field \\
                       & $p$       & Pressure field (corresponds to pore fluid pressure) \\
                       & $\epsilon_{v}$ & Volumetric (trace) strain \\
    \hline
    Derived quantities & $\tensor{\sigma}$ & Cauchy stress tensor \\
                       & $\tensor{\epsilon}$ & Cauchy strain tensor \\
                       & $\zeta$ & Variation of fluid content (variation of fluid vol. per unit vol. of PM), $\alpha \epsilon_{v} + \frac{p}{M}$ \\
                       & $\rho_{b}$ & Bulk density, $\left(1 - \phi\right) \rho_{s} + \phi \rho_{f}$ \\
                       & $\vec{q}$ & Darcy flux, $-\frac{\tensor{k}}{\mu_{f}} \cdot \left(\nabla p - \vec{f}_{f}\right)$ \\
                       & $M$ & Biot Modulus, $\frac{K_{f}}{\phi} + \frac{K_{s}}{\alpha - \phi}$ \\
    \hline
    Common constitutive parameters & $\rho_{f}$ & Fluid density \\
                       & $\rho_{s}$ & Solid (matrix) density \\
                       & $\phi$ & Porosity \\
                       & $\tensor{k}$ & Permeability \\
                       & $\mu_{f}$ & Fluid viscosity \\
                       & $K_{s}$ & Solid grain bulk modulus \\
                       & $K_{f}$ & Fluid bulk modulus \\
                       & $K_{d}$ & Drained bulk modulus \\
                       & $\alpha$ & Biot coefficient, $1 - \frac{K_{d}}{K_{s}}$ \\
    \hline
    Source terms       & $\vec{f}$ & Body force per unit volume, for example: $\rho_{b} \vec{g}$ \\
                       & $\vec{f}_{f}$ & Fluid body force, for example: $\rho_{f} \vec{g}$ \\
                       & $\gamma$ & Source density; rate of injected fluid per unit volume of the porous solid \\
    \hline
  \end{tabular}}
\end{table}
\end{frame}

% ------------------------------------------------------------ SLIDE
\begin{frame}
  \frametitle{Terzaghi's Problem Test Results}
  \summary{}

		\begin{minipage}{0.49\textwidth}
			\includefigure[width=\textwidth]{figs/terzaghi_pressure.png}
			%\caption{Mandel's Problem test result for pressure.}
		\end{minipage}
		\begin{minipage}{0.49\textwidth}
			\includefigure[width=\textwidth]{figs/terzaghi_displacement.png}
			%\caption{Mandel's Problem test result for displacement in the x direction}
		\end{minipage}
		%\caption{Examples of Horizontal Subfigures}

	\begin{figure}
	    \centering
	    \includefigure[height=0.20\textwidth]{./figs/terzaghi_domain.png}	    
	\end{figure}
  
\end{frame}

% ------------------------------------------------------------ SLIDE
\begin{frame}
  \frametitle{Mandel's Problem Test Results}
  \summary{}

		\begin{minipage}{0.49\textwidth}
			\includefigure[width=\textwidth]{figs/mandel_pressure_z_0.png}
			%\caption{Mandel's Problem test result for pressure.}
		\end{minipage}
		\begin{minipage}{0.49\textwidth}
			\includefigure[width=\textwidth]{figs/mandel_displacement_x.png}
			%\caption{Mandel's Problem test result for displacement in the x direction}
		\end{minipage}
		%\caption{Examples of Horizontal Subfigures}

	\begin{figure}
	    \centering
	    \includefigure[height=0.20\textwidth]{./figs/Mandels-Problem-Quarter-Domain.png}	    
	\end{figure}
\end{frame}

% ------------------------------------------------------------ SLIDE
\begin{frame}
  \frametitle{Next Steps}
  \summary{Ordered by Difficulty Level}
  
  \begin{itemize}
  \item Straightforward
    \begin{itemize}
        \item Clean and test dynamic poroelasticity for functionality
        \item Update functions for auxiliary variables
        \item Conversion scripts for poroelastic benchmark cases (e.g. SPE10)
    \end{itemize}
  \item Medium
    \begin{itemize}
      \item Analytical tests for dynamic poroelasticity
      \item Well model combined with point source
    \end{itemize}
  \item More difficult  
    \begin{itemize}
    \item Multiphase model
    \item Fully poroelastic faults
    \end{itemize}
  \end{itemize}


\end{frame}



% ------------------------------------------------------------ SLIDE
\begin{frame}
  \frametitle{Dynamic Poroelasticity}
  \summary{Explicit Time Stepping}
  
  \begin{itemize}
    \item Explicit time stepping with the PETSc TS requires $\lhs{F(t,s,\dot{s})} = \dot{s}$.
    \item Normally $\lhs{F(t,s,\dot{s})}$ contains the inertial term ($\rho \ddot{u}$).
    \item  Therefore, when using explicit time stepping we transform our equation into the form:
      \begin{align*}
        \lhs{F^*(t,s,\dot{s})} = \lhs{\dot{s}} &= \rhs{G^*(t,s)} \\
        \lhs{\dot{s}} &= \lhs{M^{-1}}\rhs{G(t,s)}.
      \end{align*} 
    \item Terms must be rewritten to ensure that the LHS consists only of time derivatives and coefficients.
    \item Velocity is introduced as an unknown, again resulting in a three field solution
  \end{itemize}
\end{frame}

% -------------------------------------------------------------- SLIDE
\begin{frame}
  \frametitle{Dynamic Poroelasticity}
  \summary{Strong Formulation}
{\small
  For compatibility with PETSc TS algorithms, we want to turn the second
order equation elasticity equation into two first order equations. We
introduce the velocity as a unknown,
$\vec{v}=\frac{\partial u}{\partial t}$, which leads to a slightly
different three field problem,
\begin{gather*}
% Solution
\vec{s}^{T} = \left(\vec{u} \quad p \quad \vec{v}\right) \\
% Displacement
\frac{\partial \vec{u}}{\partial t} = \vec{v} \text{ in } \Omega \\
% Pressure
\frac{1}{M}\frac{\partial p}{\partial t }  =  \gamma(\vec{x},t) - \alpha \left(\nabla \cdot \dot{\vec{u}} \right) - \nabla \cdot \vec{q}\left(p\right) = 0 \text{ in } \Omega \\
% Velocity
\rho_{b} \frac{\partial \vec{v}}{\partial t} = \vec{f}(\vec{x},t) + \nabla \cdot \tensor{\sigma}(\vec{u},p) \text{ in } \Omega \\
% Neumann traction
\tensor{\sigma} \cdot \vec{n} = \vec{\tau}(\vec{x},t) \text{ on } \Gamma_{\tau} \\
% Dirichlet displacement
\vec{u} = \vec{u}_{0}(\vec{x}, t) \text{ on } \Gamma_{u} \\
% Neumann flow
\vec{q} \cdot \vec{n} = q_{0}(\vec{x}, t) \text{ on } \Gamma_{q} \\
% Dirichlet pressure
p = p_{0}(\vec{x},t) \text{ on } \Gamma_{p}
\end{gather*}}
  
\end{frame}



% -------------------------------------------------------------- SLIDE
\begin{frame}
  \frametitle{Dynamic Poroelasticity}
  \summary{Weak Formulation}
Using trial functions $\trialvec[u]$, $\trialscalar[p]$, and $\trialvec[v]$, and incorporating the
Neumann boundary conditions, the weak form may be written as:
{\small
\begin{align*}
    % Displacement
    \int_{\Omega} \trialvec[u] \cdot \left( \frac{\partial \vec{u}}{\partial t} \right)d \Omega &= \int_{\Omega} \trialvec[u] \cdot \left( \vec{v} \right) d \Omega \\
    % Pressure
    \int_{\Omega} \trialscalar[p] \left( \frac{1}{M}\frac{\partial p}{\partial t} \right) d\Omega &=
    \int_{\Omega} \trialscalar[p] \left[\gamma(\vec{x},t) - \alpha \left(\nabla \cdot \dot{\vec{u}}\right) \right]  + \nabla \trialscalar[p] \cdot \vec{q}(p) \ d\Omega +
    \int_{\Gamma_q} \trialscalar[p] (-q_0(\vec{x},t)) \ d\Gamma, \\
   % Velocity
   \int_\Omega \trialvec[v] \cdot \left( \rho_{b} \frac{\partial
   \vec{v}}{\partial t} \right) \,
   d\Omega &= \int_\Omega \trialvec[v] \cdot \vec{f}(\vec{x},t) + \nabla \trialvec[v] :
   -\tensor{\sigma} (\vec{u},p_f) \, d\Omega + \int_{\Gamma_\tau} \trialvec[u]
   \cdot \vec{\tau}(\vec{x},t) \, d\Gamma.
\end{align*}}
  
\end{frame}


% ------------------------------------------------------------ SLIDE
\begin{frame}
  \frametitle{Residual Functions}
  \summary{for Dynamic Isotropic Linear Poroelasticity}
  {\small
  \begin{align*}
 % Displacement
 G^u(t,s) &= \int_\Omega \trialvec[u] \cdot \left(\eqnannotate{\vec{v}}{\vec{g}^u_0}\right) \, d\Omega \\
 % Pressure
  G^p(t,s) &= \int_\Omega  \trialscalar[p] \left[\eqnannotate{\gamma \left(\vec{x},t \right) - \alpha \left(\nabla \cdot \dot{\vec{u}} \right)}{g^p_0} \right]
                     + \nabla \trialscalar[p] \cdot \eqnannotate{\vec{q}\left(p\right)}{\vec{g}^p_1} \, d\Omega
                     + \int_{\Gamma_q} \trialscalar[p] \left(\eqnannotate{q_0(\vec{x},t)}{g^p_0}\right) \, d\Gamma, \\
 % Velocity
  G^v(t,s) &= \int_\Omega \trialvec[v] \cdot \eqnannotate{\vec{f}\left(\vec{x},t\right)}{\vec{g}^v_0}
                     + \nabla \trialvec[v] : \eqnannotate{-\tensor{\sigma}\left(\vec{u},p\right)}{\tensor{g}^v_1} \, d\Omega
                     + \int_{\Gamma_\tau} \trialvec[v] \cdot \eqnannotate{\vec{\tau}\left(\vec{x},t\right)}{\vec{g}^v_0} \, d\Gamma,
\end{align*}}
\end{frame}

% ------------------------------------------------------------ SLIDE
\begin{frame}
  \frametitle{Jacobian Functions}
   \summary{for Dynamic Isotropic Linear Poroelasticity}
{\small
These are the pointwise functions associated with $M_{u}$, $M_{p}$, and $M_{v}$ for computing the lumped LHS Jacobian. We premultiply the
RHS residual function by the inverse of the lumped LHS Jacobian while s tshift remains on the LHS with $\dot{s}$ . As a result, we use
LHS Jacobian pointwise functions, but set s tshift = 1 . The LHS Jacobians are:}

\begin{align*}
% Displacement
M_{u} &= J_F^{uu} = \frac{\partial F^u}{\partial u} + s_{tshift} \frac{\partial F^u}{\partial \dot{u}} =
\int_{\Omega} \trialscalar[u]_{i} \eqnannotate{s_{tshift} \delta_{ij}}{J^{uu}_{f0}} \basisscalar[u]_{j} \, d \Omega \\
% Pressure
M_{p} &= J_F^{pp} = \frac{\partial F^p}{\partial p} + t_{shift} \frac{\partial F^p}{\partial \dot{p}} =
\int_{\Omega} \trialscalar[p] \eqnannotate{\left(s_{tshift} \frac{1}{M}\right)}{J_{f0}^{pp}} \basisscalar[p] \ d\Omega \\
% Velocity
M_{v} &= J_F^{vv} = \frac{\partial F^v}{\partial v} + t_{shift} \frac{\partial F^v}{\partial \dot{v}} =
\int_{\Omega} \trialscalar[v]_{i}\eqnannotate{\rho_{b}(\vec{x}) s_{tshift} \delta_{ij}}{J^{vv}_{f0}} \basisscalar[v]_{j} \  d \Omega
\end{align*}   
   
\end{frame}

% --------------------------------------------------------- DOCUMENT
\end{document}

% End of file
