\documentclass[aspectratio=169]{beamer}
\usepackage[none]{hyphenat}
\usepackage{amsmath}
\usepackage{booktabs}
\usepackage{listings}
\usepackage{bm}
\usepackage{mathtools} % equations
\usepackage{tikz}
\usetikzlibrary{shapes,calc}

\title{PyLith User Forum}
\author{Brad Aagaard}
\date{March 16, 2021}

\institute{\includegraphics[scale=0.4]{../logos/cig_blackfg}}

\newcommand{\highlight}[1]{{\bf\color{green}#1}}

\usetikzlibrary{arrows,shapes}
\newcommand{\tikzmark}[1]{\tikz[remember picture,overlay,baseline=-0.5ex] \coordinate (#1);}
\tikzstyle{popup-figure} = [very thick,
  rectangle callout, callout absolute pointer={#1}, anchor=center,
  font=\mdseries, draw=purple,
  fill=white, fill opacity=0.9, text opacity=1.0, text=black]


\usefonttheme[onlymath]{serif}
\newcommand{\lhs}[1]{{\color{blue}#1}}
\newcommand{\rhs}[1]{{\color{orange}#1}}
\newcommand{\annotateL}[2]{%
  {\color{blue}\underbrace{\color{blue}#1}_{\color{blue}\mathclap{#2}}}}
\newcommand{\annotateR}[2]{%
  {\color{orange}\underbrace{\color{orange}#1}_{\color{orange}\mathclap{#2}}}}
\newcommand{\trialvec}[1][]{{\vec{\psi}_\mathit{trial}^{#1}}}
\newcommand{\trialscalar}[1][]{{\psi_\mathit{trial}^{#1}}}
\newcommand{\basisvec}[1][]{{\vec{\psi}_\mathit{basis}^{#1}}}
\newcommand{\basisscalar}[1][]{{\psi_\mathit{basis}^{#1}}}

\newcommand{\tensor}[1]{\bm{#1}}


% ---------------------------------------------------- CUSTOMIZATION
\usetheme{Logo}
\usecolortheme{cig}

\logo{\includegraphics[height=4.5ex]{../logos/cig_blackfg}}
% --------------------------------------------------------- DOCUMENT
\begin{document}

% ------------------------------------------------------------ SLIDE
\maketitle

% ------------------------------------------------------------ SLIDE
\begin{frame}
  \frametitle{PyLith v3}
  \summary{Major rewrite of PyLith started in 2016}

  \small{\begin{tabular}{p{2.0cm}p{5.5cm}p{5.5cm}}
    \toprule
    Feature
    & PyLith v2
    & PyLith v3 \\
    \midrule
    Governing eqn
    & Hardwired (elasticity)
    & Flexible (elasticity, incompressible elasticity, poroelasticity)
    \\[\baselineskip]
    Temporal discretization
    & Backward Euler, Newmark (central difference)
    & PETSc TS (primarily Runge Kutta)
      \\[2\baselineskip]
    Spatial discretization
    & Hardwired (1st order)
    & Flexible (tested up to 4th order)
           \\[2\baselineskip]
           Finite-element definition
           & PyLith
           & PETSc \\
    \bottomrule
  \end{tabular}}
    
\end{frame}


% ------------------------------------------------------------ SLIDE
\begin{frame}[fragile]
  \frametitle{Multiphysics Formulation}
  \summary{Implement physics using point-wise integration functions}  

  We want to solve equations in which the weak \tikzmark{aside-weak-form}form can be expressed
  as
  \begin{gather}
    \lhs{F(t,s,\dot{s})} = \rhs{G(t,s)} \\
    s(t_0) = s_0
  \end{gather}
  where $\lhs{F}$ and $\rhs{G}$ are vector functions, $t$ is time, and $s$ is the solution vector.\pause

  \vspace*{\baselineskip}

  Using the finite-element method and divergence theorem, we cast the weak form into
  \begin{multline}
    \int_\Omega \trialvec \cdot \lhs{\vec{f}_0(t,s,\dot{s})} + \nabla \trialvec : \lhs{\tensor{f}
    _1(t,s,\dot{s})} \, 
    d\Omega = \\
    \int_\Omega \trialvec \cdot \rhs{\vec{g}_0(t,s)} + \nabla \trialvec : \rhs{\tensor{g}_1(t,s)} \, 
    d\Omega,
  \end{multline}
  where $\vec{f}_0$ and $\vec{g}_0$ are vectors, and $\tensor{f}_1$ and
  $\tensor{g}_1$ are tensors.\pause


  \begin{tikzpicture}[remember picture, overlay]
    \node<3>[popup-figure={(aside-weak-form)}] at (7.0,3.0) {
      \begin{minipage}{0.75\textwidth}
      \highlight{Aside: Finite-Element Method Weak Form}
  
      \vspace*{\baselineskip}

      Solve governing equation in integrated sense:
      \begin{equation}
        \int_\Omega \trialscalar \cdot \mathit{PDE} \, d\Omega = 0,
      \end{equation}
      by minimizing the error with respect to the unknown coefficients.

      \vspace*{\baselineskip}

      This usually leads to equations of the form:
      \begin{equation}
        \int_\Omega \trialscalar \cdot f_0(x,t) + \nabla \trialscalar \cdot f_1(x,t) \, d\Omega = 0.
      \end{equation}
    \end{minipage}};\end{tikzpicture}\pause
  
\end{frame}


% ------------------------------------------------------------ SLIDE
\begin{frame}
  \frametitle{Multiphysics Formulation: Dynamic Elasticity without Faults}
  \summary{}

  \only<1>{\highlight{Strong form}
  \begin{align}
      % Solution
      \vec{s}^T &= \left( \vec{u} \quad \ \vec{v} \right)^T, \\
      % Elasticity
    \lhs{\rho \frac{\partial^2\vec{u}}{\partial t^2}} &= \rhs{\vec{f}(\vec{x},t) + \tensor{\nabla} \cdot \tensor{\sigma}(\vec{u})} \text{ in } \Omega, \\
    % Neumann
    \tensor{\sigma} \cdot \vec{n} &= \vec{\tau}(\vec{x},t) \text{ on } \Gamma_\tau, \\
    % Dirichlet
    \vec{u} &= \vec{u}_0(\vec{x},t) \text{ on } \Gamma_u.
  \end{align}}

  \only<2>{\highlight{Weak form}
  
\begin{gather}
  % Displacement-velocity
  \annotateL{\frac{\partial \vec{u}}{\partial t}}{f^u_0}  = M_u^{-1} \int_\Omega \trialvec[u] \cdot \annotateR{\vec{v}}{g^u_0} \, d\Omega, \\
  % Elasticity
  \annotateL{\frac{\partial \vec{v}}{\partial t}}{f^v_0} = M_v^{-1} \int_\Omega \trialvec[v] \cdot \annotateR{\vec{f}(\vec{x},t)}{\vec{g}^v_0} + \nabla \trialvec[v] : \annotateR{-\tensor{\sigma}(\vec{u})}{\tensor{g^v_1}} \, d\Omega
  + \int_{\Gamma_\tau} \trialvec[v] \cdot \annotateR{\vec{\tau}(\vec{x},t)}{\vec{g}^v_0} \, d\Gamma, \\
  % Mu
  M_u = \mathit{Lump}\left( \int_\Omega \trialscalar[u]_i \annotateL{\delta_{ij}}{J^{uu}_{f0}} \basisscalar[u]_j \, d\Omega \right), \\
  % Mv
  M_v = \mathit{Lump}\left( \int_\Omega \trialscalar[v]_i \annotateL{\rho(\vec{x}) \delta_{ij}}{J^{vv}_{f0}} \basisscalar[v]_j \, d\Omega \right).
\end{gather}}
  
\end{frame}


% ------------------------------------------------------------ SLIDE
\begin{frame}
  \frametitle{Multiphysics Formulation: Incompressible elasticity}
  \summary{Formulation shown for quasistatic case without faults}

  \only<1>{\highlight{Strong form}
\begin{gather}
  % Solution
  \vec{s}^T = \left( \vec{u} \quad \ p \right)^T, \\
  % Elasticity
  \lhs{\vec{f}(t) + \tensor{\nabla} \cdot \left(\tensor{\sigma}^\mathit{dev}(\vec{u}) - p\tensor{I}\right)} = \vec{0} \text{ in }\Omega, \\
  % Pressure
  \lhs{\vec{\nabla} \cdot \vec{u} + \frac{p}{K}} = 0 \text{ in }\Omega, \\
  % Neumann
  \tensor{\sigma} \cdot \vec{n} = \vec{\tau} \text{ on }\Gamma_\tau, \\
  % Dirichlet
  \vec{u} = \vec{u}_0 \text{ on }\Gamma_u, \\
  p = p_0 \text{ on }\Gamma_p.
\end{gather}}


  \only<2>{\highlight{Weak form}

    \begin{gather}
      % Elasticity
  \int_\Omega \trialvec[u] \cdot \annotateL{\vec{f}(t)}{f_0^u} + \nabla \trialvec[u] :
  \annotateL{\left(-\tensor{\sigma}^\mathit{dev}(\vec{u}) + p\tensor{I}\right)}{f_1^u}  \, d\Omega
  + \int_{\Gamma_\tau} \trialvec[u] \cdot \annotateL{\vec{\tau}(t)}{f_0^u} \, d\Gamma = 0, \\
% Pressure
  \int_\Omega \trialscalar[p] \cdot \annotateL{\left(\vec{\nabla} \cdot \vec{u} + 
\frac{p}{K} \right)}{f_0^p} \, d\Omega = 0.
\end{gather}}
    
\end{frame}


% ------------------------------------------------------------ SLIDE
\begin{frame}
  \frametitle{Multiphysics Formulation: Poroelasticity}
  \summary{Formulation shown for quasistatic case without fault}

  \only<1>{\highlight{Strong form}
    \begin{gather}
      % Solution
      \vec{s}^{T} = \left(\vec{u} \quad p \quad \epsilon_v\right), \\
      % Elasticity
      \lhs{\vec{f}(t) + \nabla \cdot \tensor{\sigma}(\vec{u},p)} = \vec{0} \text{ in } \Omega, \\
      % Pressure
      \lhs{\frac{\partial \zeta(\vec{u},p)}{\partial t} - \gamma(\vec{x},t) + \nabla \cdot \vec{q}(p)} = 0 \text{ in } \Omega, \\
      % Vol. Strain
      \lhs{\nabla \cdot \vec{u} - \epsilon_{v}} = 0 \text{ in } \Omega, \\
      % Neumann traction
      \tensor{\sigma} \cdot \vec{n} = \vec{\tau}(\vec{x},t) \text{ on } \Gamma_{\tau}, \\
      % Dirichlet displacement
      \vec{u} = \vec{u}_0(\vec{x}, t) \text{ on } \Gamma_{u}, \\
      % Neumann flow
      \vec{q} \cdot \vec{n} = q_0(\vec{x}, t) \text{ on } \Gamma_{q}, \text{ and } \\
      % Dirichlet pressure
      p = p_0(\vec{x},t) \text{ on } \Gamma_{p}.
\end{gather}}

  \only<2>{\highlight{Weak form}
    \begin{gather}
    % Displacement
      \int_\Omega \trialvec[u] \cdot \annotateL{\vec{f}(\vec{x},t)}{f^u_0}
      + \nabla \trialvec[u] : \annotateL{-\tensor{\sigma}(\vec{u},p_f)}{f^u_1} \, d\Omega
      + \int_{\Gamma_\tau} \trialvec[u] \cdot \annotateL{\vec{\tau}(\vec{x},t)}{f^u_0} \, d\Gamma = 0, \\
% Pressure
      \int_\Omega  \trialscalar[p] \annotateL{\left(\frac{\partial \zeta(\vec{u},p_f)}{\partial t} - \gamma(\vec{x},t)\right)} {f^p_0} \, d\Omega
      + \nabla \trialscalar[p] \cdot \annotateL{-\vec{q}(p_f)}{f^p_1} \, d\Omega
      + \int_{\Gamma_q} \trialscalar[p] \annotateL{q_0(\vec{x},t)}{f^p_0} \, d\Gamma = 0, \\
 % Volumetric Strain
      \int_\Omega \trialscalar[\epsilon_{v}] \cdot \annotateL{\left(\nabla \cdot \vec{u} - \epsilon_{v} \right)}{f^{\epsilon_{v}}_{0}} \, d\Omega = 0.
    \end{gather}}
    
\end{frame}


% ------------------------------------------------------------ SLIDE
\begin{frame}
  \frametitle{Poroelasticity Implementation}
  \summary{}

  \vfill
  More information on the poroelasticity implementation
  
  \vfill
  Robert Walker\\
  University of New York, Buffalo\\
  Postdoc with Matthew Knepley
  
\end{frame}


% ------------------------------------------------------------ SLIDE
\begin{frame}
  \frametitle{Summary of Multiphysics Implementation}
  \summary{}

  \highlight{We decouple the finite-element definition from the weak
    form equation, using pointwise functions that look like the PDE.}

  \vfill
  Each material and boundary condition contribute pointwise functions.

  \vfill    
  \begin{description}
  \item[Flexibility] The cell traversal, handled by PETSc,
    accommodates arbitrary cell shapes. The problem can be posed in
    any spatial dimension with an arbitrary number of physical fields.
  \item[Extensibility] PETSc needs to maintain only a single method,
    easing language transitions (CUDA, OpenCL). A new discretization
    scheme could be enabled in a single place in the code.
  \item[Efficiency] Only a few PETSc routines need to be
    optimized. The application scientist is no longer responsible for
    proper vectorization, tiling, and other traversal optimization.
  \end{description}

\end{frame}


% ------------------------------------------------------------ SLIDE
\begin{frame}
  \frametitle{Improved Testing Infrastructure}
  \summary{}

  \begin{itemize}
  \item \highlight{Multiple levels of testing}
    \begin{itemize}
    \item {\bf Unit tests}: Tests of individual class methods or functions
    \item {\bf Method of Manufactured solutions}: Verify implementation of physics
    \item {\bf Full scale tests} Verify all output for simple simulation (can be parallel)
    \end{itemize}
  \item \highlight{Continuous integration}
    \begin{itemize}
    \item Use Docker images to allow testing on many (8) Linux
      distributions via GitLab pipelines
    \item All dependencies included in custom Docker images
    \item Test runners build PETSc + CIG code in Docker containers and run tests
    \end{itemize}
  \end{itemize}
    
\end{frame}


% ------------------------------------------------------------ SLIDE
\begin{frame}
  \frametitle{Recent Development Activities}
  \summary{}  

  \begin{itemize}
  \item \highlight{Updated Pythia/Pyre to Python 3}
    \begin{itemize}
    \item Added unit tests for features used by PyLith
    \item Compatible with Python 3.6 or later
    \item Reorganized code into single pythia package
    \item Integrated nemesis into pythia (no longer a separate install)
    \end{itemize}\pause
  \item \highlight{Updated SpatialData to Proj 6.x/7.x}
    \begin{itemize}
    \item Updated SpatialData to use current Proj API (v6.0 and later)
    \item Specify coordinate system\tikzmark{cs-example} using Proj parameters, EPSG codes,
      or WKT strings
    \end{itemize}
  \end{itemize}

  \begin{tikzpicture}[remember picture, overlay]
    \node<3>[popup-figure={(cs-example)}] at (7.0,-1.0) {
      \begin{minipage}{0.8\textwidth}
        
      {\bf UTM zone 11 northern hemisphere}

      \footnotesize{\begin{description}
      \item[Proj] +proj=utm +zone=11 +ellps=GRS80 +datum=NAD83 +units=m
      \item[EPSG] EPSG:26911
      \end{description}}
    \end{minipage}};\end{tikzpicture}

  
\end{frame}


% ------------------------------------------------------------ SLIDE
\begin{frame}
  \frametitle{PyLith v3 - Physics Capabilities}
  \summary{Release before June 2021 hackathon}  

  \begin{itemize}
  \item \highlight{Governing Equations}
    \begin{itemize}
    \item Quasistatic and dynamic elasticity
    \item Quasistatic incompressible elasticity
    \item Quasistatic poroelasticity
    \end{itemize}
  \item \highlight{Bulk Rheologies}
    \begin{itemize}
    \item Isotropic, linear elasticity (all governing equations)
    \item Viscoelasticity (quasistatic elasticity)
    \end{itemize}
  \item \highlight{Fault}
    \begin{itemize}
    \item Prescribed slip for quasistatic and dynamic problems
    \end{itemize}
  \item \highlight{Boundary Conditions}
    \begin{itemize}
    \item Dirichlet
    \item Neumann
    \item Absorbing
    \end{itemize}
  \end{itemize}
  
\end{frame}


% ------------------------------------------------------------ SLIDE
\begin{frame}
  \frametitle{PyLith v3 Documentation Improvements}
  \summary{}

  \begin{itemize}
  \item \highlight{{\tt pyre\_doc}: Get help for any Pyre component}\\
    \begin{itemize}
    \item Description of component
    \item List of properties and facilities with default values
    \end{itemize}
  \item \highlight{{\tt pylith\_cfgsearch}: Search {\tt .cfg} files based on keywords and features}
    \begin{itemize}
    \item Find example(s) demonstrating feature(s)
    \item Search your own files based on keyword(s) or feature(s)
    \end{itemize}
  \end{itemize}
  
\end{frame}


% ------------------------------------------------------------ SLIDE
\begin{frame}
  \frametitle{PyLith Features Coming Soon}
  \summary{Features with very high priority for implementation}  

  \begin{itemize}
  \item Fault friction
  \item Import Gmsh (open-source mesh generation) files
  \item Parallel mesh loading
  \item Reimplement Drucker-Prager elastoplastic bulk rheology
  \end{itemize}
  
\end{frame}


% ------------------------------------------------------------ SLIDE
\begin{frame}
  \frametitle{Planned Changes to PyLith Documentation}
  \summary{}

  \begin{itemize}
  \item \highlight{Change distribution: PDF $\rightarrow$ online + PDF}
    \begin{itemize}
    \item Switch documentation source from \LaTeX to Markedly
      Structured Text\\ Markdown processed by Sphinx
    \item Online: Better search, stable development version + releases
    \end{itemize}
  \item \highlight{Examples via Jupyter Notebooks}
    \begin{itemize}
    \item Consolidate each example into a single
      notebook
    \item Available as Jupyter notebook, which is integrated into
      (online) documentation
    \end{itemize}
  \item \highlight{Timeline}
    \begin{itemize}
    \item 1-2 years
    \item Could be faster with help from the PyLith community
    \end{itemize}
  \end{itemize}
  
\end{frame}



% ------------------------------------------------------------ SLIDE
\begin{frame}
  \frametitle{PyLith Hackathon, June 2021}
  \summary{}

  Online Hackathon, June 10-14 (Mon-Thu) and June 17-19 (Mon-Wed)

  \vfill

  \begin{itemize}
  \item 3--4 projects with 3--4 people working on each project as a team
  \item Registration will open later this week
  \item PyLith development environment
    \begin{itemize}
    \item We will provide a Docker container with {\bf all} dependencies
    \item Virtual Studio Code Integrated Development Environment\\
      Run VSCode locally (outside container) but work is done in the container
    \item We will provide instructions for how to setup the development environment
    \end{itemize}
  \item Prerequisites
    \begin{itemize}
    \item Experience using PyLith
    \item Comfortable with using the command line
    \item Some coding experience (depends on the project)
    \item Available to do ``homework'' before the hackathon
    \end{itemize}
  \end{itemize}
  
\end{frame}

% ------------------------------------------------------------ SLIDE
\begin{frame}
  \frametitle{Potential Hackathon Projects}
  \summary{We will work with you to refine the project scope/details {\em before} the hackathon}

  \begin{itemize}
  \item Poroelasticity: examples, benchmarks, fault implementation w/pore pressure
  \item Point sources: wells; seismic moment tensors
  \item Earthquake cycle modeling: start top-level implementation
  \item Gmsh integration: Gmsh Python scripts for existing examples
  \item Full scale tests: Create simulations with known solutions
  \item Documentation: Convert to MyST + Sphinx; examples as Jupyter notebooks
  \end{itemize}
  \vfill
  \begin{itemize}
  \item Adaptive mesh refinement
  \item Viscoelastic bulk rheologies using strain rate
  \item Time-dependent Green's functions
  \end{itemize}

  \vfill
  You may propose your own project too!
  
\end{frame}


% ------------------------------------------------------------ SLIDE
\begin{frame}
  \frametitle{Open Discussion}
  \summary{}

  \vfill
  Questions/discussion related to anything presented

  \vfill
  Questions/discussion on anything PyLith related
  
\end{frame}





% --------------------------------------------------------- DOCUMENT
\end{document}


% End of file
