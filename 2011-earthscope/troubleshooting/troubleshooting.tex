% -*- TeX -*-
%
% ----------------------------------------------------------------------
%
%                           Brad T. Aagaard
%                        U.S. Geological Survey
%
% {LicenseText}
%
% ----------------------------------------------------------------------
%

\documentclass{beamer}
\usepackage{amsmath}
\usepackage{array}
\usepackage{mpmulti}

\title{PyLith Troubleshooting Tips/Tricks}
\subtitle{}
\author{Brad Aagaard \\
  Charles Williams \\
  Matt Knepley}
\institute{}
\date{May 17, 2011}


% ---------------------------------------------------- CUSTOMIZATION
\newcommand{\thispdfpagelabel}[1]{}
\newcommand{\important}[1]{{\bf\color{red}#1}}
\usetheme{CIG}

% ========================================================= DOCUMENT
\begin{document}

% ------------------------------------------------------------ SLIDE
\maketitle

% ------------------------------------------------------------ SLIDE
\logo{\includegraphics[height=4.5ex]{../../logos/cig}}

% ========================================================== SECTION
\section{Troubleshooting}
\subsection{Overview}

% ------------------------------------------------------------ SLIDE
\begin{frame}
  \frametitle{Outline}
  \summary{}

  \begin{itemize}
  \item General numerical modeling tips
  \item Mesh generation
  \item Running PyLith
  \end{itemize}
  
\end{frame}

% ========================================================== SECTION
\subsection{General}

% ------------------------------------------------------------ SLIDE
\begin{frame}
  \frametitle{General Numerical Modeling Tips}
  \summary{Start simple and progressively add complexity and increase
    resolution}
 
  \begin{itemize}
  \item \important{Start in 2-D, if possible, and then go to 3-D}
    \begin{itemize}
    \item Much smaller problems $\Rightarrow$ much faster turnaround
    \item Experiment with meshing, boundary conditions, solvers, etc
    \item Keep in mind how physics differs from 3-D
    \end{itemize}
  \item \important{Start with coarse resolution and then increase resolution}
    \begin{itemize}
    \item Much smaller problems $\Rightarrow$ much faster turnaround
    \item Experiment with meshing, boundary conditions, solvers, etc.
    \item Increase resolution until solution resolves features of interest
      \begin{itemize}
      \item Resolution will depend on spatial scales in BC, initial
        conditions, deformation, and geologic structure
      \item Is geometry of domain important? At what resolution?
      \item Displacement field is integral of strains/stresses
      \item Resolving stresses/strains requires fine resolution simulations
      \end{itemize}
    \end{itemize}
  \item \important{Use your intuition and analogous solutions to check
      your results!}
  \end{itemize}

\end{frame}


% ========================================================== SECTION
\subsection{Meshing}

% ------------------------------------------------------------ SLIDE
\begin{frame}
  \frametitle{Mesh Generation Tips}
  \summary{There is no silver bullet in finite-element mesh generation}
 
  \begin{itemize}
  \item Hex/Quad versus Tet/Tri
    \begin{itemize}
    \item Hex/Quad are slightly more accurate and faster
    \item Tet/Tri easily handle complex geometry
    \item Easy to vary discretization size with Tet, Tri, and Quad cells
    \item There is no easy answer\\
      For a given accuracy, a finer resolution Tet mesh that varies
      the discretization size in a more optimal way {\bf\it might} run
      faster than a Hex mesh
    \end{itemize}
  \item Check and double-check your mesh
    \begin{itemize}
    \item Were there any errors when running the mesher?
    \item Do all of the nodesets and blocks look correct?
    \item Check mesh quality (aspect ratio should be close to 1)
    \end{itemize}
  \item CUBIT
    \begin{itemize}
    \item Name objects and use APREPRO or Python for robust scripts
    \item Number of points in spline curves/surfaces has huge affect
      on mesh generation runtime
    \end{itemize}
  \end{itemize}

\end{frame}


% ========================================================== SECTION
\subsection{PyLith}

% ------------------------------------------------------------ SLIDE
\begin{frame}
  \frametitle{PyLith Tips}
  \summary{}
 
  \begin{itemize}
  \item \important{Read the PyLith User Manual}
  \item \important{Do not ignore error messages and warnings!}
  \item Use an example/benchmark as a starting point
  \item Quasi-static simulations
    \begin{itemize}
    \item Start with a static simulation and then add time dependence
    \item \important{Check that the solution converges at every time step}
    \end{itemize}
  \item Dynamic simulations
    \begin{itemize}
    \item Start with a static simulation
    \item \important{Shortest wavelength seismic waves control cell size}
    \end{itemize}
  \item CIG Short-Term Crustal Dynamics mailing list\\
    {\tt cig-short@geodynamics.org}
  \item Short-Term Crustal Dynamics wiki (under construction)
  \item CIG bug tracking system\\
    {\tt http://www.geodynamics.org/roundup}
  \end{itemize}

\end{frame}


% ------------------------------------------------------------ SLIDE
\begin{frame}
  \frametitle{PyLith Debugging Tools}
  \summary{}

  \begin{itemize}
  \item {\tt pylithinfo [--verbose] [PyLith args]}\\
    Dumps all parameters with their current values to text file
  \item Command line arguments
    \begin{itemize}
    \item {\tt --help}
    \item {\tt --help-components}
    \item {\tt --help-properties}
    \item {\tt --petsc.start\_in\_debugger} (run in xterm)
    \item {\tt --nodes=N} (to run on N processors on local machine)
    \end{itemize}
  \item Journal info flags turn on writing progress/inf\\
    {\tt [pylithapp.journal.info]}\\
    {\tt timedependent = 1} \\
    \begin{itemize}
    \item Turns on/off info for each type of component independently
    \item Examples turn on writing lots of info to stdout using journal flags
    \end{itemize}
  \end{itemize}

\end{frame}




% ======================================================================
\end{document}


% End of file
