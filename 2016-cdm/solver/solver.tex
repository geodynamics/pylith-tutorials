\documentclass[dvipsnames]{beamer}

\mode<presentation>
{
  %%\usetheme[right]{Hannover}
  %%\usetheme{Goettingen}
  %%\usetheme{Dresden}
  \usetheme{Warsaw}
  \useoutertheme{infolines}
  \useinnertheme{rounded}

  \setbeamercovered{transparent}
}

\usepackage[english]{babel}
\usepackage[latin1]{inputenc}
\usepackage{times}
\usepackage[T1]{fontenc} % Note that the encoding and the font should match. If T1 does not look nice, try deleting this line
\usepackage{amsfonts, amsmath, subfigure, multirow}
\usepackage{hyperref}
\newlength\LL \settowidth\LL{1000}

%\usepackage{beamerseminar}
\usepackage{graphicx}
\usepackage{array}
\usepackage{ulem}
\usepackage{listings}
\usepackage{xfrac}
\usepackage{xspace}
\usepackage{colortbl}
\usepackage{alltt}

\usepackage{mathtools}       % For \mathllap
\usepackage[boxed]{algorithm}
\usepackage{algpseudocode}   % For \Procedure on algorithm

\usepackage{tikz}
\usepackage{pgflibraryshapes}
\usetikzlibrary{decorations.text}
\usetikzlibrary{decorations.pathreplacing}
\usetikzlibrary{backgrounds}
\usetikzlibrary{calc}
\usetikzlibrary{arrows}
\usetikzlibrary{snakes}
\usetikzlibrary{positioning}
\definecolor{cffffff}{RGB}{255,255,255}
\definecolor{cff00c0}{RGB}{255,0,192}
\definecolor{c0f00ff}{RGB}{15,0,255}
\definecolor{c00ff20}{RGB}{0,255,32}
\definecolor{cfffd00}{RGB}{255,253,0}
\definecolor{c00e0ff}{RGB}{0,224,255}

\usepackage{pgfplots}
\usepackage{pgfplotstable}

\usepackage[absolute,overlay]{textpos}
\setlength{\TPHorizModule}{\textwidth}
\setlength{\TPVertModule}{\textheight}
\textblockorigin{0pt}{0pt} % start everything near the top-left corner

% Text macros
\makeatletter
\DeclareRobustCommand\onedot{\futurelet\@let@token\@onedot}
\def\@onedot{\ifx\@let@token.\else.\null\fi\xspace}
\def\eg{{e.g}\onedot} \def\Eg{{E.g}\onedot}
\def\ie{{i.e}\onedot} \def\Ie{{I.e}\onedot}
\def\cf{{c.f}\onedot} \def\Cf{{C.f}\onedot}
\def\etc{{etc}\onedot}
\def\vs{{vs}\onedot}
\def\wrt{w.r.t\onedot}
\def\dof{d.o.f\onedot}
\def\etal{{et al}\onedot}
\makeatother

\newenvironment{changemargin}[2]{%
  \begin{list}{}{%
    \setlength{\topsep}{0pt}%
    \setlength{\leftmargin}{#1}%
    \setlength{\rightmargin}{#2}%
    \setlength{\listparindent}{\parindent}%
    \setlength{\itemindent}{\parindent}%
    \setlength{\parsep}{\parskip}%
  }%
  \item[]}{\end{list}}
\newcommand{\D}{\mathcal{D}}
\newcommand{\E}{\mathcal{E}}
\newcommand{\F}{\mathcal{F}}
\newcommand{\I}{\mathcal{I}}
\newcommand{\M}{\mathcal{M}}
\newcommand{\N}{\mathcal{N}}
\newcommand{\bigO}{\mathcal{O}}
\newcommand{\Q}{\mathcal{Q}}
\newcommand{\R}{\mathbb{R}}
\newcommand{\X}{\mathbb{X}}
\newcommand{\kb}{\tt}
\newcommand{\vF}{\Vec{F}}
\newcommand{\vJ}{\Vec{J}}
\newcommand{\vM}{\Vec{M}}
\newcommand{\vN}{\Vec{N}}
\newcommand{\vb}{\Vec{b}}
\newcommand{\vd}{\Vec{d}}
\newcommand{\vf}{\Vec{f}}
\newcommand{\vj}{\Vec{j}}
\newcommand{\vk}{\Vec{k}}
\newcommand{\vn}{\Vec{n}}
\newcommand{\vr}{\Vec{r}}
\newcommand{\vu}{\Vec{u}}
\newcommand{\vv}{\Vec{v}}
\newcommand{\vx}{\Vec{x}}
\newcommand{\vy}{\Vec{y}}
\newcommand{\vz}{\Vec{z}}
\newcommand\oneitem[1]{\begin{itemize} \item #1 \end{itemize}}
\def\shell#1{{\tt \$ #1}}

%   Default fixed font does not support bold face
\DeclareFixedFont{\ttb}{T1}{txtt}{bx}{n}{12} % for bold
\DeclareFixedFont{\ttm}{T1}{txtt}{m}{n}{12}  % for normal

% Color macros
\newcommand{\black}{\textcolor{black}}
\newcommand{\blue}{\textcolor{blue}}
\newcommand{\green}{\textcolor{green}}
\newcommand{\red}{\textcolor{red}}
\newcommand{\brown}{\textcolor{brown}}
\newcommand{\cyan}{\textcolor{cyan}}
\newcommand{\magenta}{\textcolor{magenta}}
\newcommand{\yellow}{\textcolor{yellow}}

\usepackage{color}
\definecolor{deepblue}{rgb}{0,0,0.5}
\definecolor{deepred}{rgb}{0.6,0,0}
\definecolor{deepgreen}{rgb}{0,0.5,0}

% Nonlinear preconditioning
\newcommand{\vecname}[1]{\ensuremath{\mathbf{#1}}}
\newcommand{\restrict}{\vecname{R}}
\newcommand{\interp}{\vecname{P}}
\newcommand{\inject}{\vecname{\widehat{R}}}
\newcommand{\pc}{\ensuremath{-}}
\newcommand{\lp}{\ensuremath{\pc_{L}}}
\newcommand{\rp}{\ensuremath{\pc_{R}}}
\newcommand{\lin}{\ensuremath{\backslash}}
\newcommand{\solvername}[1]{\ensuremath{\begingroup\text{#1}\endgroup}}
\newcommand {\NRICH}{\solvername{NRICH}}
\newcommand {\NGMRES}{\solvername{NGMRES}}
\newcommand {\NEWT}{\mathcal{N}}
\newcommand {\NCG}{\solvername{NCG}}
\newcommand {\NGS}{\solvername{NGS}}
\newcommand {\FAS}{\solvername{FAS}}
\newcommand {\QN}{\solvername{QN}}
\newcommand {\NASM}{\solvername{NASM}}
\newcommand {\RAS}{\solvername{RAS}}
\newcommand {\ASM}{\solvername{ASM}}
\newcommand {\MG}{\solvername{MG}}
\newcommand {\GSN}{\solvername{GSN}}
\newcommand {\GS}{\solvername{GS}}
\newcommand {\GSPIN}{\solvername{GSPIN}}
\newcommand {\SOR}{\solvername{SOR}}
\newcommand {\ASPIN}{\solvername{ASPIN}}
\newcommand {\CG}{\solvername{CG}}
\newcommand {\GMRES}{\solvername{GMRES}}
\newcommand {\krylov}{\solvername{K}}
\newcommand {\FGMRES}{\solvername{FGMRES}}
\newcommand {\ILU}{\solvername{ILU}}
\newcommand {\LU}{\solvername{LU}}
\newcommand {\NK}{\NEWT\lin\krylov}
\newcommand {\AND}{\solvername{Anderson}}
\newcommand {\GB}{\solvername{GB}}

% Optimization macros
\newcommand{\mini}{\mathop{\rm minimize}}
\newcommand{\st}{\mbox{subject to }}

% PDE macros
\newcommand{\lap}[1]{\Delta #1}
\newcommand{\grad}[1]{\nabla #1}
\renewcommand{\div}[1]{\nabla \cdot #1}

% Tutorial code macros
\def\code#1{{\tt #1}}
\def\enum#1{{\sc #1}}
\def\class#1{{\tt\bf #1}}
\def\function#1{\code{#1}}
\def\Update#1{\frametitle{Code Update} \begin{center}\Huge Update to {Revision \green{#1}}\end{center}}

% Python style for highlighting
\newcommand\pythonstyle{\lstset{
language=Python,
basicstyle=\ttm\scriptsize,
otherkeywords={self},             % Add keywords here
keywordstyle=\ttb\scriptsize\color{deepblue},
emph={MyClass,__init__},          % Custom highlighting
emphstyle=\ttb\scriptsize\color{deepred},    % Custom highlighting style
commentstyle=\scriptsize\color{deepgreen},
frame=tb,                         % Any extra options here
showstringspaces=false            % 
}}


% Python environment
\lstnewenvironment{python}[1][]
{
\pythonstyle
\lstset{#1}
}
{}

% Python for external files
\newcommand\pythonexternal[2][]{{
\pythonstyle
\lstinputlisting[#1]{#2}}}

% Python for inline
\newcommand\pythoninline[1]{{\pythonstyle\lstinline!#1!}}

% C style for highlighting
\newcommand\cstyle{\lstset{
language=C,
basicstyle=\ttm\scriptsize,
otherkeywords={MPI_Comm,TS,SNES,KSP,NPC,PC,DM,Mat,Vec,VecScatter,IS,PetscSF,PetscSection,PetscObject,PetscInt,PetscScalar,PetscReal,PetscBool,InsertMode,PetscErrorCode}, % Add keywords here
keywordstyle=\ttb\scriptsize\color{deepblue},
emph={PETSC_COMM_WORLD,PETSC_NULL,SNES_NGMRES_RESTART_PERIODIC},          % Custom highlighting
emphstyle=\ttb\scriptsize\color{deepred},    % Custom highlighting style
commentstyle=\scriptsize\color{brown},
stringstyle=\ttm\scriptsize\color{deepgreen},
frame=tb,                         % Any extra options here
showstringspaces=false            % 
}}

% C environment
\lstnewenvironment{cprog}[1][]
{
\cstyle
\lstset{#1}
}
{}

% C for external files
\newcommand\cexternal[2][]{{
\cstyle
\lstinputlisting[#1]{#2}}}

% C for inline
\newcommand\cinline[1]{{\cstyle\lstinline!#1!}}

% bash style for highlighting
\newcommand\bashstyle{\lstset{
language=bash,
basicstyle=\scriptsize\ttfamily,
}}

% bash environment
\lstnewenvironment{bash}[1][]
{
\bashstyle
\lstset{#1}
}
{}

% bash for inline
\newcommand\bashinline[1]{{\bashstyle\lstinline!#1!}}

% C++ style for highlighting
\newcommand\cppstyle{\lstset{
language=C++,
basicstyle=\ttm\scriptsize,
otherkeywords={MPI_Comm,TS,SNES,KSP,PC,DM,Mat,Vec,VecScatter,IS,PetscSF,PetscSection,PetscObject,PetscInt,PetscScalar,PetscReal,PetscBool,InsertMode,PetscErrorCode}, % Add keywords here
keywordstyle=\ttb\scriptsize\color{deepblue},
emph={PETSC_COMM_WORLD,PETSC_NULL,SNES_NGMRES_RESTART_PERIODIC},          % Custom highlighting
emphstyle=\ttb\scriptsize\color{deepred},    % Custom highlighting style
commentstyle=\scriptsize\color{brown},
stringstyle=\ttm\scriptsize\color{deepgreen},
frame=tb,                         % Any extra options here
showstringspaces=false            % 
}}

% C++ environment
\lstnewenvironment{cpp}[1][]
{
\cppstyle
\lstset{#1}
}
{}

% Fortran style for highlighting
\newcommand\fortranstyle{\lstset{
language=Fortran,
basicstyle=\ttm\scriptsize,
otherkeywords={MPI_Comm,TS,SNES,KSP,PC,DM,Mat,Vec,VecScatter,IS,PetscSF,PetscSection,PetscObject,PetscInt,PetscScalar,PetscReal,PetscBool,InsertMode,PetscErrorCode}, % Add keywords here
keywordstyle=\ttb\scriptsize\color{deepblue},
emph={PETSC_COMM_WORLD,PETSC_NULL,SNES_NGMRES_RESTART_PERIODIC},          % Custom highlighting
emphstyle=\ttb\scriptsize\color{deepred},    % Custom highlighting style
commentstyle=\scriptsize\color{brown},
stringstyle=\ttm\scriptsize\color{deepgreen},
frame=tb,                         % Any extra options here
showstringspaces=false            % 
}}

% Fortran environment
\lstnewenvironment{fortran}[1][]
{
\fortranstyle
\lstset{#1}
}
{}

% Make style for highlighting
\newcommand\makestyle{\lstset{
language=Make,
basicstyle=\ttm\scriptsize,
otherkeywords={MPI_Comm,TS,SNES,KSP,PC,DM,Mat,Vec,VecScatter,IS,PetscSF,PetscSection,PetscObject,PetscInt,PetscScalar,PetscReal,PetscBool,InsertMode,PetscErrorCode}, % Add keywords here
keywordstyle=\ttb\scriptsize\color{deepblue},
emph={PETSC_COMM_WORLD,PETSC_NULL,SNES_NGMRES_RESTART_PERIODIC},          % Custom highlighting
emphstyle=\ttb\scriptsize\color{deepred},    % Custom highlighting style
commentstyle=\scriptsize\color{brown},
stringstyle=\ttm\scriptsize\color{deepgreen},
frame=tb,                         % Any extra options here
showstringspaces=false            % 
}}

% Make environment
\lstnewenvironment{make}[1][]
{
\makestyle
\lstset{#1}
}
{}

\AtBeginSection[]
{
  \begin{frame}<beamer>
    \frametitle{Outline}
    \tableofcontents[currentsection,hideothersubsections]
  \end{frame}
}

\AtBeginSubsection[]
{
  \begin{frame}<beamer>
    \frametitle{Outline}
    \tableofcontents[sectionstyle=show/hide,subsectionstyle=show/shaded/hide]
  \end{frame}
}

\usepackage{listings}

\title[Solvers]{Using PETSc Solvers in PyLith}
\author[M.~Knepley]{Matthew~Knepley, Brad Aagaard, and Charles Williams}
\date[CIG16]{CIG All-Hands PyLith Tutorial 2016\\UC Davis \quad June 19, 2016}
% - Use the \inst command if there are several affiliations
\institute[Rice]{
  Computational and Applied Mathematics\\
  Rice University
}
\subject{PyLith}

\begin{document}

\begin{frame}
  \titlepage
  \begin{center}
  \includegraphics[scale=0.12]{figures/logos/anl-white-background-modern.jpg}\hspace{1.0in}
  \includegraphics[scale=0.10]{figures/logos/RiceLogo_TMCMYK300DPI.jpg}
  \end{center}
  \vskip0.4in
\end{frame}
%
\begin{frame}<testing>{Abstract}\small
The idea here is to show people how to understand waht is going on, rather than just give cookbook
examples since they are already in PyLith.
\end{frame}
%
\begin{frame}{Main Point}

\Huge
We want to enable users to\\
\bigskip
\pause
assess solver performance,\\
\bigskip
\pause
and optimize solvers\\
\qquad for particular problems.
\end{frame}
%
%
\section{Mathematical Background}
\begin{frame}{Measures}\Large

\begin{center}\LARGE What do we care about?\end{center}

\bigskip

\begin{itemize}
  \item<2-> Error\\
    The difference between my computed answer and\\
    the true solution of my equations % not about real life

  \bigskip

  \item<3-> Residual\\
    How close my computed answer comes to satisfying my equations
\end{itemize}
\end{frame}
%
\begin{frame}{Measures}\Large
\begin{center}\LARGE This gives us two metrics for measuring the quality of our computed solutions.\end{center}

\bigskip

\begin{overprint}
\onslide<1>
\begin{center}They are not the same, but they are related.\end{center}
\onslide<2>
\begin{align*}
  \frac{\|u - u^*\|}{\|u^*\|} \le \kappa \frac{\|F(u)\|}{\|F(0)\|}
\end{align*}
\onslide<3>
For linear equations, $F(u) = A u - b$, this becomes
\begin{align*}
  \frac{\|u - u^*\|}{\|u^*\|} &= \frac{\|A^{-1} (Au - b)\|}{\|A^{-1} b\|}\\
                              &\le \frac{\|A\| \|A^{-1}\| \|Au - b\|}{\|b\|}\\
                              &\le \kappa(A) \frac{\|Au - b\|}{\|b\|}
\end{align*}
\onslide<4>
For linear equations, $F(u) = A u - b$, this becomes
\begin{align*}
  \kappa(A) = \|A\| \|A^{-1}\|
\end{align*}
\end{overprint}
\end{frame}
%
%
\section{Controlling the Solver}
%
\begin{frame}[fragile]{Controlling PETSc}
\begin{center}\LARGE
  All of PETSc can be controlled by \blue{options}
\end{center}
\begin{itemize}
  \item[] \verb|-ksp_type gmres|
  \item[] \verb|-start_in_debugger|
  \item[] All objects can have a prefix, \verb|-velocity_pc_type jacobi|
\end{itemize}
\pause
\medskip
\begin{center}\LARGE
  All PETSc options can be given to PyLith
\end{center}
\begin{itemize}
  \item[] \verb|--petsc.ksp_type=gmres|
  \item[] \verb|--petsc.start_in_debugger|
\end{itemize}
\end{frame}
%
\begin{frame}[fragile]{Examples}

{\LARGE We will illustrate options using}
\bigskip
\begin{center}
  PETSc SNES \magenta{\href{http://www.mcs.anl.gov/petsc/petsc-current/src/snes/examples/tutorials/ex19.c.html}{ex19}}, located at \verb|$PETSC_DIR/src/snes/examples/tutorials/ex19.c|
\end{center}
\bigskip
{\LARGE and}
\bigskip
\begin{center}
  PyLith Example \magenta{\href{http://www.geodynamics.org/svn/cig/short/3D/PyLith/trunk/examples/3d/hex8/}{hex8}}, located at \verb|$PYLITH_DIR/examples/3d/hex8/|
\end{center}
\end{frame}
%
\section{Where do I begin?}
%
\begin{frame}{What solvers can I choose from?}\Large
\begin{itemize}
  \item Direct (LU, Cholesky)
  \begin{itemize}
    \item Robust, large memory and time
  \end{itemize}
  \bigskip
  \item Multigrid
  \begin{itemize}
    \item Touchy, small memory and time
  \end{itemize}
  \bigskip
  \item Domain Decomposition
  \begin{itemize}
    \item Somewhat robust, medium memory and time
  \end{itemize}
  \bigskip
  \item Krylov
  \begin{itemize}
    \item Ineffectual alone, small memory and time
  \end{itemize}
\end{itemize}
\end{frame}
%
\begin{frame}{What is a Krylov solver?}\Large
A Krylov solver builds a small model of a linear operator $A$, using a subspace defined by
\begin{equation*}
  \mathcal{K}(A,r) = \mathrm{span}\{r, Ar, A^2r, A^3r, \ldots \}
\end{equation*}
where $r$ is the initial residual.
\bigskip

\pause
The small system is solved directly, and the solution is projected back to the original space.
\end{frame}
%
\begin{frame}{What Should I Know About Krylov Solvers?}
\LARGE
\begin{itemize}
  \item Strength is \textit{adaptivity}
  \begin{itemize}
    \item They can handle \textit{low-mode} errors
  \end{itemize}
  \bigskip
  \item They are \textit{not} stand-alone solvers
  \begin{itemize}
    \item They need preconditioners
  \end{itemize}
  \bigskip
  \item Scalability suffers after many iterates
  \begin{itemize}
    \item They do a lot of inner products
  \end{itemize}
\end{itemize}
\end{frame}
%
\begin{frame}{What is a Preconditioner?}\Large
A preconditioner $M$ changes a linear system,
\begin{equation*}
  M^{-1} A x = M^{-1} b
\end{equation*}
so that the effective operator is $M^{-1} A$, which is hopefully \red{better} for Krylov methods.
\bigskip
\begin{itemize}
  \item<2-> Preconditioner should be inexpensive
  \medskip
  \item<3-> Preconditioner should accelerate convergence
\end{itemize}
\end{frame}
%
\begin{frame}[fragile]{{\bf Always} start with LU}
Always, always start with LU:
\begin{itemize}
  \item No iterative tolerance
  \medskip
  \item (Almost) no condition number dependence
  \medskip
  \item Check for accidental singularity
\end{itemize}
\pause
\bigskip
In parallel, you need a 3rd party package
\begin{itemize}
  \item MUMPS (\verb|--download-mumps|)
  \medskip
  \item SuperLU (\verb|--download-superlu_dist|)
\end{itemize}
\end{frame}
%%     What is LU fails? Use full Schur complement
\begin{frame}[fragile]{What if LU fails?}\Large
LU will fail for
\begin{itemize}
  \item Singular problems
  \medskip
  \item Saddle-point problems
\end{itemize}
\pause
\smallskip
For saddles use \magenta{\href{http://www.mcs.anl.gov/petsc/petsc-current/docs/manualpages/PC/PCFIELDSPLIT.html}{\texttt{PC\_FIELDSPLIT}}}
\begin{itemize}
  \item Separately solves each field
  \medskip
  \item Decomposition is automatic in PyLith
  \medskip
  \item Autodetect with {\small \verb|-pc_fieldsplit_detect_saddle_point|}
  \medskip
  \item Exact with full Schur complement solve
\end{itemize}
\end{frame}
%
%
\section{How do I improve?}
\subsection{Look at what you have}
%
\begin{frame}[fragile]{What solver did you use?}
Use \verb|-snes_view| or \verb|-ksp_view| to output a description of the solver:
\small
\begin{verbatim}
KSP Object:        (fieldsplit_0_)         1 MPI processes
  type: fgmres
    GMRES: restart=100, using Classical (unmodified) Gram-
      Schmidt Orthogonalization with no iterative refinement
    GMRES: happy breakdown tolerance 1e-30
  maximum iterations=1, initial guess is zero
  tolerances:  relative=1e-09, absolute=1e-50,
    divergence=10000
  right preconditioning
  has attached null space
  using UNPRECONDITIONED norm type for convergence test
\end{verbatim}
\end{frame}
%
\begin{frame}[fragile]{What did the convergence look like?}
Use \verb|-snes_monitor| and \verb|-ksp_monitor|, or \verb|-log_summary|:
\small
\begin{overprint}
\onslide<2>
\smallskip
\begin{verbatim}
  0 SNES Function norm 0.207564 
  1 SNES Function norm 0.0148968 
  2 SNES Function norm 0.000113968 
  3 SNES Function norm 6.9256e-09 
  4 SNES Function norm < 1.e-11
\end{verbatim}
\onslide<3>
\smallskip
\begin{verbatim}
  0 KSP Residual norm 1.61409 
      Residual norms for mg_levels_1_ solve.
      0 KSP Residual norm 0.213376 
      1 KSP Residual norm 0.0192085 
    Residual norms for mg_levels_2_ solve.
    0 KSP Residual norm 0.223226 
    1 KSP Residual norm 0.0219992 
      Residual norms for mg_levels_1_ solve.
      0 KSP Residual norm 0.0248252 
      1 KSP Residual norm 0.0153432 
    Residual norms for mg_levels_2_ solve.
    0 KSP Residual norm 0.0124024 
    1 KSP Residual norm 0.0018736 
  1 KSP Residual norm 0.02282 
\end{verbatim}
\onslide<4>
\smallskip
\begin{verbatim}
Event       Count      Time (sec)     Flops      Total
           Max Ratio  Max     Ratio   Max  Ratio Mflop/s
KSPSetUp      12 1.0 3.6259e-03 1.0 0.00e+00 0.0     0
KSPSolve       3 1.0 4.8937e-01 1.0 8.93e+05 1.0     2
SNESSolve      1 1.0 4.9477e-01 1.0 9.22e+05 1.0     2
\end{verbatim}
\end{overprint}
\end{frame}
%
\begin{frame}[fragile]{Look at timing}
Use \verb|-log_summary|:
\small
\smallskip
\begin{verbatim}
Event        Time (sec)     Flops     --- Global ---  Total
           Max     Ratio   Max  Ratio %T %f %M %L %R  MF/s
VecMDot   1.8904e-03 1.0 2.49e+04 1.0  0  3  0  0  0    13
MatMult   2.1026e-03 1.0 2.65e+05 1.0  0 29  0  0  0   126
PCApply   4.6001e-01 1.0 7.78e+05 1.0 58 84  0  0 64     2
KSPSetUp  3.6259e-03 1.0 0.00e+00 0.0  0  0  0  0  4     0
KSPSolve  4.8937e-01 1.0 8.93e+05 1.0 61 97  0  0 90     2
SNESSolve 4.9477e-01 1.0 9.22e+05 1.0 62100  0  0 92     2
\end{verbatim}
\pause
Use \verb|-log_view ::ascii_info_detail| to get this information as a Python module
\end{frame}
%
\subsection{Back off in steps}
%
\begin{frame}[fragile]{Weaken the KSP}

 GMRES\quad $\Longrightarrow$\quad BiCGStab
\begin{itemize}
  \item \verb|-ksp_type bcgs|
  \medskip
  \item Less storage
  \medskip
  \item Fewer dot products (less work)
  \medskip
  \item Variants \verb|-ksp_type bcgsl| and \verb|-ksp_type ibcgs|
\end{itemize}
\bigskip
\begin{center}
  Complete \magenta{\href{http://www.mcs.anl.gov/petsc/documentation/linearsolvertable.html}{Table}} of Solvers and Preconditioners
\end{center}
\end{frame}
%
\begin{frame}[fragile]{Weaken the PC}

 LU\quad $\Longrightarrow$\quad ILU
\begin{itemize}
  \item \verb|-pc_type ilu|
  \medskip
  \item Less storage and work
\end{itemize}
\pause
\smallskip
In parallel,
\begin{itemize}
  \item Hypre \verb|-pc_type hypre -pc_hypre_type euclid|
  \medskip
  \item Block-Jacobi \verb|-pc_type bjacobi -sub_pc_type ilu|
  \medskip
  \item Additive Schwarz \verb|-pc_type asm -sub_pc_type ilu|
\end{itemize}
\pause
\bigskip
\begin{center}
  Default for MG smoother is Chebychev/SOR(2)
\end{center}
\end{frame}
%
\begin{frame}[fragile]{Algebraic Multigrid (AMG)}
\begin{itemize}
  \item Can solve elliptic problems
  \begin{itemize}
    \item Laplace, elasticity, Stokes
  \end{itemize}
  \medskip
  \item Works for unstructured meshes
  \medskip
  \item \verb|-pc_type gamg|,  \verb|-pc_type ml|, \verb|-pc_type hypre -pc_hypre_type boomeramg|
  \medskip
  \item \red{CRUCIAL} to have an accurate near-null space
  \begin{itemize}
    \item \magenta{\href{http://www.mcs.anl.gov/petsc/petsc-current/docs/manualpages/Mat/MatSetNearNullSpace.html}{MatSetNearNullSpace()}}
    \item PyLith provides this automatically
  \end{itemize}
  \medskip
  \item Use \verb|-pc_mg_log| to put timing in its own log stage
\end{itemize}
\end{frame}
%
\begin{frame}[fragile]{PC\_FieldSplit}
\begin{itemize}
  \item {\Large Separate solves for block operators}
  \begin{itemize}
    \item Physical insight for subsystems
    \smallskip
    \item Have optimial PCs for simpler equations
    \smallskip
    \item Suboptions \verb|fs_fieldsplit_0_*|
  \end{itemize}
  \bigskip
  \item {\Large Flexibly combine subsolves}
  \begin{itemize}
    \item Jacobi: \verb|fs_pc_fieldsplit_type = additive|
    \smallskip
    \item Gauss-Siedel: \verb|fs_pc_fieldsplit_type = multiplicative|
    \smallskip
    \item Schur complement: \verb|fs_pc_fieldsplit_type = schur|
  \end{itemize}
\end{itemize}
\end{frame}
%
\begin{frame}[fragile]{Stokes example}
The common block preconditioners for Stokes require only options:
\begin{overprint}
\onslide<1>
\begin{columns}
\begin{column}[t]{0.67\textwidth}
\begin{center}\Huge The Stokes System\end{center}
\end{column}
\begin{column}[t]{0.33\textwidth}
\Huge
\begin{equation*}
\begin{pmatrix}
A & B \\
B^T & 0
\end{pmatrix}
\end{equation*}
\end{column}
\end{columns}
%
\onslide<2>
\begin{columns}
\begin{column}[t]{0.67\textwidth}
\begin{itemize}
  \item[] \code{-pc\_type fieldsplit}
  \item[] \code{-pc\_field\_split\_type additive}
  \medskip
  \item[] \code{-fieldsplit\_0\_pc\_type ml}
  \item[] \code{-fieldsplit\_0\_ksp\_type preonly}
  \medskip
  \item[] \code{-fieldsplit\_1\_pc\_type jacobi}
  \item[] \code{-fieldsplit\_1\_ksp\_type preonly}
\end{itemize}
\end{column}
\begin{column}[t]{0.33\textwidth}
\Huge\begin{center}PC\end{center}
\begin{equation*}
\begin{pmatrix}
\hat A & 0 \\
    0  & I
\end{pmatrix}
\end{equation*}
\end{column}
\end{columns}
\medskip
\scriptsize
  Cohouet \& Chabard, \emph{Some fast 3D finite element solvers for the generalized Stokes problem}, 1988.
%
\onslide<3>
\begin{columns}
\begin{column}[t]{0.67\textwidth}
\begin{itemize}
  \item[] \code{-pc\_type fieldsplit}
  \item[] \code{-pc\_field\_split\_type multiplic}
  \medskip
  \item[] \code{-fieldsplit\_0\_pc\_type hypre}
  \item[] \code{-fieldsplit\_0\_ksp\_type preonly}
  \medskip
  \item[] \code{-fieldsplit\_1\_pc\_type jacobi}
  \item[] \code{-fieldsplit\_1\_ksp\_type preonly}
\end{itemize}
\end{column}
\begin{column}[t]{0.33\textwidth}
\Huge\begin{center}PC\end{center}
\begin{equation*}
\begin{pmatrix}
\hat A & B \\
    0  & I
\end{pmatrix}
\end{equation*}
\end{column}
\end{columns}
\medskip
\scriptsize
  Elman, \emph{Multigrid and Krylov subspace methods for the discrete Stokes equations}, 1994.
%
\onslide<4>
\begin{columns}
\begin{column}[t]{0.67\textwidth}
\begin{itemize}
  \item[] \code{-pc\_type fieldsplit}
  \item[] \code{-pc\_field\_split\_type schur}
  \medskip
  \item[] \code{-fieldsplit\_0\_pc\_type gamg}
  \item[] \code{-fieldsplit\_0\_ksp\_type preonly}
  \medskip
  \item[] \code{-fieldsplit\_1\_pc\_type none}
  \item[] \code{-fieldsplit\_1\_ksp\_type minres}
\end{itemize}
\end{column}
\begin{column}[t]{0.33\textwidth}
\Huge\begin{center}PC\end{center}
\begin{equation*}
\begin{pmatrix}
\hat A & 0 \\
    0  & -\hat S
\end{pmatrix}
\end{equation*}
\end{column}
\end{columns}
\begin{itemize}
  \item[] \code{-pc\_fieldsplit\_schur\_factorization\_type diag}
\end{itemize}
\scriptsize
  May and Moresi, \emph{Preconditioned iterative methods for Stokes flow problems arising in computational geodynamics}, 2008.\\
  Olshanskii, Peters, and Reusken, \emph{Uniform preconditioners for a parameter dependent saddle point problem with application to generalized Stokes interface equations}, 2006.
%
\onslide<5>
\begin{columns}
\begin{column}[t]{0.67\textwidth}
\begin{itemize}
  \item[] \code{-pc\_type fieldsplit}
  \item[] \code{-pc\_field\_split\_type schur}
  \medskip
  \item[] \code{-fieldsplit\_0\_pc\_type gamg}
  \item[] \code{-fieldsplit\_0\_ksp\_type preonly}
  \medskip
  \item[] \code{-fieldsplit\_1\_pc\_type none}
  \item[] \code{-fieldsplit\_1\_ksp\_type minres}
\end{itemize}
\end{column}
\begin{column}[t]{0.33\textwidth}
\Huge\begin{center}PC\end{center}
\begin{equation*}
\begin{pmatrix}
\hat A  & 0 \\
    B^T & \hat S
\end{pmatrix}
\end{equation*}
\end{column}
\end{columns}
\begin{itemize}
  \item[] \code{-pc\_fieldsplit\_schur\_factorization\_type lower}
\end{itemize}
\scriptsize
  May and Moresi, \emph{Preconditioned iterative methods for Stokes flow problems arising in computational geodynamics}, 2008.
%
\onslide<6>
\begin{columns}
\begin{column}[t]{0.67\textwidth}
\begin{itemize}
  \item[] \code{-pc\_type fieldsplit}
  \item[] \code{-pc\_field\_split\_type schur}
  \medskip
  \item[] \code{-fieldsplit\_0\_pc\_type gamg}
  \item[] \code{-fieldsplit\_0\_ksp\_type preonly}
  \medskip
  \item[] \code{-fieldsplit\_1\_pc\_type none}
  \item[] \code{-fieldsplit\_1\_ksp\_type minres}
\end{itemize}
\end{column}
\begin{column}[t]{0.33\textwidth}
\Huge\begin{center}PC\end{center}
\begin{equation*}
\begin{pmatrix}
\hat A & B \\
     0 & \hat S
\end{pmatrix}
\end{equation*}
\end{column}
\end{columns}
\begin{itemize}
  \item[] \code{-pc\_fieldsplit\_schur\_factorization\_type upper}
\end{itemize}
\scriptsize
  May and Moresi, \emph{Preconditioned iterative methods for Stokes flow problems arising in computational geodynamics}, 2008.
%
\onslide<7>
\begin{columns}
\begin{column}[t]{0.67\textwidth}
\begin{itemize}
  \item[] \code{-pc\_type fieldsplit}
  \item[] \code{-pc\_field\_split\_type schur}
  \medskip
  \item[] \code{-fieldsplit\_0\_pc\_type gamg}
  \item[] \code{-fieldsplit\_0\_ksp\_type preonly}
  \medskip
  \item[] \code{-fieldsplit\_1\_pc\_type lsc}
  \item[] \code{-fieldsplit\_1\_ksp\_type minres}
\end{itemize}
\end{column}
\begin{column}[t]{0.33\textwidth}
\Huge\begin{center}PC\end{center}
\begin{equation*}
\begin{pmatrix}
\hat A & B \\
     0 & \hat S_{\mathrm{LSC}}
\end{pmatrix}
\end{equation*}
\end{column}
\end{columns}
\begin{itemize}
  \item[] \code{-pc\_fieldsplit\_schur\_factorization\_type upper}
\end{itemize}
\scriptsize
  May and Moresi, \emph{Preconditioned iterative methods for Stokes flow problems arising in computational geodynamics}, 2008.\\
  Kay, Loghin and Wathen, \emph{A Preconditioner for the Steady-State N-S Equations}, 2002.\\
  Elman, Howle, Shadid, Shuttleworth, and Tuminaro, \emph{Block preconditioners based on approximate commutators}, 2006.
%
\onslide<8>
\begin{itemize}
  \item[] \code{-pc\_type fieldsplit}
  \item[] \code{-pc\_field\_split\_type schur}
  \item[] \code{-pc\_fieldsplit\_schur\_factorization\_type full}
\end{itemize}
\Huge
\begin{center}PC\end{center}
\begin{equation*}
\begin{pmatrix}
      I        &  0 \\
    B^T A^{-1}  &  I
\end{pmatrix}
\begin{pmatrix}
\hat A & 0 \\
    0  & \hat S
\end{pmatrix}
\begin{pmatrix}
 I & A^{-1} B \\
 0 & I
\end{pmatrix}
\end{equation*}
\end{overprint}
\end{frame}
%
%
\ifx\@restTalk\@empty
\begin{frame}[fragile]{Stokes example}
All block preconditioners can be \textit{embedded} in MG using only options:
\begin{columns}
\begin{column}[c]{0.67\textwidth}
\begin{itemize}\scriptsize
  \item[]<1-> \code{-pc\_type mg -pc\_mg\_levels 5 -pc\_mg\_galerkin}
  \item[]<2-> \code{-mg\_levels\_pc\_type fieldsplit}
  \item[]<2-> \code{-mg\_levels\_pc\_field\_split\_type \only<2>{additive}\only<3>{multiplicative}\only<4->{schur}}

  \only<2-7>{
  \medskip
  \item[]<2-> \code{-mg\_levels\_fieldsplit\_0\_pc\_type gamg}
  \item[]<2-> \code{-mg\_levels\_fieldsplit\_0\_ksp\_type preonly}

  \medskip

  \item[]<2-> \only<2-3>{\code{-mg\_levels\_fieldsplit\_1\_pc\_type jacobi}}  \only<4-6>{\code{-mg\_levels\_fieldsplit\_1\_pc\_type none}} \only<7>{\code{-mg\_levels\_fieldsplit\_1\_pc\_type lsc}}
  \item[]<2-> \only<2-3>{\code{-mg\_levels\_fieldsplit\_1\_ksp\_type preonly}}\only<4->{\code{-mg\_levels\_fieldsplit\_1\_ksp\_type minres}}
  }
\end{itemize}
\end{column}
%
\only<2-7>{
\Huge
\only<2-7>{\begin{center}Smoother PC\end{center}}
\begin{equation*}
\left(
\only<2>{
\begin{matrix}
\hat A & 0 \\
    0  & I
\end{matrix}
}
\only<3>{
\begin{matrix}
\hat A & B \\
    0  & I
\end{matrix}
}
\only<4>{
\begin{matrix}
\hat A & 0 \\
    0  & -\hat S
\end{matrix}
}
\only<5>{
\begin{matrix}
\hat A & 0 \\
  B^T  & \hat S
\end{matrix}
}
\only<6>{
\begin{matrix}
\hat A & B \\
    0  & \hat S
\end{matrix}
}
\only<7>{
\begin{matrix}
\hat A & B \\
    0  & \hat S_{\text{LSC}}
\end{matrix}
}
\right)
\end{equation*}
}
\end{column}
\end{columns}

\medskip

\only<4->{\scriptsize
\hskip1pt\quad\ \ \code{-mg\_levels\_pc\_fieldsplit\_schur\_factorization\_type \only<4>{diag}\only<5>{lower}\only<6-7>{upper}}
}

\only<1>{
\Huge
\begin{center}\Huge System on each Coarse Level\end{center}
\begin{equation*}
R \left(
\begin{matrix}
 A    & B \\
 B^T  & 0
\end{matrix}
\right) P
\end{equation*}
}
\end{frame}
\else
  \relax
\fi

%
\begin{frame}{Why use FGMRES?}\Large
Flexible GMRES (FGMRES) allows a\\
\blue{different preconditioner} at each step:
\medskip
\begin{itemize}
  \item Takes twice the memory
  \medskip
  \item Needed for iterative PCs
  \medskip
  \item Avoided sometimes with a careful PC choice
\end{itemize}
\end{frame}
%
%
\section{Can We Do It?}
%
\begin{frame}{Example 3D Hex8 \texttt{step10.cfg}}
\begin{center}\Huge
Looks nice,\\
But can you do this for\\
a real PyLith Example?
\end{center}
\end{frame}
%
\begin{frame}[fragile]{Example 3D Hex8 \texttt{step10.cfg}}

\begin{block}{First, we try LU on the whole problem \texttt{solver\_lu.cfg}}
\begin{verbatim}
[pylithapp.petsc]
snes_view = true
pc_type = lu
\end{verbatim}
\end{block}
\pause
\begin{center}
  \Huge \red{FAIL}
\end{center}
\pause
This is due to the saddle point introduced to handle the fault.
\end{frame}
%
\begin{frame}[fragile]{Example 3D Hex8 \texttt{step10.cfg}}

\begin{block}{Next, we split fields using \texttt{PC\_FIELDSPLIT} \texttt{solver\_fault\_additive.cfg}}
\small
% solver02.cfg
\begin{verbatim}
[pylithapp.timedependent.formulation]
matrix_type = aij
split_fields = True

[pylithapp.petsc]
ksp_max_it = 1000
fs_pc_type = fieldsplit
fs_pc_use_amat = true
fs_pc_fieldsplit_type = additive
fs_fieldsplit_displacement_ksp_type = preonly
fs_fieldsplit_displacement_pc_type = lu
fs_fieldsplit_lagrange_multiplier_ksp_type = preonly
fs_fieldsplit_lagrange_multiplier_pc_type = jacobi
\end{verbatim}
\end{block}
\pause
\begin{center}
  Converges in 58 itertions because preconditioner is not that strong
\end{center}
\end{frame}
%
\begin{frame}[fragile]{Example 3D Hex8 \texttt{step10.cfg}}

% solver03.cfg
\begin{block}{We need to use a full Schur factorization \texttt{solver\_fault\_exact.cfg}}
\begin{verbatim}
[pylithapp.petsc]
fs_pc_type = fieldsplit
fs_pc_use_amat = true
fs_pc_fieldsplit_type = schur
fs_pc_fieldsplit_schur_factorization_type = full
fs_fieldsplit_displacement_ksp_type = preonly
fs_fieldsplit_displacement_pc_type = lu
fs_fieldsplit_lagrange_multiplier_pc_type = jacobi
fs_fieldsplit_lagrange_multiplier_ksp_type = gmres
fs_fieldsplit_lagrange_multiplier_ksp_rtol = 1.0e-11
\end{verbatim}
\end{block}
\pause
\begin{center}
  Works in one iterate! This is good for checking the physics.
\end{center}
\end{frame}
%
\begin{frame}[fragile]{Example 3D Hex8 \texttt{step10.cfg}}

\begin{block}{We can add a user defined preconditioner for the Schur complement}
\begin{verbatim}
[pylithapp.timedependent.formulation]
use_custom_constraint_pc = True
[pylithapp.petsc]
fs_pc_fieldsplit_schur_precondition = user
\end{verbatim}
\end{block}
\pause
\begin{overprint}
\onslide<2>
\begin{block}{The initial convergence}\scriptsize
\begin{verbatim}
0 SNES Function norm 1.547533880440e-02 
  Linear solve converged due to CONVERGED_RTOL iterations 30
  0 KSP Residual norm 1.158385264202e-02 
  Linear solve converged due to CONVERGED_RTOL iterations 30
  1 KSP Residual norm 2.231623131220e-13 
Linear solve converged due to CONVERGED_RTOL iterations 1
1 SNES Function norm 1.146037096697e-13 
\end{verbatim}
\end{block}
\onslide<3>
\begin{block}{improves to \texttt{solver\_fault\_schur\_custompc.cfg}}{\scriptsize
\begin{verbatim}
0 SNES Function norm 1.547533880440e-02 
  Linear solve converged due to CONVERGED_RTOL iterations 24
  0 KSP Residual norm 1.158385264203e-02 
  Linear solve converged due to CONVERGED_RTOL iterations 25
  1 KSP Residual norm 5.404403812155e-14 
Linear solve converged due to CONVERGED_RTOL iterations 1
1 SNES Function norm 2.201265688755e-14 
\end{verbatim}}
and gets much better for larger problems.
\end{block}
\end{overprint}
\end{frame}
%
\begin{frame}[fragile]{Example 3D Hex8 \texttt{step10.cfg}}

\begin{block}{You can back off the Schur complement tolerance}
\begin{verbatim}
ksp_type = fgmres
fs_fieldsplit_lagrange_multiplier_ksp_rtol = 1.0e-05
\end{verbatim}
at the cost of more iterates
{\scriptsize
\begin{verbatim}
0 SNES Function norm 1.547533880440e-02 
  0 KSP Residual norm 1.547533880440e-02 
  Linear solve converged due to CONVERGED_RTOL iterations 10
  1 KSP Residual norm 9.761444929927e-08 
  Linear solve converged due to CONVERGED_RTOL iterations 15
  2 KSP Residual norm 4.058177976336e-13 
Linear solve converged due to CONVERGED_RTOL iterations 2
1 SNES Function norm 2.763748407804e-13 
\end{verbatim}
}
\end{block}
\end{frame}
%
\begin{frame}[fragile]{Example 3D Hex8 \texttt{step10.cfg}}

\begin{block}{You can back off the primal LU solver}
\begin{verbatim}
fs_fieldsplit_displacement_ksp_type = preonly
fs_fieldsplit_displacement_pc_type  = gamg
\end{verbatim}
at the cost of more iterates
{\scriptsize
\begin{verbatim}
0 SNES Function norm 1.547533880440e-02 
  0 KSP Residual norm 1.547533880440e-02 
  Linear solve converged due to CONVERGED_RTOL iterations 12
  1 KSP Residual norm 3.659593456893e-04 
  Linear solve converged due to CONVERGED_RTOL iterations 15
  2 KSP Residual norm 9.111591440754e-06 
  Linear solve converged due to CONVERGED_RTOL iterations 16
  .
  .
  6 KSP Residual norm 3.526238448332e-12 
  Linear solve converged due to CONVERGED_RTOL iterations 17
  7 KSP Residual norm 8.640836102392e-14 
Linear solve converged due to CONVERGED_RTOL iterations 7
1 SNES Function norm 8.641267905609e-14 
\end{verbatim}
}
\end{block}
\end{frame}
%
\begin{frame}[fragile]{Example 3D Hex8 \texttt{step10.cfg}}

\begin{block}{You can restore the behavior with a lower tolerance}
\begin{verbatim}
fs_fieldsplit_displacement_ksp_type = gmres
fs_fieldsplit_displacement_ksp_rtol = 5.0e-10
\end{verbatim}
but it is quite sensitive to the tolerance.
{\scriptsize
\begin{verbatim}
0 SNES Function norm 1.547533880440e-02 
  0 KSP Residual norm 1.547533880440e-02 
  Linear solve converged due to CONVERGED_RTOL iterations 10
  1 KSP Residual norm 9.761445192979e-08 
  Linear solve converged due to CONVERGED_RTOL iterations 15
  2 KSP Residual norm 7.227466516039e-13 
Linear solve converged due to CONVERGED_RTOL iterations 2
1 SNES Function norm 2.391873654238e-13 
\end{verbatim}
}
\end{block}
\end{frame}
%
%
\section{Nonlinear Systems}
%
  \begin{frame}[fragile]{Driven Cavity Problem}

\magenta{\href{http://www.mcs.anl.gov/petsc/petsc-current/src/snes/examples/tutorials/ex19.c.html}{SNES ex19.c}}
\smallskip
\begin{semiverbatim}\scriptsize
./ex19 -lidvelocity 100 \only<1-2>{-grashof 1e2}\only<3-4>{\blue{-grashof 1e4}}\only<5-7>{\blue{-grashof 1e5}}
  -da\_grid\_x 16 -da\_grid\_y 16 -da\_refine 2 \only<7>{\blue{-pc\_type lu}}
  -snes\_monitor\_short -snes\_converged\_reason -snes\_view
\end{semiverbatim}
\smallskip
\begin{overprint}\scriptsize
\onslide<2>
\begin{semiverbatim}
lid velocity = 100, prandtl # = 1, grashof # = 100
  0 SNES Function norm 768.116 
  1 SNES Function norm 658.288 
  2 SNES Function norm 529.404 
  3 SNES Function norm 377.51 
  4 SNES Function norm 304.723 
  5 SNES Function norm 2.59998 
  6 SNES Function norm 0.00942733 
  7 SNES Function norm 5.20667e-08 
Nonlinear solve converged due to CONVERGED_FNORM_RELATIVE iterations 7
\end{semiverbatim}
\onslide<4>
\begin{semiverbatim}
lid velocity = 100, prandtl # = 1, grashof # = 10000
  0 SNES Function norm 785.404 
  1 SNES Function norm 663.055 
  2 SNES Function norm 519.583 
  3 SNES Function norm 360.87 
  4 SNES Function norm 245.893 
  5 SNES Function norm 1.8117 
  6 SNES Function norm 0.00468828 
  7 SNES Function norm 4.417e-08 
Nonlinear solve converged due to CONVERGED_FNORM_RELATIVE iterations 7
\end{semiverbatim}
\onslide<6>
\begin{semiverbatim}
lid velocity = 100, prandtl # = 1, grashof # = 100000
  0 SNES Function norm 1809.96 
Nonlinear solve did not converge due to DIVERGED_LINEAR_SOLVE iterations 0
\end{semiverbatim}
\onslide<7>
\begin{semiverbatim}
lid velocity = 100, prandtl # = 1, grashof # = 100000
  0 SNES Function norm 1809.96 
  1 SNES Function norm 1678.37 
  2 SNES Function norm 1643.76 
  3 SNES Function norm 1559.34 
  4 SNES Function norm 1557.6 
  5 SNES Function norm 1510.71 
  6 SNES Function norm 1500.47 
  7 SNES Function norm 1498.93 
  8 SNES Function norm 1498.44 
  9 SNES Function norm 1498.27 
 10 SNES Function norm 1498.18 
 11 SNES Function norm 1498.12 
 12 SNES Function norm 1498.11 
 13 SNES Function norm 1498.11 
 14 SNES Function norm 1498.11 
 \ldots
\end{semiverbatim}
\end{overprint}
\end{frame}

  \begin{frame}{Why isn't SNES converging?}

\begin{itemize}
  \item The Jacobian is wrong (maybe only in parallel)
  \begin{itemize}
    \item Check with \code{-snes\_check\_jacobian -snes\_check\_jacobian\_view}
  \end{itemize}
  \item The linear system is not solved accurately enough
  \begin{itemize}
    \item Check with \code{-pc\_type lu}
    \item Check \code{-ksp\_monitor\_true\_residual}, try right preconditioning
  \end{itemize}
  \item The Jacobian is singular with inconsistent right side
  \begin{itemize}
    \item Use \class{MatNullSpace} to inform the \class{KSP} of a known null space
    \item Use a different Krylov method or preconditioner
  \end{itemize}
  \item The nonlinearity is just really strong
  \begin{itemize}
    \item Run with \code{-info} or \code{-snes\_ls\_monitor} to see line search
    \item Try using trust region instead of line search \code{-snes\_type tr}
    \item Try grid sequencing if possible \code{-snes\_grid\_sequence}
    \item Use a continuation
  \end{itemize}
\end{itemize}
\end{frame}

  \begin{frame}[fragile]{Nonlinear Preconditioning}
\begin{columns}
\begin{column}[t]{0.45\textwidth}
{\Large \magenta{\href{http://www.mcs.anl.gov/petsc/petsc-dev/docs/manualpages/PC/PC.html}{\class{PC}}} preconditions \magenta{\href{http://www.mcs.anl.gov/petsc/petsc-dev/docs/manualpages/KSP/KSP.html}{\class{KSP}}}}
\begin{semiverbatim}
  -ksp_type gmres

  -pc_type richardson
\end{semiverbatim}
\end{column}
%
\pause
\begin{column}[t]{0.55\textwidth}
{\Large \magenta{\href{http://www.mcs.anl.gov/petsc/petsc-dev/docs/manualpages/SNES/SNES.html}{\class{SNES}}} preconditions \magenta{\href{http://www.mcs.anl.gov/petsc/petsc-dev/docs/manualpages/SNES/SNES.html}{\class{SNES}}}}
\begin{semiverbatim}
  -snes_type ngmres

  -npc_snes_type nrichardson
\end{semiverbatim}
\end{column}
\end{columns}
\end{frame}

  \begin{frame}[fragile]{Nonlinear Use Cases}

{\bf Warm start Newton}
\begin{semiverbatim}
  -snes_type newtonls
  -npc_snes_type nrichardson -npc_snes_max_it 5
\end{semiverbatim}

{\bf Cleanup noisy Jacobian}
\begin{semiverbatim}
  -snes_type ngmres -snes_ngmres_m 5
  -npc_snes_type newtonls
\end{semiverbatim}

{\bf Additive-Schwarz Preconditioned Inexact Newton}
\begin{semiverbatim}
  -snes_type aspin -snes_npc_side left
  -npc_snes_type nasm -npc_snes_nasm_type restrict
\end{semiverbatim}
\end{frame}

  \begin{frame}[fragile]{Nonlinear Preconditioning}
\only<1>{\subtitle{Also called \textit{globalization}}}
\only<11>{\LARGE
  See discussion in:\\
  \vskip1em
  \magenta{\Large\href{http://dx.doi.org/10.1137/130936725}{Composing Scalable Nonlinear Algebraic Solvers}},\\
  {\large Peter Brune, Matthew Knepley, Barry Smith, and Xuemin Tu,\\
  SIAM Review, {\bf 57}(4), 535--565, 2015.}\\
  \magenta{\normalsize\href{http://www.mcs.anl.gov/uploads/cels/papers/P2010-0112.pdf}{http://www.mcs.anl.gov/uploads/cels/papers/P2010-0112.pdf}}
}
\only<10>{
\begin{tabular}{l|ccccccc}
Solver & T & N. It & L. It & Func & Jac & PC & NPC \\
\hline
$(\NK\pc\MG)$ & 9.83 & 17 & 352 & 34 & 85 & 370 & -- \\
$\NGMRES\rp{}$ & 7.48 & 10 & 220 & 21 & 50 & 231 & 10 \\
$\ (\NK\pc\MG)$&      &    &     &    &    &     & \\
$\FAS$ & 6.23 & 162 & 0 & 2382 & 377 & 754 & -- \\
$\FAS+(\NK\pc\MG)$ & 8.07 & 10 & 197 & 232 & 90 & 288 & -- \\
$\FAS*(\NK\pc\MG)$ & 4.01 & 5 & 80 & 103 & 45 & 125 & -- \\
$\NRICH\lp\FAS$ & 3.20 & 50 & 0 & 1180 & 192 & 384 & 50 \\
$\NGMRES\rp\FAS$ & 1.91 & 24 & 0 & 447 & 83 & 166 & 24 \\
\end{tabular}
}
\begin{overprint}\scriptsize
\onslide<1>
\begin{semiverbatim}
./ex19 -lidvelocity 100 -grashof 5e4 -da_refine 4 -snes_monitor_short
 -snes_type newtonls -snes_converged_reason
 \blue{-pc_type lu}
\end{semiverbatim}
\onslide<2>
\begin{semiverbatim}
./ex19 -lidvelocity 100 -grashof 5e4 -da_refine 4 -snes_monitor_short
 \blue{-snes_type fas} -snes_converged_reason
 \blue{-fas_levels_snes_type gs -fas_levels_snes_max_it 6}
\end{semiverbatim}
\onslide<3>
\begin{semiverbatim}
./ex19 -lidvelocity 100 -grashof 5e4 -da_refine 4 -snes_monitor_short
 -snes_type fas -snes_converged_reason
 -fas_levels_snes_type gs -fas_levels_snes_max_it 6
  \blue{-fas_coarse_snes_converged_reason}
\end{semiverbatim}
\onslide<4>
\begin{semiverbatim}
./ex19 -lidvelocity 100 -grashof 5e4 -da_refine 4 -snes_monitor_short
 -snes_type fas -snes_converged_reason
 -fas_levels_snes_type gs -fas_levels_snes_max_it 6
  \blue{-fas_coarse_snes_linesearch_type basic}
  -fas_coarse_snes_converged_reason
\end{semiverbatim}
\onslide<5>
\begin{semiverbatim}
./ex19 -lidvelocity 100 -grashof 5e4 -da_refine 4 -snes_monitor_short
 \blue{-snes_type nrichardson -npc_snes_max_it 1} -snes_converged_reason
 \blue{-npc_}snes_type fas \blue{-npc_}fas_coarse_snes_converged_reason
  \blue{-npc_}fas_levels_snes_type gs \blue{-npc_}fas_levels_snes_max_it 6
  \blue{-npc_}fas_coarse_snes_linesearch_type basic
\end{semiverbatim}
\onslide<6>
\begin{semiverbatim}
./ex19 -lidvelocity 100 -grashof 5e4 -da_refine 4 -snes_monitor_short
 \blue{-snes_type ngmres} -npc_snes_max_it 1 -snes_converged_reason
 -npc_snes_type fas -npc_fas_coarse_snes_converged_reason
  -npc_fas_levels_snes_type gs -npc_fas_levels_snes_max_it 6
  -npc_fas_coarse_snes_linesearch_type basic
\end{semiverbatim}
\onslide<7>
\begin{semiverbatim}
./ex19 -lidvelocity 100 -grashof 5e4 -da_refine 4 -snes_monitor_short
 -snes_type ngmres -npc_snes_max_it 1 -snes_converged_reason
 -npc_snes_type fas -npc_fas_coarse_snes_converged_reason
 \blue{-npc_fas_levels_snes_type newtonls} -npc_fas_levels_snes_max_it 6
  \blue{-npc_fas_levels_snes_linesearch_type basic}
  \blue{-npc_fas_levels_snes_max_linear_solve_fail 30}
  \blue{-npc_fas_levels_ksp_max_it 20 -npc_fas_levels_snes_converged_reason}
  -npc_fas_coarse_snes_linesearch_type basic
\end{semiverbatim}
\onslide<8>
\begin{semiverbatim}
./ex19 -lidvelocity 100 -grashof 5e4 -da_refine 4 -snes_monitor_short
 \blue{-snes_type composite -snes_composite_type additiveoptimal}
 \blue{-snes_composite_sneses fas,newtonls} -snes_converged_reason  
 \blue{-sub_0_fas_levels_snes_type gs -sub_0_fas_levels_snes_max_it 6}
   \blue{-sub_0_fas_coarse_snes_linesearch_type basic}
 \blue{-sub_1_snes_linesearch_type basic -sub_1_pc_type mg}
\end{semiverbatim}
\onslide<9>
\begin{semiverbatim}
./ex19 -lidvelocity 100 -grashof 5e4 -da_refine 4 -snes_monitor_short
 -snes_type composite \blue{-snes_composite_type multiplicative}
 -snes_composite_sneses fas,newtonls -snes_converged_reason  
 -sub_0_fas_levels_snes_type gs -sub_0_fas_levels_snes_max_it 6
   -sub_0_fas_coarse_snes_linesearch_type basic
 -sub_1_snes_linesearch_type basic -sub_1_pc_type mg
\end{semiverbatim}
\end{overprint}

\begin{overprint}\scriptsize
\onslide<1>
\begin{semiverbatim}
lid velocity = 100, prandtl \# = 1, grashof \# = 50000
  0 SNES Function norm 1228.95 
  1 SNES Function norm 1132.29 
  2 SNES Function norm 1026.17 
  3 SNES Function norm 925.717 
  4 SNES Function norm 924.778 
  5 SNES Function norm 836.867 
  \vdots
 21 SNES Function norm 585.143 
 22 SNES Function norm 585.142 
 23 SNES Function norm 585.142 
 24 SNES Function norm 585.142 
  \vdots
\end{semiverbatim}
\onslide<2>
\begin{semiverbatim}
lid velocity = 100, prandtl \# = 1, grashof \# = 50000
  0 SNES Function norm 1228.95 
  1 SNES Function norm 574.793 
  2 SNES Function norm 513.02 
  3 SNES Function norm 216.721 
  4 SNES Function norm 85.949 
Nonlinear solve did not converge due to DIVERGED_INNER iterations 4
\end{semiverbatim}
\onslide<3>
\begin{semiverbatim}
lid velocity = 100, prandtl \# = 1, grashof \# = 50000
  0 SNES Function norm 1228.95 
    Nonlinear solve converged due to CONVERGED_FNORM_RELATIVE its 12
  1 SNES Function norm 574.793 
    Nonlinear solve did not converge due to DIVERGED_MAX_IT its 50
  2 SNES Function norm 513.02 
    Nonlinear solve did not converge due to DIVERGED_MAX_IT its 50
  3 SNES Function norm 216.721 
    Nonlinear solve converged due to CONVERGED_FNORM_RELATIVE its 22
  4 SNES Function norm 85.949 
    Nonlinear solve did not converge due to DIVERGED_LINE_SEARCH its 42
Nonlinear solve did not converge due to DIVERGED_INNER iterations 4
\end{semiverbatim}
\onslide<4>
\begin{semiverbatim}
lid velocity = 100, prandtl \# = 1, grashof \# = 50000
  0 SNES Function norm 1228.95 
    Nonlinear solve converged due to CONVERGED_FNORM_RELATIVE its 6
  \vdots
 47 SNES Function norm 78.8401 
    Nonlinear solve converged due to CONVERGED_FNORM_RELATIVE its 5
 48 SNES Function norm 73.1185 
    Nonlinear solve converged due to CONVERGED_FNORM_RELATIVE its 6
 49 SNES Function norm 78.834 
    Nonlinear solve converged due to CONVERGED_FNORM_RELATIVE its 5
 50 SNES Function norm 73.1176 
    Nonlinear solve converged due to CONVERGED_FNORM_RELATIVE its 6
  \vdots
\end{semiverbatim}
\onslide<5>
\begin{semiverbatim}
lid velocity = 100, prandtl \# = 1, grashof \# = 50000
  0 SNES Function norm 1228.95 
    Nonlinear solve converged due to CONVERGED_FNORM_RELATIVE its 6
  1 SNES Function norm 552.271 
    Nonlinear solve converged due to CONVERGED_FNORM_RELATIVE its 27
  2 SNES Function norm 173.45 
    Nonlinear solve converged due to CONVERGED_FNORM_RELATIVE its 45
  \vdots
 43 SNES Function norm 3.45407e-05 
    Nonlinear solve converged due to CONVERGED_SNORM_RELATIVE its 2
 44 SNES Function norm 1.6141e-05 
    Nonlinear solve converged due to CONVERGED_SNORM_RELATIVE its 2
 45 SNES Function norm 9.13386e-06 
Nonlinear solve converged due to CONVERGED_FNORM_RELATIVE iterations 45
\end{semiverbatim}
\onslide<6>
\begin{semiverbatim}
lid velocity = 100, prandtl \# = 1, grashof \# = 50000
  0 SNES Function norm 1228.95 
    Nonlinear solve converged due to CONVERGED_FNORM_RELATIVE its 6
  1 SNES Function norm 538.605 
    Nonlinear solve converged due to CONVERGED_FNORM_RELATIVE its 13
  2 SNES Function norm 178.005 
    Nonlinear solve converged due to CONVERGED_FNORM_RELATIVE its 24
  \vdots
 27 SNES Function norm 0.000102487 
    Nonlinear solve converged due to CONVERGED_FNORM_RELATIVE its 2
 28 SNES Function norm 4.2744e-05 
    Nonlinear solve converged due to CONVERGED_SNORM_RELATIVE its 2
 29 SNES Function norm 1.01621e-05 
Nonlinear solve converged due to CONVERGED_FNORM_RELATIVE iterations 29
\end{semiverbatim}
\onslide<7>
\begin{semiverbatim}
lid velocity = 100, prandtl \# = 1, grashof \# = 50000
  0 SNES Function norm 1228.95 
    Nonlinear solve did not converge due to DIVERGED_MAX_IT its 6
    \vdots
        Nonlinear solve converged due to CONVERGED_SNORM_RELATIVE its 1
    \vdots
  1 SNES Function norm 0.1935 
  2 SNES Function norm 0.0179938 
  3 SNES Function norm 0.00223698 
  4 SNES Function norm 0.000190461 
  5 SNES Function norm 1.6946e-06 
Nonlinear solve converged due to CONVERGED_FNORM_RELATIVE iterations 5
\end{semiverbatim}
\onslide<8>
\begin{semiverbatim}
lid velocity = 100, prandtl \# = 1, grashof \# = 50000
  0 SNES Function norm 1228.95
  1 SNES Function norm 541.462
  2 SNES Function norm 162.92
  3 SNES Function norm 48.8138
  4 SNES Function norm 11.1822
  5 SNES Function norm 0.181469
  6 SNES Function norm 0.00170909
  7 SNES Function norm 3.24991e-08
Nonlinear solve converged due to CONVERGED_FNORM_RELATIVE iterations 7
\end{semiverbatim}
\onslide<9>
\begin{semiverbatim}
lid velocity = 100, prandtl \# = 1, grashof \# = 50000
  0 SNES Function norm 1228.95
  1 SNES Function norm 544.404
  2 SNES Function norm 18.2513
  3 SNES Function norm 0.488689
  4 SNES Function norm 0.000108712
  5 SNES Function norm 5.68497e-08
Nonlinear solve converged due to CONVERGED_FNORM_RELATIVE iterations 5
\end{semiverbatim}
\end{overprint}
\end{frame}

%
\begin{frame}{Other Solver Issues}
\begin{center}\Huge
For any other solver problems,
\end{center}
\bigskip
\begin{center}\Huge
contact \magenta{\href{mailto:cig-short@geodynamics.org}{cig-short@geodynamics.org}}
\end{center}
\end{frame}

\end{document}
