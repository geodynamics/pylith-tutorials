% -*- TeX -*-
\documentclass{beamer}

\usepackage{amsmath}
\usepackage{array}
\usepackage{tikz}

\usetikzlibrary{decorations.pathreplacing}
\usetikzlibrary{fit,matrix}

\title{Crustal Deformation Modeling Tutorial}
\subtitle{Computing Static Green's Functions}
\author{Charles Williams \\
  Brad Aagaard \\
  Matthew Knepley}
\institute{\includegraphics[scale=0.4]{../../logos/cig_blackfg}}
\date{June 23, 2014}


% ---------------------------------------------------- CUSTOMIZATION
\newcommand{\thispdfpagelabel}[1]{}
\newcommand{\pylith}[1]{{\color{green}#1}}
\newcommand{\python}[1]{{\color{red}#1}}
\usetheme{CIG}

\newcommand{\tensor}[1]{\overline{#1}}

% ========================================================= DOCUMENT
\begin{document}

% ------------------------------------------------------------ SLIDE
\maketitle

% ------------------------------------------------------------ SLIDE
\logo{\includegraphics[height=4.5ex]{../../logos/cig_blackfg}}

% ========================================================== SECTION
\section{Introduction}

% ------------------------------------------------------------ SLIDE
\begin{frame}
  \frametitle{Concepts Covered in this Session}
  \summary{}

  \begin{itemize}
  \item Generation of Green's functions in 2D and 3D
  \item Solution output at a specified set of points (\pylith{OutputSolnPoints})
  \item Postprocessing of HDF5 output using \python{h5py}
  \item Simple linear inversion using \python{numpy}
  \item Plotting of inversion results using \python{matplotlib}
  \end{itemize}
  
\end{frame}


% ========================================================== SECTION
\section{Overview}

% ------------------------------------------------------------ SLIDE
\begin{frame}
  \frametitle{Green's Functions}
  \summary{}

  \begin{itemize}
  \item Compute deformation due to unit (i.e., 1 m) slip at fault
    vertex for use in an inversion for fault slip
    \begin{itemize}
    \item Slip decreases {\bf linearly} to 0 at surrounding vertices
    \item Similar but not equivalent to uniform slip over a patch
      (Okada dislocation)
    \end{itemize}
  \item Provides ability to compute Green's functions with arbitrarily
    complex elastic structure
  \end{itemize}
  
\end{frame}


% ------------------------------------------------------------ SLIDE
\begin{frame}
  \frametitle{Green's Functions Examples}
  \summary{}

  \begin{itemize}
  \item 2-D examples: {\tt\color{red} examples/2d/greensfns}
    \begin{itemize}
    \item Example components
      \begin{enumerate}
      \item Compute synthetic (fake) observations for an earthquake
      \item Compute displacements at sites for Green's functions
      \item Invert for fault slip
      \end{enumerate}
    \item See Section 7.15 of the PyLith User Manual
    \end{itemize}
  \item 3-D example: {\tt\color{red} examples/3d/hex8/step21}
    \begin{itemize}
    \item Limited to computing displacements at sites for Green's functions
    \item No inversion
    \end{itemize}
  \end{itemize}
  
\end{frame}


% ========================================================== SECTION
\section{2-D Green's Fns}

% ------------------------------------------------------------ SLIDE
\begin{frame}
  \frametitle{2-D Green's Functions Example}
  \summary{Invert for slip on reverse fault}

  \vfill
  Files are in {\tt\color{red} examples/2d/greensfns/reverse}
  \vfill

  \begin{enumerate}
  \item Generate mesh using CUBIT
  \item Compute synthetic (fake) observations for an earthquake\\
    {\tt pylith eqsim.cfg}
  \item Compute displacements at sites for Green's functions\\
    {\tt pylith --problem=pylith.problems.GreensFns}
  \item Invert for fault slip\\
    See README in {\tt examples/2d/greensfns}
  \item Visualize inversion results using matplotlib Python package\\
    See README in {\tt examples/2d/greensfns}
  \end{enumerate}
  
\end{frame}


% ------------------------------------------------------------ SLIDE
\begin{frame}
  \frametitle{Python Packages Needed}
  \summary{}

  \begin{description}
  \item[numpy] Arrays plus scientific computing tools in Python
    \begin{itemize}
    \item Similar to numerical functions in Matlab
    \item Included in PyLith binary distribution
    \end{itemize}
  \item[h5py] Python wrappers for HDF5 library
    \begin{itemize}
    \item Provides high-level access to HDF5 files
    \item Included in PyLith binary distribution
    \end{itemize}
  \item[matplotlib] 2-D plotting for Python
    \begin{itemize}
    \item Designed to be very similar to Matlab
    \item {\bf Not} included in PyLith binary distribution
    \item Available from {\tt matplotlib.org}
    \end{itemize}
  \end{description}
  
\end{frame}


% ------------------------------------------------------------ SLIDE
\begin{frame}
  \frametitle{Simple Linear Inversion}
  \summary{Parameters}

  \begin{description}
  \item[\textit{G}] Green's function matrix
  \item[\textit{d}] Unknown fault slip
  \item[\begin{math}d_{apriori}\end{math}] A priori estimate of fault slip
  \item[\begin{math}u_{obs}\end{math}] Observed displacement
  \item[\textit{D}] Penalty matrix
  \item[\begin{math}\theta\end{math}] Penalty parameter
  \end{description}
  
  \vfill
  The matrix \begin{math}{G_{ij}}\end{math} gives displacement component \textit{i} due to a unit of slip from component \textit{j}.
  \vfill

\end{frame}


% ------------------------------------------------------------ SLIDE
\begin{frame}
  \frametitle{Simple Linear Inversion}
  \summary{Equations}

  \begin{itemize}
  \item Original system of equations:
    \begin{equation}
      G d = u_{obs}
    \end{equation}

  \item Augmented system of equations:
    \begin{equation}
      G_a d = u_a \text{, where } 
      G_a = \left[ \begin{array}{c} G \\ \theta D \end{array} \right]
      \text{ and }
      u_a = \left[ \begin{array}{c} u_{obs} \\ d_{apriori} \end{array} \right]
    \end{equation}
    
  \item Generalized inverse:
    \begin{gather}
      G^{-g} = \left( G_a^T G_a \right)^{-1} G_a^T \\
      d_{est} = G^{-g} u_a
    \end{gather}
  \end{itemize}
  
\end{frame}


% ========================================================== SECTION
\section{3-D Green's Fns}

% ------------------------------------------------------------ SLIDE
\begin{frame}
  \frametitle{3-D Green's Functions Example}
  \summary{Demonstrate computing Green's functions; no inversion}

  \vfill
  Files are in {\tt\color{red} examples/3d/hex8}
  \vfill

  \begin{itemize}
  \item Compute responses due to strike-slip and reverse slip separately
  \item Parameters are distributed across multiple {\tt .cfg} files
    \begin{description}
    \item[{\tt pylithapp.cfg}] General parameters for this mesh
    \item[{\tt greensfns.cfg}] General Green's function problem settings
    \item[{\tt step21.cfg}] Parameters specific to this example
    \end{description}
  \end{itemize}

  \vfill
  {\tt pylith --problem=pylith.problems.GreensFns step21.cfg}
  \vfill
  
\end{frame}




% ======================================================================
\end{document}


% End of file
